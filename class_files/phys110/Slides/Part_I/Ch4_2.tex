\documentclass[pdf, aspectratio=169]{beamer}
\usepackage[]{hyperref,graphicx,siunitx,lmodern,booktabs,tikz,caption}
\usepackage{pdfpc-commands}

\usepackage[mode=image]{standalone}
\mode<presentation>{\usetheme{Astro}}

\graphicspath{ {../Images/} }

\sisetup{per-mode=symbol}
\usetikzlibrary{calc,angles,quotes}

\newcommand{\moon}[1] {
  \coordinate (moon) at (#1:4);
  \node[transform shape] at (moon) {\includegraphics[width=0.8cm]{ch2_moon_small2.png}};
  \fill[black, opacity=0.5] (moon) -- +(0,.4cm) arc (90:270:.39cm) -- cycle;
}
\newcommand{\moonphase}[1] {
  \coordinate (moon) at (-3.5,-3.5);
  \node[transform shape] at (moon) {\includegraphics[width=2cm]{ch2_moon_small2.png}};
  \fill[black, opacity=0.5] (moon) +(0,1cm) arc (90:270:1cm) arc (270:90:#1 cm and 1cm);
}

%preamble
\title{Total Eclipse of the Heart}
\date{September 14, 2018}
\author{Jed Rembold}

\begin{document}
\renewcommand*{\theenumi}{\Alph{enumi}}

\begin{frame}{Announcements}
	\begin{itemize}
		\item New WebWork posted. Due Monday morning.
		\item No physics tea today. SCRP talks in Ford 122.
		\item Test 1 two weeks from today
		\item Polling: \url{rembold-class.ddns.net}
	\end{itemize}
\end{frame}

\begin{frame}{News of the Day: Cassini no more}
  \begin{center}
	  \href{https://www.youtube.com/watch?list=PLTiv_XWHnOZpKPaDTVy36z0U8GxoiIkZa&v=xrGAQCq9BMU}{\includegraphics[width=\textwidth]{Cassini.png}}
  \end{center}
  
\end{frame}

%\begin{frame}{Review Question}
  %\begin{columns}
	%\column{.5\textwidth}
	%\begin{center}
	  %\begin{tikzpicture}
		%\node at (0,0) {\includegraphics[width=4cm]{ch2_moon_small2.png}};
		%\fill[rotate=-160, black, opacity=0.5] (0,1.98) arc (90:270:1.98cm) arc (270:90:8mm and 1.98cm);
	  %\end{tikzpicture}
	%\end{center}
	%\column{.5\textwidth}
	%Which of the following describes the phase of the moon to the left, assuming the image was take in Syndey, Australia?
	%\begin{enumerate}
	  %\item Waxing Gibbous
	  %\item Waning Gibbous
	  %\item First Quarter
	  %\item New
	%\end{enumerate}
  %\end{columns}
%\end{frame}

\begin{frame}{Review Question}
	Say an orbiting planet lost some of it's mass in a process that kept it's angular momentum constant. Potential effects could be:
	\begin{enumerate}
		\item The planet slowing down but staying the same distance away from the Sun
		\item The planet moving closer to the Sun but staying the same speed
		\item \alert<2>{The planet moving further from the Sun but staying the same speed}
		\item None of the above
	\end{enumerate}
\end{frame}

\begin{frame}{Our Wobbly Planet}
  \begin{columns}
	\column{.5\textwidth}
	\begin{itemize}
	  \item Just like a spinning top, Earth has a bit of a wobble
	  \item Relative to the celestial sphere, the direction of our North Pole oscillates every \SI{26000}{yrs}
	  \item ``North Star'' changes!
	\end{itemize}
	\column{.5\textwidth}
	\begin{figure}[h!]
	  \centering
	  \includegraphics[width=.9\textwidth]{ch2_pole_precess.jpg}
	\end{figure}
  \end{columns}
\end{frame}

\begin{frame}{Let's Talk the Moon}
  Two obvious observations we are interested in explaining:
  \begin{itemize}
	\item Why do we see the Moon going through phases?
	\item What causes eclipses?
  \end{itemize}
\end{frame}

\begin{frame}{Lunar Facts}
  \begin{itemize}
	\item Moon orbits Earth in \SI{27.3}{\day}
	\item Lunar Phases repeat every \SI{29.5}{\day}
	  \begin{itemize}
		\item Difference due to Earth moving about the sun during this time
	  \end{itemize}
	\item Sunlight hits Earth and Moon at same angle because the Sun is so far away
	\item Moon's phase is also closely related to its rising and setting times
	\item The Man in the Moon is always watching
	  \begin{itemize}
		\item We always see the same face of the moon
	  \end{itemize}
  \end{itemize}
\end{frame}

\begin{frame}{It's just a phase}
  \begin{center}
	\begin{tikzpicture}[scale=.6]
	  %\draw[help lines] (-5,-5) grid (5,5);
	  \node[transform shape] at (0,0) {\includegraphics[width=1.44cm]{world.png}};
	  \fill[black,opacity=0.5] (0,0) +(0,0.72) arc (90:270:0.72) -- cycle;
	  \draw[help lines, dashed] (0,0) circle (4cm);
	  \foreach \y in {-5,-4,...,5}{
		\draw[yellow, -latex] (6,\y) -- +(-1,0);
	  }
	  \fill[black,rounded corners] (-5,-5) rectangle (-2,-2);
	  \node at (-3.5,-5.25) {Observer View};
	  \onslide<1>{
		\moon{0};
		\moonphase{-1};
		\fill[red] (0:0.8) circle (1mm);
	  }
	  \onslide<2>{
		\moon{45};
		\moonphase{-.5};
		\fill[red] (90:0.8) circle (1mm);
	  }
	  \onslide<3>{
		\moon{90};
		\moonphase{-0};
		\fill[red] (90:0.8) circle (1mm);
	  }
	  \onslide<4>{
		\moon{135};
		\moonphase{.5};
		\fill[red] (180:0.8) circle (1mm);
	  }
	  \onslide<5>{
		\moon{180};
		\moonphase{1};
		\fill[red] (180:0.8) circle (1mm);
	  }
	\end{tikzpicture}
  \end{center}
\end{frame}

\begin{frame}{Wax on, Wane off}
  \begin{itemize}
	\item Waxing - the moon is becoming more full
	  \begin{itemize}
		\item Visible afternoon/evening
	  \end{itemize}
	\item Waning - the moon is becoming less full
	  \begin{itemize}
		\item Visible late night / morning
	  \end{itemize}
	\item In Northern Hemisphere:
	  \begin{itemize}
		\item Light on Right $\Rightarrow$ waxing
		\item Opposite in Southern Hemisphere
	  \end{itemize}
  \end{itemize}
  \begin{center}
	\begin{tikzpicture}[scale=.7, transform shape]
	  %\draw[help lines] (0,0) grid (12,4);
	  \node at (5.5,4.5) {\Large Waxing Moons};
	  \fill[black!80!blue, rounded corners] (0,0) rectangle +(3,4);
	  \fill[black!85!blue, rounded corners] (4,0) rectangle +(3,4);
	  \fill[black!90!blue, rounded corners] (8,0) rectangle +(3,4);
	  \node (north) at (1.5,2) {\includegraphics[width=2cm]{ch2_moon_small2.png}};
	  \node (equator) at (5.5,2) {\includegraphics[width=2cm]{ch2_moon_small2.png}};
	  \node (south) at (9.5,2) {\includegraphics[width=2cm]{ch2_moon_small2.png}};
	  \fill[black, opacity=0.7] (north) +(0,1) arc (90:270:1) arc (-90:90:0.5cm and 1cm);
	  \fill[black, opacity=0.7, rotate around={-90:(equator)}] (equator) +(0,1) arc (90:270:1) arc (-90:90:0.5cm and 1cm);
	  \fill[black, opacity=0.7, rotate around={180:(south)}] (south) +(0,1) arc (90:270:1) arc (-90:90:0.5cm and 1cm);
	  \node at (1.5,.5) {\SI{+45}{\degree N}};
	  \node at (5.5,.5) {Equator};
	  \node at (9.5,.5) {\SI{+45}{\degree S}};
	\end{tikzpicture}
  \end{center}
\end{frame}

\begin{frame}{Don't Turn Your Back!}
  Why do we only see one side of the moon?
  \begin{itemize}
	\item If the moon did not spin as it orbited us, we'd see all parts of the moon
	\item If the moon spun either quickly or slowly, we'd see all parts of the moon
	\item But if it spun \emph{jussst riighht\ldots}
  \end{itemize}
\end{frame}

\begin{frame}{Synchronous Spinning}
  \begin{center}
	\begin{tikzpicture}[scale=.9]
	  %\draw[help lines] (0,0) grid (10,6);
	  \coordinate (earth) at (5,3);
	  \node at (earth) {\includegraphics[width=1cm]{world.png}};
	  \draw[help lines, dashed] (earth) circle (3cm);
	  \foreach \t in {0,45,...,360}{
		\begin{scope}[rotate around={\t:(earth)}, transform shape]
		  \coordinate (moon) at (8,3);
		  \node at (moon) {\includegraphics[width=1cm]{ch2_moon_small2.png}};
		  \draw[red] (moon) +(-.6,0) node {N} +(.6,0) node {F};
		  \draw[red,-latex] (moon) +(.3,0) arc (0:180:.3);
		\end{scope}
	  }
	\end{tikzpicture}
  \end{center}
\end{frame}

\begin{frame}{Libration!}
  In truth, we actually can see upwards of 60\% of the Moon's surface over a year
  \begin{itemize}
	\item \alert{Libration} describes this oscillating wobble
	\item Libration in longitude due to the Moon's orbit not being perfectly circular
	\item Libration in latitude due to the Moon's tilt (similar to Earth's seasons!)
	\item Diurnal libration do to an observer getting two different perspectives from each side of the Earth daily
  \end{itemize}
\end{frame}

\begin{frame}{The Extra Bits}
  \begin{figure}[h!]
	\centering
	\includegraphics[width=.7\textwidth]{ch2_libration.jpg}
  \end{figure}
\end{frame}

\begin{frame}{Understanding Check}
  Imagine you built yourself a lovely house on the surface of the moon. One day you are talking to your mom back on Earth and she mentions that it is a lovely full moon tonight. From your vantage point on the moon, what is Earth's phase?
  \begin{enumerate}
	\item \alert<2>{New}
	\item Waxing
	\item Full
	\item Waning
  \end{enumerate}
\end{frame}

\begin{frame}{Eclipse Reminder}
  \begin{center}
	\inlineMovie{../Videos/LunarEclipse.ogv}{../Videos/LunarEclipse.png}{width=.8\textwidth}
  \end{center}
\end{frame}

\begin{frame}{Lunar Eclipse Facts}
  \begin{itemize}
	\item Lunar eclipses can \emph{only} occur when the Moon is full!
	  \begin{itemize}
		\item Otherwise the Earth isn't between the Moon and the Sun
	  \end{itemize}
	\item Lunar eclipses can be penumbral, partial, or total, depending on what part of the shadow overlaps the moon
	\item Note that anyone on Earth sees the same eclipsed Moon (assuming its night)
  \end{itemize}
  \begin{center}
	\begin{tikzpicture}
	  \node[anchor=south west] at (0,0) {\includestandalone[width=\textwidth]{../Images/Standalone/ch2_eclipse_fig}};
	  \onslide<1>{
		\node at (12.5,2.8) {\includegraphics[width=3mm]{ch2_moon_small2.png}};
		\node at (5,0) {Penumbral Eclipse};
	  }
	  \onslide<2>{
		\node (moon) at (12.5,2.5) {\includegraphics[width=3mm]{ch2_moon_small2.png}};
		\fill[black!70!red, opacity=0.5] (moon) ++(-.15,0) arc (180:350:1.5mm) -- cycle;
		\node at (5,0) {Partial Eclipse};
	  }
	  \onslide<3>{
		\node (moon) at (12.5,2.0) {\includegraphics[width=3mm]{ch2_moon_small2.png}};
		\fill[black!50!red, opacity=0.5] (moon) circle (1.5mm);
		\node at (5,0) {Total Eclipse};
	  }
	  %\draw[white] (0,0) grid (10,3);
	\end{tikzpicture}
  \end{center}
\end{frame}

\begin{frame}{Solar Eclipses}
  \begin{center}
	\inlineMovie{../Videos/SolarEclipse.ogv}{../Videos/SolarEclipse.png}{width=.8\textwidth}
  \end{center}
\end{frame}

\begin{frame}{Solar Eclipses Facts}
  \begin{itemize}
	\item Moon casting shadow onto the Earth
	\item Solar eclipses can only occur when the Moon is new!
	\item Solar eclipses can be partial, total, or annular
	\item Note that only select portions of the Earth observe the eclipsed Sun!
  \end{itemize}
  \begin{center}
	\includestandalone[width=\textwidth]{../Images/Standalone/ch2_eclipse_sun}
  \end{center}
\end{frame}

\begin{frame}{Why not eclipses every month?}
  \begin{itemize}
	\item The Moon's orbit is tilted \SI{5}{\degree} to the ecliptic
	\item Results in two seasons each year when eclipses can happen
  \end{itemize}
  \begin{center}
	\begin{tikzpicture}
	  \coordinate (earth) at (-4,0);
	  \coordinate (rec) at (2,-4.5);
	  %\draw[help lines] (-5,-3) grid (5,3);
	  \draw[orange, thick] (-5,0) -- (5,0);
	  \fill[inner color=yellow, outer color=orange] (0,0) circle (5mm);
	  \onslide<1>{
		\draw[purple, thick] (earth) +(175:1cm) node {\includegraphics[width=2mm]{ch2_moon_small2.png}} -- ++(-5:1cm);
	  }
	  \onslide<2>{
		\draw[purple, thick] (earth) ellipse (1cm and 3mm);
		\node at ($ (earth) - (1,0) $) {\includegraphics[width=2mm]{ch2_moon_small2.png}};
		\draw[green, ultra thick,-latex] ($ (earth)-(0,1) $) node[below] {Eclipse Points} -- +(135:.9);
		\draw[green, ultra thick,-latex] ($ (earth)-(0,1) $) -- +(45:.9);
	  }
	  \onslide<3>{
		\draw[purple, thick] (earth) +(5:1cm) node {\includegraphics[width=2mm]{ch2_moon_small2.png}} -- ++(185:1cm);
	  }
	  \onslide<4>{
		\draw[purple, thick] (earth) ellipse (1cm and 3mm);
		\node at ($ (earth) - (1,0) $) {\includegraphics[width=2mm]{ch2_moon_small2.png}};
		\draw[green, ultra thick,-latex] ($ (earth)-(0,1) $) node[below] {Eclipse Points} -- +(135:.9);
		\draw[green, ultra thick,-latex] ($ (earth)-(0,1) $) -- +(45:.9);
	  }
	  \node at (earth) {\includegraphics[width=5mm]{world.png}};
	  \fill[rounded corners,black] (rec) rectangle +(4,4);
	  \node[above right] at (rec) {Top View};
	  \onslide<1>{
		\coordinate (center) at ($ (rec)+(2,2) $);
		\coordinate (e) at ($ (center)+(180:1.25) $);
		\coordinate (m) at ($ (e)+(180:5mm) $);
		\draw[inner color=yellow, outer color=orange] (center) circle (1mm);
		\draw[orange] (center) circle(1.25cm);
		\node at (e) {\includegraphics[width=2mm]{world.png}};
		\draw[purple] (e) circle (5mm);
		\node at (m) {\includegraphics[width=1mm]{ch2_moon_small2.png}};
	  }
	  \onslide<2>{
		\coordinate (center) at ($ (rec)+(2,2) $);
		\coordinate (e) at ($ (center)+(270:1.25) $);
		\coordinate (m) at ($ (e)+(270:5mm) $);
		\draw[inner color=yellow, outer color=orange] (center) circle (1mm);
		\draw[orange] (center) circle(1.25cm);
		\node at (e) {\includegraphics[width=2mm]{world.png}};
		\draw[purple] (e) circle (5mm);
		\node at (m) {\includegraphics[width=1mm]{ch2_moon_small2.png}};
	  }
	  \onslide<3>{
		\coordinate (center) at ($ (rec)+(2,2) $);
		\coordinate (e) at ($ (center)+(0:1.25) $);
		\coordinate (m) at ($ (e)+(180:5mm) $);
		\draw[inner color=yellow, outer color=orange] (center) circle (1mm);
		\draw[orange] (center) circle(1.25cm);
		\node at (e) {\includegraphics[width=2mm]{world.png}};
		\draw[purple] (e) circle (5mm);
		\node at (m) {\includegraphics[width=1mm]{ch2_moon_small2.png}};
	  }
	  \onslide<4->{
		\coordinate (center) at ($ (rec)+(2,2) $);
		\coordinate (e) at ($ (center)+(90:1.25) $);
		\coordinate (m) at ($ (e)+(90:5mm) $);
		\draw[inner color=yellow, outer color=orange] (center) circle (1mm);
		\draw[orange] (center) circle(1.25cm);
		\node at (e) {\includegraphics[width=2mm]{world.png}};
		\draw[purple] (e) circle (5mm);
		\node at (m) {\includegraphics[width=1mm]{ch2_moon_small2.png}};
	  }
	\end{tikzpicture}
  \end{center}
  \onslide<5>{
	\vspace{-4.5cm}
	\begin{columns}
	  \column{.6\textwidth}
	  \begin{alertblock}{Eclipse Conditions}
		An eclipse can only occur when:
		\begin{itemize}
		  \item There is a full moon (lunar eclipse) or new moon (solar eclipse)
		  \item The Moon is at or near one of its two ecliptic crossing points
		\end{itemize}
	  \end{alertblock}
	  \column{.4\textwidth}
	\end{columns}
  }
\end{frame}

\begin{frame}{The Saros Cycle}
  \begin{itemize}
	\item Ecliptic crossing points slowly move around the Moon's orbit
	  \begin{itemize}
		\item Put's potential eclipse seasons about 173 days apart
	  \end{itemize}
	\item Combine with lunar cycle to get one eclipse about every 18 years, 11 days, and 8 hours.
	  \begin{itemize}
		\item Called the \alert{Saros Cycle}
	  \end{itemize}
	\item Can predict \emph{when} an eclipse will occur but not where or what type
  \end{itemize}
  \begin{figure}[h!]
	\centering
	\includegraphics[width=.5\textwidth]{ch3_saros.jpg}
  \end{figure}
\end{frame}

\end{document}

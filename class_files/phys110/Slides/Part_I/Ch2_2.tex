\documentclass[pdf, aspectratio=169]{beamer}
\usepackage[]{hyperref,graphicx,siunitx,lmodern,booktabs,tikz,wasysym,caption}
\usepackage{pdfpc-commands}
\usepackage[mode=buildnew]{standalone}
\mode<presentation>{\usetheme{Astro}}

\sisetup{per-mode=symbol}
\usetikzlibrary{calc,angles,quotes}

\graphicspath{ {../Images/} }

%preamble
\title{Historic Sizing}
\date{August 31, 2018}
\author{Jed Rembold}

\begin{document}
\renewcommand*{\theenumi}{\Alph{enumi}}

\begin{frame}{Announcements}
	\begin{itemize}
		\item New WebWorK posted. Will be due on \emph{Wednesday} at 10am
		\item No class on Monday!
		\item No lab next week!
	  \item Poll: \url{rembold-class.ddns.net}
	\end{itemize}
\end{frame}

\begin{frame}{Back to APOD}
	\begin{center}
		\includegraphics[width=.9\textwidth]{APOD_Goat.jpg}
	\end{center}
\end{frame}

\begin{frame}{Review Question}
 	On the celestial sphere, the latitude (declination) of the star Vega is list as \ang{38; 47;7}. What decimal angle does this correspond to?
	\begin{enumerate}
		\item \ang{3.318}
		\item \ang{38.477}
		\item \SI{38.477}{\hour}
		\item \alert<2>{\ang{38.785}}
	\end{enumerate}
\end{frame}

\begin{frame}{Celesial Sphere Demonstrations}
  \begin{example}
	\begin{itemize}
	  \item How long is the sun up today here on the 45th parallel?
	  \item How long is the sun up today up in Alaska on the 65th parallel?
	  \item How high will the sun rise in the sky for us on December 10th?
	  \item What time will Orion rise on October 31?
	  \item What altitude and azimuth will Vega have at midnight tonight?
	\end{itemize}
  \end{example}
\end{frame}


\begin{frame}{A Historical Perspective on Science}
  \begin{itemize}
	\item Scientific reasoning is based on using ideas of observation and trial-and-error to inform and gain understanding
	\item Astronomy is among the oldest of the sciences
	\item Generally used for very practical purposes
	  \begin{itemize}
		\item Tracking time and the seasons
		  \begin{itemize}
			\item Agricultural purposes
			\item Religious purposes
		  \end{itemize}
		\item Aiding in Navigation
	  \end{itemize}
  \end{itemize}
\end{frame}

\begin{frame}{Ancient Astronomic Uses}
  \begin{figure}[h!]
	\centering
	\includegraphics[width=\textwidth]{ch3_rainyseasons.jpg}
	\caption*{The ancient people of central Africa (6500 BC) could predict rainy seasons from the orientation of the Moon.}
  \end{figure}
\end{frame}

\begin{frame}{Modern Holdovers}
  The days of the week are named after the major 7 visible heavenly objects visible to the ancients!
  \begin{table}[h!]
	\centering
	\begin{tabular}{llll}
	  \toprule
	  Object & English & French & Spanish \\
	  \midrule
	  Sun & Sunday & Dimanche & Domingo \\
	  Moon & Monday & Lundi & Lunes \\
	  Mars & Tuesday & Mardi & Martes \\
	  Mercury & Wednesday & Mercredi & Mi\'{e}rcoles \\
	  Jupiter & Thursday & Jeudi & Jueves \\
	  Venus & Friday & Vendredi & Viernes \\
	  Saturn & Saturday & Samedi & S\'{a}bado \\
	  \bottomrule
	\end{tabular}
  \end{table}
\end{frame}

\begin{frame}{Astronomic Achievements of the Ancients}
  \begin{columns}
	\column{.5\textwidth}
	\begin{itemize}
	  \item<1-> Daily Timekeeping
	  \item<2-> Tracking the seasons
	  \item<3-> Monitoring lunar cycles
	  \item<4-> Monitoring planets and stars
	  \item<5-> Predicting eclipses
	\end{itemize}
	\column{.5\textwidth}
	\begin{center}
	  \includegraphics<1>[width=\textwidth]{ch3_obelisk.jpg}
	  \includegraphics<2>[width=\textwidth]{ch3_solsticedagger.jpg}
	  \includegraphics<3>[width=\textwidth]{ch3_stonehenge.jpg}
	  \includegraphics<4>[width=\textwidth]{ch3_machupichu.jpg}
	  \includegraphics<5>[width=\textwidth]{ch3_nazcalines.jpg}
	\end{center}
  \end{columns}
\end{frame}

\begin{frame}{Tis all Greek}
  \begin{itemize}
	\item Ancient Greece located at a crossroads for the exchange of knowledge
	  \begin{itemize}
		\item Much of our math and science heritage heralds from the Middle East.
	  \end{itemize}
  \end{itemize}
  \begin{figure}[h!]
	\centering
	\includegraphics[width=.8\textwidth]{ch3_macedonia.jpg}
  \end{figure}
\end{frame}

\begin{frame}{The Essence of Modern Science}
  \begin{columns}
	\column{.5\textwidth}
	\begin{itemize}
	  \item Ancient Greeks were the first known people to make \alert{models} of nature.
	  \item Tried to explain patterns in nature without resorting to myth or spiritual
	  \item Explanations had to agree with observations
	\end{itemize}
	\column{.5\textwidth}
	\begin{figure}[h!]
	  \centering
	  \includegraphics[width=.7\textwidth]{ch3_greekmodel.jpg}
	  \caption*{Greek interpretation of the geocentric model of the solar system}
	\end{figure}
  \end{columns}
\end{frame}

\begin{frame}{Eratosthenes}
  The first accurate estimate of the Earth's circumference
  \begin{center}
	\begin{tikzpicture}
	  \coordinate (earth) at (0,0);
	  \coordinate (sun1) at ($ (earth)+(0:2cm) $);
	  \coordinate (sun2) at ($ (earth)+(20:2cm) $);
	  \node at (earth) {\includegraphics[width=4cm]{world.png}};
	  \draw[latex-, yellow, thick] (sun1) -- +(2,0) node[below right,align=left] {Sun overhead\\in Syene};
	  \draw[latex-, yellow, thick] (sun2) -- +(2,0) node[above right, align=left] {Sun in \\Alexandria} node (sun3) {};
	  \draw[purple, dashed, thick] (sun2) -- +(20:2cm) node (zenith) {};
	  \draw pic["\SI{7}{\degree}", draw=orange, <->, angle radius=8mm, angle eccentricity=1.5, color=orange] {angle=sun3--sun2--zenith};
	  \onslide<2->{
		\draw[thick, dashed, blue] (sun1)--(earth)--(sun2);
		\draw pic["\SI{7}{\degree}", color=blue,draw=blue, <->, angle radius=1cm, angle eccentricity=1.5] {angle=sun1--earth--sun2};
	  }
	\end{tikzpicture}
  \end{center}
\end{frame}

\begin{frame}{Eratosthenes Work}
  \begin{itemize}
	\item Straight line distance between Syene and Alexandria $\approx\SI{5000}{stadia}$
	\item Can set up a proportion between angular and measured distances
	  \[\frac{\text{Part}}{\text{Whole}} = \frac{\SI{7}{\degree}}{\SI{360}{\degree}} = \frac{\SI{5000}{stadia}}{\text{Circumference}}\]
	\item So the circumference equaled:
	  \[\text{Circumference} = \SI{257143}{stadia}\]
	\item One stadia is approximately \SI{0.167}{\kilo\meter}, so:
	  \[\text{Circumference} = \SI{42857}{\kilo\meter}\]
	\item This is remarkably close to the actual value of \SI{40075}{\kilo\meter}!
  \end{itemize}
\end{frame}

\begin{frame}{Putting it to Practice}
  \begin{example}
	Imagine you and a fellow astronaut found yourself stranded on a strange planet. You synchronize your watches, and then he or she walks approximately \SI{500}{\kilo\meter} due South. Then, at the same time, you both take pictures of the sky before meeting back up to compare. Given your images and how far your friend traveled, what planet are you likely on?
  \end{example}
\end{frame}

\begin{frame}{Your Images}
  \vspace{-5mm}
  \begin{columns}
	\column{.5\textwidth}
	\begin{figure}[h!]
	  \centering
	  \includegraphics[width=.7\textwidth]{ch3_Pos1.pdf}
	  \caption*{Your image}
	\end{figure}
	\column{.5\textwidth}
	\begin{figure}[h!]
	  \centering
	  \includegraphics[width=.7\textwidth]{ch3_Pos2.pdf}
	  \caption*{Your friend's image}
	\end{figure}
  \end{columns}
\end{frame}

\begin{frame}{Your Calculations}
  \begin{itemize}
	\item The sun is out, so you probably can't see the stars, but you could have chosen any point to measure from
	\item The angular separation seems to be about \SI{8}{\degree}
	  \[\frac{\SI{8}{\degree}}{\SI{360}{\degree}} = \frac{\SI{500}{\kilo\meter}}{\text{Circumference}}\]
	\item Thus the planets circumference is approximately \SI{22500}{\kilo\meter}
	\item Corresponds to a radius of approximately \SI{3600}{\kilo\meter}
	\item Checking a table, it would seem you are most likely on Mars!
  \end{itemize}
\end{frame}

\begin{frame}{A Starry Backdrop}
  \begin{itemize}
	\item Because stars are so far away, any movement they have is too small to be seen, thus they appear fixed
	\item Closer objects (like planets) do have large enough movements to be seen
	\item Easiest to measure these visible movements against the ``fixed'' starry backdrop
	\item One of the earliest methods of tracking planets
  \end{itemize}
\end{frame}

\begin{frame}{Mars' Movement}
  \begin{center}
	\begin{tikzpicture}
	  \node[anchor=south west] at (0,0) {\includegraphics[width=8cm]{ch3_Mars3.png}};
	  \node[anchor=south west] at (0,2.3) {\includegraphics[width=8cm]{ch3_Mars2.png}};
	  \node[anchor=south west] at (0,4.6) {\includegraphics[width=8cm]{ch3_Mars1.png}};
	  \node[anchor=west] at (0.2,5.1) {August 25th};
	  \node[anchor=west] at (0.2,2.8) {September 4th};
	  \node[anchor=west] at (0.2,.5) {September 17th};
	\end{tikzpicture}
  \end{center}
\end{frame}

\begin{frame}[fragile]{The Geocentric Model}
  \begin{center}
	\begin{tikzpicture}
	  \coordinate (o) at (0,0);
	  \newcommand{\orbit}[4]{
		\draw[dashed, #1, -latex] (#2,0) arc (0:90:#2cm);
		\fill[#1] (#2,0) circle (#3mm);
		\node[below, yshift=-2mm] at (#2,0) {\small #4};
	  }
	  \node at (o) {\includegraphics[width=1cm]{world.png}};
	  \orbit{red!50}{1}{1}{\mercury};
	  \orbit{yellow!50}{2}{1}{\venus};
	  \orbit{yellow}{3}{3}{$\odot$};
	  \orbit{black!30!red}{4}{1}{\mars};
	  \orbit{red}{5}{2}{\jupiter};
	  \orbit{black!30!yellow}{6}{1.5}{\saturn};
	\end{tikzpicture}
  \end{center}
\end{frame}

\begin{frame}{The Main Complication}
  \begin{columns}
	\column{.5\textwidth}
	\begin{itemize}
	  \item Retrograde Motion
		\begin{itemize}
		  \item For parts of the year planets move \alert{backwards} relative to their normal paths
		  \item More obvious for some planets than others
		  \item This posed a major hurtle in trying to explain the motion of the planets
		\end{itemize}
	\end{itemize}
	\column{.5\textwidth}
	\begin{center}
	  \inlineMovie{../Videos/Retrograde.ogv}{../Videos/Retrograde.png}{width=.8\textwidth}
	\end{center}
  \end{columns}
\end{frame}


\begin{frame}[fragile]{A Corrected Geocentric Model: Ptolemy}
  \begin{center}
	\begin{tikzpicture}
	  \newcommand{\orbit}[3]{
		\draw[dashed, #1, -latex] (#2,0) arc (0:90:#2cm);
		\coordinate (p) at (#3:#2);
		\draw[dotted, #1, -latex] (p) +(0:4mm) arc(0:340:4mm);
		\fill[#1] (p) +(0:4mm) circle(1mm);
	  }
	  \node at (o) {\includegraphics[width=1cm]{world.png}};
	  \orbit{red!50}{1}{0};
	  \orbit{yellow!50}{2}{20};
	  \orbit{yellow}{3}{30};
	  \orbit{black!10!red}{4}{40};
	  \orbit{red}{5}{50};
	  \orbit{black!10!yellow}{6}{60};
	\end{tikzpicture}
  \end{center}
\end{frame}

\begin{frame}{The Ptolemaic Model ($\approx$ 150 AD)}
  \begin{itemize}
	\item The basic Aristotle idea of the geocentric model is very simple
	\item In practice though, the Ptolemaic model is actually extremely complex
	  \begin{itemize}
		\item Some circles larger than others
		\item Some circles slightly rotating off-center
	  \end{itemize}
	\item It DID, however, quite accurately predict the observed motion of the planets
  \end{itemize}
\end{frame}

\begin{frame}{Enter Copernicus (1473-1543)}
  \begin{itemize}
	\item Found that the planetary motion tables based on Ptolemaic model were growing more inaccurate
	\item Disliked the complexity of Ptolemaic model on aesthetic grounds
	\item Adopted a \alert{heliocentric} model with the Sun at the center
	  \begin{itemize}
		\item Naturally explains retrograde motion
		\item Naturally explained Mercury and Venus never being a large angle from the Sun
	  \end{itemize}
	\item The Heliocentric model was \alert{not} significantly more accurate than the Ptolemaic model at the time.
  \end{itemize}
\end{frame}

\begin{frame}{Objection! The Church}
  \begin{quote}
	There is talk of a new astrologer who wants to prove that
the earth moves and goes around, instead of the sky,
the sun, the moon, just as if somebody were moving in a
carriage or ship might hold that he was sitting still and at
rest while the earth and the trees walked and moved.
But that is how things are nowadays: when a man
wishes to be clever he must needs invent something
special, and the way he does it must needs be the best!
The fool wants to turn the whole art of astronomy
upside-down. However, as Holy Scripture tells us, so did
Joshua bid the sun to stand still and not the earth. 
  \end{quote}
  \begin{flushright}
	-Martin Luther
  \end{flushright}
\end{frame}

\begin{frame}{Objection! Science}
  \begin{columns}
	\column{.5\textwidth}
	\begin{itemize}
	  \item Why aren't we dizzy?
	  \item Why are we the only planet with a moon?
	  \item Why don't we see stellar parallax?
	\end{itemize}
	\column{.5\textwidth}
	\begin{center}
	  \begin{tikzpicture}[scale=0.9]
		\draw[inner color=yellow, outer color=orange] (0,0) circle(2mm);
		\draw[dashed] (0,0) circle (1.5cm);
		\draw[dashed] (0,0) circle (3cm);
		\coordinate (e2) at (330:1.5);
		\coordinate (e1) at (180:1.5);
		\node at (e1) {\includegraphics[width=2mm]{world.png}};
		\node at (e2) {\includegraphics<3->[width=2mm]{world.png}};
		\foreach \a in {0,20,...,340}{
		  \node (\a) at (\a:3) {$\bigstar$};
		}
		\fill<2->[cyan, fill opacity=0.4] (180)--(e1)--(220);
		\fill<3->[orange, fill opacity=0.4] (180)--(e2)--(220);
	  \end{tikzpicture}
	\end{center}
  \end{columns}
\end{frame}

\end{document}

\documentclass[pdf,aspectratio=169]{beamer}
\usepackage[]{hyperref,graphicx,siunitx,lmodern,booktabs,tikz,tensor}
\usepackage{pgfplots, pgfplotstable}
\usepackage{pdfpc-commands}
\usepackage[mode=buildnew]{standalone}
\mode<presentation>{\usetheme{Astro}}

\graphicspath{ {../Images/} }

\sisetup{per-mode=symbol}
\usetikzlibrary{calc,intersections, decorations.pathmorphing,shadings}
\tikzstyle{image}=[inner sep=0pt, outer sep=0pt, anchor=south west]
%\tikzstyle{proton}=[circle, minimum size = 7mm, ball color=red, black,transform shape]
%\tikzstyle{neutron}=[circle, minimum size=7mm, ball color=gray, black, transform shape]
%\tikzstyle{gammaray}=[ultra thick, -latex, decorate, decoration={snake, post length=3mm}]


%preamble
\title{Pu\ldots ul\ldots se\ldots ars\ldots}
\date{November 9, 2018}
\author{Jed Rembold}

\begin{document}
\renewcommand*{\theenumi}{\Alph{enumi}}

\begin{frame}{Announcements}
  \begin{itemize}
	\item WebWork due on Monday
	\item Test 3 coming up a week from today!
	  \begin{itemize}
		  \item Should finish testable material on Monday
		\item I'll get study questions and old tests up today or tomorrow
	  \end{itemize}
  \item Lab Group B has lab next Monday
	\item Polling: \url{rembold-class.ddns.net}
  \end{itemize}
\end{frame}

\begin{frame}{Old News!}
	\begin{itemize}
		\item One of the oldest stars ever found spotted!
			\begin{itemize}
				\item Tiny red dwarf here in the Milky Way
				\item Estimated at 13.5 billions years old
				\item Extremely low metallicity
				\item Proof that small stars could form in the early universe
			\end{itemize}
		\item Correction/Clarification
			\begin{itemize}
				\item In high mass stars, as they burn through each new fusion process, they never reach the electron degeneracy limit
				\item Massive enough that the core gets hot and dense enough to start the next round of fusion before the star reaches that point
			\end{itemize}
	\end{itemize}
\end{frame}

%\begin{frame}{Warm Up Question}
  %Suppose that gravity is actually a million times as strong as it currently is (so say $G=\num{6.67e-5}$ instead of $\num{6.67e-11}$). All other things remaining the same, how would this effect the frequency of supernovas?
  %\begin{enumerate}
	%\item Supernovas would no longer occur at all
	%\item Supernovas would be less common than they currently are, but still occur
	%\item Supernovas would occur at about the same frequency they currently do
	%\item \alert<2>{Supernovas would be more common than they currently are}
  %\end{enumerate}
%\end{frame}

\begin{frame}{Warm Up}
	When a low mass star ceases to have fusion reactions in its core, what is the core comprised of?
	\begin{enumerate}
		\item Hydrogen
		\item Helium
		\item \alert<2>{Carbon}
		\item Iron
	\end{enumerate}
\end{frame}

\begin{frame}{Our Onion Star}
  At this point we have multiple layers of shell fusion happening throughout the star
  \begin{center}
	\begin{tikzpicture}[scale=.8, lbl/.style={minimum width=2cm, minimum height=.75cm, anchor=west}]
	  \shade[inner color=yellow, outer color=Background, even odd rule] (0,0) circle (2cm) (0,0) circle (4cm);
	  \fill[orange!50!yellow] (0,0) circle (3.5cm);
	  \fill[red] (0,0) circle (2cm);
	  \fill[red!50!black] (0,0) circle (1cm);
	  \fill[blue!70!black] (0,0) circle (5mm);
	  \fill[green!50!black] (0,0) circle (2mm);
	  \fill[gray!50!black] (0,0) circle (1mm);

	  \node[fill=orange!50!yellow,lbl, text=black] at (4.5,2.5) {Hydrogen};
	  \node[fill=red,lbl, text=black] at (4.5,1.5) {Helium};
	  \node[fill=red!50!black, lbl] at (4.5,.5) {Carbon};
	  \node[fill=blue!70!black, lbl] at (4.5,-.5) {Oxygen};
	  \node[fill=green!50!black, lbl, text=black] at (4.5,-1.5) {Silicon};
	  \node[fill=gray!50!black, lbl] at (4.5,-2.5) {Iron};
	\end{tikzpicture}
  \end{center}
\end{frame}

\begin{frame}{Iron: The Great Killjoy \scriptsize Or is it!?}
  \begin{columns}
	\column{.5\textwidth}
	\begin{center}
	  \begin{tikzpicture}
		\begin{axis}[
			width=6.5cm,
			height=7cm,
			xlabel = Mass Number (Neutrons + Protons),
			ylabel = Binding Energy (MeV),
			ylabel near ticks,
		  ]
		  \addplot[scatter] table {../Data/BindingEnergy.csv};
		  \node[pin=-90:Iron] at (axis cs: 56,8.5) {};
		  \node[pin=45:Hydrogen] at (axis cs: 1,1) {};
		  \node[pin=250:Uranium] at (axis cs: 238,7.3) {};
		\end{axis}
	  \end{tikzpicture}
	\end{center}
	\column{.5\textwidth}
	\begin{itemize}
	  \item Fusing elements greater than iron \alert{uses} energy!
	  \item Iron also sucks up loose electrons running around
		\begin{itemize}
		  \item Decreases the internal electron degeneracy pressure
		\end{itemize}
	  \item Basically, once the star hits iron, it's game over
		\begin{itemize}
		  \item At least for fusion
		  \item For other things, it's party time
		\end{itemize}
	\end{itemize}
  \end{columns}
\end{frame}

\begin{frame}{Party Time!}
  \begin{itemize}
	\item Impossible for the core to contract enough to fuse iron and give off energy
	\item Can electron degeneracy stop the collapse? (Like with brown and white dwarfs?)
	  \begin{itemize}
		\item NOPE! Too much mass!
		\item Physics laws can't be violated:
		  \begin{itemize}
			\item All electrons combine with protons to make neutrons + neutrinos
		  \end{itemize}
		\item Pure neutron core continues to collapse
		\item Can neutron degeneracy pressure stop the collapse?
		  \begin{itemize}
			\item Maybe!
		  \end{itemize}
	  \end{itemize}
	\item The collapse is incredibly violent
	\item Neutrinos from the creation of the neutrons surge outward with ridiculous energies
	\item So dense and so many that they actually interact with the surrounding gas, exploding it outward
  \end{itemize}
\end{frame}

\begin{frame}{Supernova Details}
  \begin{itemize}
	\item Core with mass of Sun and size of Earth contracts
	  \begin{itemize}
		\item Final size? A few kilometers across
		\item Time to collapse? A fraction of a second
	  \end{itemize}
	\item Radiates 100 times the energy that the Sun will radiate \alert{over its entire 10 billion year lifetime}
	\item The immense heat excites the gas that is exploding outward, causing it to shine
	\item The insane pressure and energy also give elements heavier than iron a chance to form
	  \begin{itemize}
		\item Without supernovae, we wouldn't exist!
	  \end{itemize}
  \end{itemize}
\end{frame}

\fullFrameMovie{../Videos/Crab_Explosion.ogv}{../Videos/Crab_Explosion.png}

\begin{frame}{Supernova Prettiness}
  \begin{center}
	\begin{tikzpicture}[image/.style={inner sep=0pt, outer sep=0pt, anchor=south west}]
	  \node[image] at (0,0) {\includegraphics[width=6cm]{ch13_crab.jpg}};
	  \node[image] at (5.5,1) {\includegraphics[width=5cm]{ch13_SN1994D.jpg}};
	  \node[image, anchor=north west] at (10,2.5) {\includegraphics[width=4cm]{ch13_cassA.jpg}};
	\end{tikzpicture}
  \end{center}
\end{frame}

\begin{frame}{Binary Complications}
  \begin{itemize}
	\item Happy orbit each other during main sequence
	\item When one goes Giant, suddenly the other can start stealing its mass!
	\item Accelerates the life-cycle of the thief
  \end{itemize}
  \begin{center}
	\includegraphics[width=.5\textwidth]{ch13_mass_transfer.jpg}
  \end{center}
\end{frame}


\begin{frame}{Final Resting Places}
  Depending on their mass, stars will end up in one of three graves:
  \begin{center}
	\begin{tikzpicture}
	  \foreach \p/\lab in {0/{White\\Dwarf}, 3.5/{Neutron\\Star}, 7/{Black\\Hole}}{
		\begin{scope}[xshift=\p cm, text=black]
		  \node[image, anchor=center] at (0,0) {\includegraphics[width=4cm]{ch14_grave.pdf}};
		  \node[align=center, yshift=4mm, xshift=1mm, font=\Large] at (0,0) {\lab};
		\end{scope}
	  }
	\end{tikzpicture}
  \end{center}
  We want to take some time to look at each in more depth!
\end{frame}

\begin{frame}{White Dwarf Refresher}
  \begin{columns}
	\column{.5\textwidth}
	\begin{itemize}
	  \item Remnants of lower mass stars
	  \item Never got hot enough to fuse carbon
		\begin{itemize}
		  \item Thus largely comprised of carbon
		\end{itemize}
	  \item Supported by electron degeneracy pressure
	  \item Roughly Earth sized
	  \item Usually have about the mass of the Sun
	\end{itemize}
	\column{.5\textwidth}
	\begin{center}
	  \includegraphics[width=\textwidth]{ch14_SiriusB_and_Earth.jpg}
	\end{center}
  \end{columns}
\end{frame}

\begin{frame}{Fat Little Dwarfs}
  \begin{columns}
	\column{.5\textwidth}
	\begin{itemize}
	  \item Increasing a white dwarf's mass actually causes it to shrink
	  \item Increased gravity increases faster than electron degeneracy can keep up
	  \item Increasing electron degeneracy pressure requires the electrons to move around faster
	  \item What about as this approaches the speed of light?
		\begin{itemize}
		  \item Chandrasekhar Limit
		\end{itemize}
	\end{itemize}
	\column{.5\textwidth}
	\begin{center}
	  \includegraphics[width=.9\textwidth]{ch14_chandra_limit.png}
	\end{center}
  \end{columns}
\end{frame}

\begin{frame}{Best Friends}
  \begin{itemize}
	\item It is completely possible for a white dwarf to be part of a binary system
	\item Can start siphoning mass from its neighbor
	  \begin{itemize}
		\item In-falling material flattens out into an accretion disk
		\item Precious hydrogen begins to build up in a thin layer on the surface
	  \end{itemize}
  \end{itemize}
  \begin{center}
	\includegraphics[width=.5\textwidth]{ch14_wd_accrete.jpg}
  \end{center}
\end{frame}

\begin{frame}{Novae! \footnotesize (Of the non-Super type)}
  \begin{itemize}
	\item Hydrogen continues to build on surface
	\item Gets compressed and heated
	\item When the bottom layer gets up to fusion temperatures\ldots
	  \begin{itemize}
		\item A flare up!
		\item Causes the system to shine for a few weeks as a \alert{nova}
		\item Less bright than supernovae but still 100,000 Suns worth
		\item Increased heat also provides gas pressure
		  \begin{itemize}
			\item Ejecting much of the other accreted material
		  \end{itemize}
	  \end{itemize}
	\item Goes back to accreting afterwards, to repeat again later
  \end{itemize}
\end{frame}

\begin{frame}{About that limit though\ldots}
  \begin{itemize}
	\item What if accretion pushes a white dwarf past the Chandrasekhar limit?
	\item Actually gets hot enough for carbon fusion
	\item The entire star ignites almost simultaneously
	\item Since electron degeneracy pressure does not depend on temperature, the star can't expand to regulate itself, and thus explodes
	\item Creates a white dwarf supernova (Type Ia Supernova)
	  \begin{itemize}
		\item Massive stars produces Type II Supernova
	  \end{itemize}
  \end{itemize}
\end{frame}

\begin{frame}{Differences in Supernova}
  \begin{itemize}
	\item It is possible to distinguish between the supernova types in a few ways:
	  \begin{itemize}
		\item Ia Supernova have little or no hydrogen absorption lines
		\item Their luminosities decay over time in different ways
	  \end{itemize}
	\item Because all Ia supernova come from $1.4M_\odot$ stars, they have very consistent luminosities
	  \begin{itemize}
		\item Can be used as a \alert{standard candle}
		\item Known luminosity lets us use them as a ruler to determine distances!
		\item Generally the best method for measuring distances for faraway galaxies
	  \end{itemize}
  \end{itemize}
\end{frame}

\fullFrameImage{ch14_2014J.png}

\begin{frame}{Neutron Stars}
  \begin{columns}
	\column{.5\textwidth}
	\begin{itemize}
	  \item Higher mass stars collapse past the Chandrasekhar limit, going supernova and leaving behind a neutron star
	  \item Supported by \alert{neutron} degeneracy pressure
	  \item Only about 20 kilometers in diameter!
	  \item Neutrons packed side by side throughout the star
	  \item Made of\ldots \alert{neutronium}!
	  \item Incredibly dense
	\end{itemize}
	\column{.5\textwidth}
	\begin{center}
	  \includegraphics[width=.8\textwidth]{ch14_neutronstar1.jpg}
	\end{center}
  \end{columns}
\end{frame}

\begin{frame}{Stop being so Dense}
  \begin{itemize}
	\item One cubic centimeter has a greater mass than Mt Everest!
	\item That same cube would plunge through the Earth like steel through air
	\item Would make swiss cheese of the Earth as it rotated
	\item A ``mountain'' on a neutron star = \SI{10}{\centi\meter}
	  \begin{itemize}
		\item Larger mountains would create gravitational waves!
	  \end{itemize}
  \end{itemize}
  \begin{center}
	\includegraphics[width=.6\textwidth]{ch14_mountain.png}
  \end{center}
\end{frame}

\begin{frame}{How do we see them?}
  \begin{itemize}
	\item Directly
	  \begin{itemize}
		\item X-ray sources
		\item Extremely faint optical sources
	  \end{itemize}
	\item Members of a binary
	  \begin{itemize}
		\item In-falling matter emits X-rays
	  \end{itemize}
	\item Pulsars!
  \end{itemize}
\end{frame}

\begin{frame}{Little Green Men}
  \begin{itemize}
	\item First discovered in 1967
	  \begin{itemize}
		\item Joceyln Bell
		\item Discovered a source of strange radio waves
	  \end{itemize}
	\item Pulsed on and off in precise 1.337301 second intervals
	\item Jokingly called LGM for Little Green Men
	  \begin{itemize}
		\item Now known as \alert{Pulsars}
	  \end{itemize}
	\item Soon more were found in the center of supernova remnants
  \end{itemize}
\end{frame}

\fullFrameImage{ch14_crab_pulsar.jpg}

\begin{frame}{Structure of a Pulsar}
  \begin{columns}
	\column{.5\textwidth}
	\begin{itemize}
	  \item Strong magnetic fields focus outgoing radiation
	  \item Results in a ``lighthouse'' effect
	  \item Spins very fast due to conservation of angular momentum
	  \item Rapid spinning causes these beams of radiation to sweep across us
	\end{itemize}
	\column{.5\textwidth}
	\begin{center}
	  \includegraphics[width=.9\textwidth]{ch14_pulsar.jpg}
	\end{center}
  \end{columns}
\end{frame}

%\fullFrameMovie{../Videos/Pulsar.ogv}{../Videos/Pulsar.png}

%\begin{frame}{Tick Tock}
  %\begin{itemize}
	%\item Isolated pulsars will eventually slow as they lose angular momentum and energy to the outgoing gases
	%\item Pulsars as part of binary systems may actually speed up
	  %\begin{itemize}
		%\item A steady diet of more mass lends more angular momentum
		%\item Can end up pulsing every few thousandths of a second
	  %\end{itemize}
	%\item There is a rotational speed limit
	  %\begin{itemize}
		%\item If you spin too fast, gravity can't keep everything contained
		%\item Is what tells us pulsars must be neutron stars and not white dwarfs
	  %\end{itemize}
  %\end{itemize}
%\end{frame}


\end{document}

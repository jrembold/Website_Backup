\documentclass[pdf,aspectratio=169]{beamer}
\usepackage[]{hyperref,graphicx,siunitx,booktabs,lmodern}
\usepackage{physics}
\usepackage{em-commands}
\mode<presentation>{\usetheme{EM}}

%Question Numbering
\newcounter{questionnumber}
\newcommand{\qnum}{%
	\stepcounter{questionnumber}%
	Q\arabic{questionnumber}
}
\resetcounteronoverlays{questionnumber}

\graphicspath{ {../Images/} }

\sisetup{per-mode=symbol}

%preamble
\title{But what is being displaced?!}
\date{October 15, 2018}
\author{Jed Rembold}

\begin{document}
\renewcommand{\theenumi}{\Alph{enumi}}

\begin{frame}{Announcements}
	\begin{itemize}
		\item Test Results
			\begin{itemize}
				\item Trying to hand things back on Wednesday
				\item Definitely by next Monday
				\item Some people still need to take it so don't be too chatty about it yet please!
			\end{itemize}
		\item Homework 7 is posted! Due next Monday
		\item No class on Friday!
	\end{itemize}
\end{frame}

\begin{frame}{\qnum}
	If you put a polarizable material (a dielectric) in an external field $\ef_{ext}$, it polarizes, adding a new field, $\ef_{pol}$.
	These superpose, making a total field $\ef_{tot}$. What is the vector equation relating these fields?
	\begin{enumerate}
		\item $\ef_{tot} + \ef_{ext} + \ef_{pol} = 0$
		\item $\ef_{tot} = \ef_{ext} - \ef_{pol}$
		\item \alert<2>{$\ef_{tot} = \ef_{ext} + \ef_{pol}$}
		\item $\ef_{tot} = -\ef_{ext} + \ef_{pol}$
	\end{enumerate}
\end{frame}

\begin{frame}{\qnum}
	We define the ``Electric Displacement'' or ``D'' field:
	\[\va{D} = \epsilon_0\ef + \va{P}\]
	If you put a dielectric in an \emph{external} field, it polarizes, adding a new \emph{induced} field (from the bound charges). These then superpose, making a \emph{total} electric field. Which of these three fields is the $\ef$ in the formula for $\va{D}$ above?
	\begin{enumerate}
		\item $\ef_{external}$
		\item $\ef_{induced}$
		\item \alert<2>{$\ef_{total}$}
	\end{enumerate}
\end{frame}

\begin{frame}{\qnum}
	A solid non-conducting dielectric rod has been injected (``doped'') with a fixed, known charge distribution $\rho(s)$.
	The material then responds, polarizing internally.

	When computing $D$ in the rod, do you treat this $\rho(s)$ as the ``free charges'' or ``bound charges''?
	\begin{enumerate}
		\item \alert<2>{``free charge!''}
		\item ``bound charge!''
		\item Neither! $\rho(s)$ is some combination of the two
		\item Something else
	\end{enumerate}
\end{frame}

\begin{frame}{\qnum}
	We define $\va{D} = \epsilon_0 \ef + \va{P}$ with
	\[\oint \va{D}\vdot d\vA = Q_{free}\]
	A point charge $+q$ is placed at the center of a dielectric sphere of radius $R$.
	There are no other free charges anywhere.
	What is $\abs{\va{D}(r)}$?
	\begin{enumerate}
		\item \alert<2>{$\displaystyle \frac{q}{4\pi r^2}$}
		\item $\displaystyle \frac{q}{4\pi\epsilon_0 r^2}$
		\item $\displaystyle \frac{q}{4\pi r^2}$ for $r<R$, but $\displaystyle \frac{q}{4\pi\epsilon_0 r^2}$ for $r>R$
		\item None of the above, we need more information to answer.
	\end{enumerate}
\end{frame}

\begin{frame}{\qnum}
	When there are no free charges ($\rho_{free}=0$) in a dielectric material, does the electric potential $V$ in that material satisfy Laplace's Equation?
	\[\laplacian V = 0\]
	\begin{enumerate}
		\item Yes
		\item No
		\item \alert<2>{Sometimes}
	\end{enumerate}
\end{frame}

\begin{frame}{\qnum}
	An ``electret'' is the electric equivalent of a bar magnet. 
	Say you have a cylindrical electret whose baked in polarization points along the cylinder's axis, going as:
	\[\va{P} = P_0 \zhat\]
	There are no free charges present.
	Which of the following would be a true statement?
	\begin{enumerate}
		\item $\ef=0$ everywhere
		\item $\va{D}=0$ everywhere
		\item $\abs*{\va{P}}=P_0$ everywhere
		\item \alert<2>{None of the above}
	\end{enumerate}
\end{frame}

%\begin{frame}{\qnum}
	%With the previous problem, we saw that despite there being no free charges present, $\ef\neq 0$ and thus $\va{D}\neq 0$.
	%Since $\div \va{D} = \rho_{free}$, this implies that $\curl \va{D} \neq 0$. Where is there a non-zero value of the curl of $\va{D}$ in this problem?
	%\begin{enumerate}
		%\item In the center of the cylinder
		%\item At the top surface
		%\item At the bottom surface
		%\item Outside the cylinder
	%\end{enumerate}
%\end{frame}









\end{document}

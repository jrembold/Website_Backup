\documentclass[pdf,aspectratio=169]{beamer}
\usepackage[]{hyperref,graphicx,siunitx,booktabs,lmodern}
\usepackage{physics}
\usepackage{em-commands}
\mode<presentation>{\usetheme{EM}}

%Question Numbering
\newcounter{questionnumber}
\newcommand{\qnum}{%
	\stepcounter{questionnumber}%
	Q\arabic{questionnumber}
}
\resetcounteronoverlays{questionnumber}

\graphicspath{ {../Images/} }

\sisetup{per-mode=symbol}

\tikzstyle{plate}=[draw, very thick, minimum width=4cm, minimum height=1cm, fill=gray!40, anchor=south]

%preamble
\title{Coulomb per Second Law}
\date{October 29, 2018}
\author{Jed Rembold}

\begin{document}
\renewcommand{\theenumi}{\Alph{enumi}}

\begin{frame}{Announcements}
	\begin{itemize}
		\item Homework 8 is due tonight!
		\item Grade reports posted to WISE dropbox!
			\begin{itemize}
				\item If we had agreements about giving back some early late days, those have not
					been factored in yet.
			\end{itemize}
		\item Read at least part of 5.4 for Wednesday
	\end{itemize}
\end{frame}


\begin{frame}{\qnum}
	What do you expect for the direction of $\mf$ at point P? How about the direction of $d\mf$ at point P generated \textbf{only} by the segment of current $d\vell$ in red?
	\begin{center}
		\begin{tikzpicture}[use Hobby shortcut]
			\draw[very thick, ->-=.5] (0,0) .. (2,1) .. (4,1) .. (6,3) coordinate[pos=.7] (r1) coordinate[pos=.9] (r2);
			\draw[line width=2pt, Red, opacity=.75, -latex] (r1) -- (r2);
			\node[point,label={below:P}] at (3,0) {};
		\end{tikzpicture}
	\end{center}
	\begin{enumerate}
		\item $\mf$ in the plane of the page, $d\mf$ in plane of page
		\item $\mf$ into page, $d\mf$ into page
		\item \alert<2>{$\mf$ into page, $d\mf$ out of page}
		\item $\mf$ out of page, $d\mf$ out of page
	\end{enumerate}
\end{frame}

\begin{frame}{\qnum}
	\begin{columns}
		\column{0.5\textwidth}
		What is the magnitude of $\displaystyle \frac{d\vell\cross\srhat}{\srmag^2}$?
		\begin{enumerate}
			\item $\displaystyle \frac{d\ell \sin\phi}{z^2}$
			\item $\displaystyle \frac{a\, d\ell}{z^2}$
			\item $\displaystyle \frac{d\ell \sin\phi}{z^2 + a^2}$
			\item \alert<2>{$\displaystyle \frac{d\ell}{z^2+a^2}$}
		\end{enumerate}
		\column{0.5\textwidth}
		\begin{center}
			\tdplotsetmaincoords{70}{100}
			\begin{tikzpicture}[tdplot_main_coords,scale=1.25]
				\draw[ -latex] (0,0,0) -- +(3,0,0) node[left]{$\xhat$};
				\draw[ -latex] (0,0,0) -- +(0,3,0) node[right]{$\yhat$};
				\draw[ -latex] (0,0,0) -- +(0,0,3) node[above]{$\zhat$};
				\draw[ultra thick,->-=.9, xyplane=0] (2,0) arc (0:-360:2) coordinate[pos=.85] (l);
				\node[below] at (l) {$\cur$};
				\node[point,label={left:$(0,0,z)$}](p) at (0,0,2.5) {};
				\draw[cyan,-latex,very thick,xyplane=0] (0,0) -- +(30:2) coordinate(dl);
				\draw[<->, xyplane=0] (0,1) arc (90:30:1) node[midway,fill=Background,font=\small, inner sep=1pt] {$\phi$};
				\draw[teal,-latex,very thick] (dl) -- (p) node[pos=.6,right,math,black]{\sr};
				\draw[<->, shorten <>=1mm] (0,0,0) -- +(0,-2,0) node[below,math,midway] {a};
			\end{tikzpicture}
		\end{center}
		
	\end{columns}
\end{frame}

\begin{frame}{\qnum}
	Say you have two very long, parallel wires each carrying a current $\cur_1$ and $\cur_2$, respectively. In which direction is the force on the wire with current $\cur_2$?
	\begin{columns}
		\column{0.3\textwidth}
		\begin{enumerate}
			\item Up
			\item \alert<2>{Left}
			\item Right
			\item Out of the page
		\end{enumerate}
		\column{0.3\textwidth}
		\begin{center}
			\begin{tikzpicture}
				\draw[very thick, Red, ->-=.6] (0,0) -- +(0,5) node[pos=.6,left]{$\cur_1$};
				\draw[very thick, Red, ->-=.6] (2,0) -- +(0,5) node[pos=.6,right]{$\cur_2$};
			\end{tikzpicture}
		\end{center}
	\end{columns}
\end{frame}

\begin{frame}{\qnum}
	What is $\oint \mf\vdot d\vell$ around this purple Amperian loop?
	\begin{columns}
		\column{0.5\textwidth}
		\begin{center}
			\begin{tikzpicture}[use Hobby shortcut, scale=.7]
				\draw[violet,very thick, closed, ->-=.5, ->-=.8, name path=loop] 
					(45:2) .. (225:3) .. (135:3) .. (-45:2);
				\node[math,label={above:$\mathcal{I}_1$}] at (0:1.5) {\odot};
				\draw[very thick, ->-=.5] ($(180:2.5)+(120:2)$) -- (180:2.5);
				\draw[dashed,thick] (180:2.5) -- +(0,1.6) node[left,math,xshift=.5mm,pos=.7] {\theta};
				\path[name path = cur1] (180:2.5) -- +(-60:3) coordinate (end);
				\draw[very thick, -latex, name intersections={of=loop and cur1, by=p}] (p) -- (end) node[right,math] {\mathcal{I}_2};
			\end{tikzpicture}
		\end{center}
		\column{0.5\textwidth}
		\begin{enumerate}
			\item \alert<2>{$\mu_0 (\mcur_2 + \mcur_1)$}
			\item $\mu_0 (\mcur_2 - \mcur_1)$
			\item $\mu_0 (\mcur_2\sin\theta + \mcur_1)$
			\item $\mu_0 (\mcur_2\sin\theta - \mcur_1)$
		\end{enumerate}
	\end{columns}
\end{frame}

\begin{frame}{\qnum}
	Rank order $\int \vcd\vdot d\vA$ over the blue surfaces where $\vcd$ is uniform and traveling left to right.
	\begin{center}
		\begin{tikzpicture}
			\node[font=\LARGE\bf] (p) at (0,0) {i};
			\draw[fill=teal!50, even odd rule]
				($(p)-(0,2)$) circle (2.5mm and 1cm);

			\node[font=\LARGE\bf] (p) at (2,0) {ii};
			\draw[fill=teal!50, even odd rule]
				($(p)-(0,2)$) circle (2.5mm and 1cm)
				($(p)-(0,1)$) arc (90:-90:1cm) arc(270:90:2.5mm and 1cm);

			\node[font=\LARGE\bf] (p) at (4,0) {iii};
			\draw[fill=teal!50, even odd rule]
				($(p)-(0,2)$) circle (2.5mm and 1cm)
				%($(p)-(0,1)$) .. controls +(45:1) and +(90:2) .. ($(p)+(2,-2)$) ;
				($(p)-(0,1)$) .. controls +(0:1) and +(90:2.5) .. ($(p)+(2,-2)$) .. controls +(270:2.5) and +(0:1) .. ($(p)-(0,3)$) arc(270:90:2.5mm and 1cm);

			\node[font=\LARGE\bf] (p) at (7,0) {iv};
			\draw[fill=teal!50, even odd rule]
				($(p)-(0,2)$) circle (2.5mm and 1cm)
				%($(p)-(0,1)$) .. controls +(45:1) and +(90:2) .. ($(p)+(2,-2)$) ;
				($(p)-(0,1)$) .. controls +(0:1) and +(90:.45) .. ($(p)+(3,-2)$) .. controls +(270:.45) and +(0:1) .. ($(p)-(0,3)$) arc(270:90:2.5mm and 1cm);
		\end{tikzpicture}
	\end{center}
	\vspace{-1cm}
	\begin{enumerate}
		\item $iii > iv > ii > i$
		\item $iii > i > ii > iv$
		\item $i > ii > iii> iv$
		\item \alert<2>{Something else}
	\end{enumerate}
\end{frame}

%\begin{frame}{\qnum}
	%\begin{columns}
		%\column{0.75\textwidth}
		%Ampere's Law will only be useful to us when there is sufficient symmetry to pull B out of the integral. So we need some methods to understand when we might have the needed symmetry.

		%For the case of an infinitely long wire, can $\mf$ point radially (i.e., in the $\vu{s}$ direction)? \emph{Can you explain WHY?}
		%\begin{enumerate}
			%\item Yes
			%\item No
		%\end{enumerate}
		
		%\column{0.25\textwidth}
		%\begin{center}
			%\begin{tikzpicture}
				%\draw[very thick, Red, -latex] (0,0) -- +(0,6) node[above,math] {\cur};
				%\draw[thick, -latex] (-.1,3) -- +(-1,0) node[midway,above] {$\mf$?};
			%\end{tikzpicture}
		%\end{center}
		
	%\end{columns}
%\end{frame}

%\begin{frame}{\qnum}
	%\begin{columns}
		%\column{0.75\textwidth}
		%Continuing to refine our arguments, for an infinitely long straight wire, can $\mf$ depend on $z$ or $\phi$? \emph{Why?}
		%\begin{enumerate}
			%\item Yes
			%\item No
		%\end{enumerate}
		
		%\column{0.25\textwidth}
		%\begin{center}
			%\begin{tikzpicture}
				%\draw[very thick, Red, -latex] (0,0) -- +(0,6) node[above,math] {\cur};
			%\end{tikzpicture}
		%\end{center}
	%\end{columns}
%\end{frame}

%\begin{frame}{\qnum}
	%\begin{columns}
		%\column{0.75\textwidth}
		%And finally, for an infinitely long straight wire, can $\mf$ have a component in the $\zhat$ direction? \emph{Why?}
		%\begin{enumerate}
			%\item Yes
			%\item No
		%\end{enumerate}
		
		%\column{0.25\textwidth}
		%\begin{center}
			%\begin{tikzpicture}
				%\draw[very thick, Red, -latex] (0,0) -- +(0,6) node[above,math] {\cur};
				%\draw[thick, -latex] (-1,3) -- +(0,1) node[above] {$\mf$?};
			%\end{tikzpicture}
		%\end{center}
	%\end{columns}
%\end{frame}

%\begin{frame}{\qnum}
	%So our arguments got us a functional form of
	%\[\mf(\pos) = B(s)\vu*{\phi}\]
	%For the case of an infinitely long \emph{thick} wire of radius $a$, is this functional form still correct?
	%\begin{enumerate}
		%\item Yes
		%\item Only inside the wire ($s < a$)
		%\item Only outside the wire ($s > a$)
		%\item No
	%\end{enumerate}
%\end{frame}


\end{document}

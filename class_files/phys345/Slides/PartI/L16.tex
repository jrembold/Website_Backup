\documentclass[pdf,aspectratio=169]{beamer}
\usepackage[]{hyperref,graphicx,siunitx,booktabs,lmodern}
\usepackage{physics}
\usepackage{em-commands}
\mode<presentation>{\usetheme{EM}}

%Question Numbering
\newcounter{questionnumber}
\newcommand{\qnum}{%
	\stepcounter{questionnumber}%
	Q\arabic{questionnumber}
}
\resetcounteronoverlays{questionnumber}

\graphicspath{ {../Images/} }

\sisetup{per-mode=symbol}

%preamble
\title{Separating the R's from the \texorpdfstring{$\Theta$}{T}'s}
\date{October 3, 2018}
\author{Jed Rembold}

\begin{document}
\renewcommand{\theenumi}{\Alph{enumi}}

\begin{frame}{Announcements}
	\begin{itemize}
		\item Homework 6 posted (now in its entirety!)
			\begin{itemize}
				\item I \emph{REALLY} suggest you start this one early. Separation of variable problems are not short and take \emph{time}.
			\end{itemize}
		\item Exam 1 a week from Friday: \emph{in class}
		\item Have read Ch 3.4 on the Multipole expansion by Friday
	\end{itemize}
\end{frame}

\begin{frame}{\qnum}
	Given $\laplacian V = 0$ in Cartesian coordinates, we separated $V(x,y,z) = X(x)Y(y)Z(z)$. Will a similar approach work for spherical coordinates? I.e.\ can we separate $V(r,\theta,\phi) = R(r)\Theta(\theta)\Phi(\phi)$?
	\begin{enumerate}
		\item Sure
		\item \alert<2>{Not quite, the angular bits can not be isolated from one another}
		\item No, because in spherical coordinates the Laplace equation has lots of cross terms in it
		\item I think so, but I'm not sure why
	\end{enumerate}
\end{frame}

\begin{frame}{\qnum}
	The general solution for the electric potential in spherical coordinates with azimuthal symmetry is:
	\[V(r,\theta) = \sum_{\ell=0}^{\infty}(A_\ell r^\ell + \frac{B_\ell}{r^{\ell+1}})P_\ell(\cos\theta)\]
	Consider a metal sphere at some constant potential. Which terms in the sum vanish outside the sphere?
	\begin{enumerate}
		\item \alert<2>{All the $A_\ell$'s}
		\item All the $A_\ell$'s except $A_0$
		\item All the $B_\ell$'s
		\item All the $B_\ell$'s except $B_0$
	\end{enumerate}
\end{frame}

\begin{frame}{\qnum}
	\[V(r,\theta) = \sum_{\ell=0}^{\infty}(A_\ell r^\ell + \frac{B_\ell}{r^{\ell+1}})P_\ell(\cos\theta)\]
	Now say that $V$ everywhere on a spherical shell is a constant $V_0$, and there are no charges inside the sphere. Which terms do you expect to appear when solving for the potential inside the sphere?
	\begin{enumerate}
		\item Many $A_\ell$ terms, but no $B_\ell$'s
		\item Many $B_\ell$ terms, but no $A_\ell$'s
		\item \alert<2>{Just $A_0$}
		\item Just $B_0$
	\end{enumerate}
\end{frame}

\begin{frame}{\qnum}
	Given an initial condition of
	\[V_0 = \sum_{\ell=0}^\infty C_\ell P_\ell(\cos\theta)\]
	we want to solve for $C_\ell$. We can do so by multiplying both sides by what and then integrating?
	\begin{enumerate}
		\item $P_m(\cos\theta)$
		\item $P_m(\sin\theta)$
		\item \alert<2>{$P_m(\cos\theta)\sin\theta$}
		\item $P_m(\sin\theta)\cos\theta$
	\end{enumerate}
\end{frame}

\begin{frame}{\qnum}
	Suppose $V$ on a spherical shell of radius $R$ is:
	\[V(R,\theta) = V_0(1+\cos^2\theta)\]
	Which terms do you expect to appear when solving for the potential inside the shell? (Again, there are no other charges inside the shell.)
	\begin{enumerate}
		\item Many $A_\ell$ terms, but no $B_\ell$'s
		\item Many $B_\ell$ terms, but no $A_\ell$'s
		\item \alert<2>{Just $A_0$ and $A_2$}
		\item Just $B_0$ and $B_2$
	\end{enumerate}
\end{frame}

\begin{frame}{Example}
	Let's work out the full solution for the potential on the inside of the spherical shell if the potential on the shell is given by:
	\[V(R,\theta) = V_0(1+\cos^2\theta)\]
	given the general form of the potential:
	\[V(r,\theta) = \sum_{\ell=0}^{\infty}(A_\ell r^\ell + \frac{B_\ell}{r^{\ell+1}})P_\ell(\cos\theta)\]
\end{frame}

\begin{frame}{\qnum}
	Suppose $V$ on a spherical shell of radius $R$ is:
	\[V(R,\theta) = V_0(1+\cos^2\theta)\]
	What would be the value of the $B_2$ term when solving for the potential outside the shell?
	\begin{enumerate}
		\item $\displaystyle \frac{4RV_0}{3}$
		\item $\displaystyle \frac{2V_0}{3}$
		\item \alert<2>{$\displaystyle \frac{2R^3V_0}{3}$}
		\item $\displaystyle \frac{4V_0}{3}$
	\end{enumerate}
\end{frame}









\end{document}

\documentclass[pdf,aspectratio=169]{beamer}
\usepackage[]{hyperref,graphicx,siunitx,booktabs,lmodern}
\usepackage{physics}
\usepackage{em-commands}
\mode<presentation>{\usetheme{EM}}

%Question Numbering
\newcounter{questionnumber}
\newcommand{\qnum}{%
	\stepcounter{questionnumber}%
	Q\arabic{questionnumber}
}
\resetcounteronoverlays{questionnumber}

\graphicspath{ {../Images/} }

\sisetup{per-mode=symbol}

%preamble
\title{Expanding the Multipole}
\date{October 5, 2018}
\author{Jed Rembold}

\begin{document}
\renewcommand{\theenumi}{\Alph{enumi}}

\begin{frame}{Announcements}
	\begin{itemize}
		\item Homework 6 due on Monday!
			\begin{itemize}
				\item Hopefully you are making progress on it! Do \alert{not} save it all till Monday, or you are going to have a really sad day!
			\end{itemize}
		\item Exam 1 a week from today! \emph{in class}
			\begin{itemize}
				\item I'm working on cleaning up my solutions in case you want to check other homework problems against them next week
				\item I'm also going to try to make a nice list of learning objectives of what I expect you to be able to do
			\end{itemize}
		\item Read Ch 4.1 for Monday. Heading into understanding electric fields in matter!
	\end{itemize}
\end{frame}

\begin{frame}{\qnum}
	\begin{columns}
		\column{0.5\textwidth}
		Two charges are positioned as shown to the right. The relative vector between them is $\va{d}$. What is the value of the dipole moment?
		\[\va{p} = \sum_i q_i \pos_i\]
		\begin{enumerate}
			\item \alert<2>{$+q\va{d}$}
			\item $-q\va{d}$
			\item 0
			\item None of these
		\end{enumerate}
		
		\column{0.5\textwidth}
		\begin{center}
			\begin{tikzpicture}
				\draw[-latex] (-.5,0) -- (3,0) node[right,math] {\xhat};
				\draw[-latex] (0,-3) -- (0,3) node[above,math] {\yhat};

				\node[circle, draw, very thick, Red, fill=Red!30] (p) at (0,2) {$+q$};
				\node[circle, draw, very thick, cyan, fill=cyan!30] (n) at (0,-2) {$-q$};

				\draw[line width=3pt, -latex] ([xshift=1cm]n.center) -- ([xshift=1cm]p.center) node[pos=.6,right,math] {\va{d}};
			\end{tikzpicture}
		\end{center}
	\end{columns}
\end{frame}

\begin{frame}{\qnum}
	\begin{columns}
		\column{0.5\textwidth}
		\begin{center}
			\begin{tikzpicture}
				\draw[-latex] (-0.5,0) -- +(3,0) node[right,math] {\xhat};
				\draw[-latex] (0,-.5) -- +(0,3) node[above,math] {\yhat};

				\node[circle, draw, very thick, Red, fill=Red!30, font=\scriptsize, label={above:$+q$}] (p) at (2,4) {};
				\node[circle, draw, very thick, cyan, fill=cyan!30, font=\scriptsize, label={below:$-q$}] (n) at (4,2) {};

				\draw[ultra thick, -latex] (n.center) -- (p.center) node[midway,sloped,math,above] {\va{d}};

			\end{tikzpicture}
			
		\end{center}
		\column{0.5\textwidth}
		Now we have positioned the two charges as shown to the left. The relative position vector between them is still $\va{d}$. What is the dipole moment of this configuration?
		\begin{enumerate}
			\item \alert<2>{$+q\va{d}$}
			\item $-q\va{d}$
			\item 0
			\item None of these. It is more complicated.
		\end{enumerate}
	\end{columns}
\end{frame}


\begin{frame}{\qnum}
	In Eq 3.103, your book derives that:
	\[\va{E}_{dip}(\pos) = \frac{p}{4\pi\epsilon_0 r^3}(2\cos\theta\, \rhat + \sin\theta\,\vu*{\theta})\]
	What does the formula predict for the direction of the electric field at $\pos=0$?
	\begin{enumerate}
		\item Down
		\item Up
		\item Some other direction
		\item \alert<2>{Something is wrong!}
	\end{enumerate}
\end{frame}

\begin{frame}{\qnum}
	What are the first two terms of the multipole expansion for the configuration shown to the right?
	\begin{columns}
		\column{0.5\textwidth}
		\begin{enumerate}
			\item \alert<2>{$\displaystyle \frac{kq}{r} + \frac{2qd}{r^2}\,\yhat\vdot\rhat$}
			\item $\displaystyle 0 + \frac{2qd}{r^2}\,\yhat\vdot\rhat$
			\item $\displaystyle \frac{kq}{r} + \frac{5qd}{2r^2}\,\yhat\vdot\rhat$
			\item $\displaystyle \frac{3kq}{r} + \frac{2qd}{r^2}\,\yhat\vdot\rhat$
		\end{enumerate}
		
		\column{0.5\textwidth}
		\begin{center}
			\begin{tikzpicture}
				\draw[-latex] (0,0) -- (2,0) node[right,math] {\xhat};
				\draw[-latex] (0,0) -- (0,3) node[above,math] {\yhat};
				\node[circle, draw, very thick, Red, fill=Red!30, font=\scriptsize, label={left:$+2q$}] (p) at (0,2) {};
				\node[circle, draw, very thick, cyan, fill=cyan!30, font=\scriptsize, label={left:$-q$}] (n) at (0,0) {};

				\draw[very thick,-latex] (n.center) -- (p.center) node[midway,right,math] {\va{d}};
			\end{tikzpicture}
		\end{center}
	\end{columns}
	
\end{frame}

\begin{frame}{\qnum}
	What about the first two terms of the multipole expansion for the configuration shown to the left (same basic configuration, but shifted)?
	\begin{columns}
		\column{0.5\textwidth}
		\begin{center}
			\begin{tikzpicture}
				\draw[-latex] (0,0) -- (2,0) node[right,math] {\xhat};
				\draw[-latex] (0,-2) -- (0,2) node[above,math] {\yhat};
				\node[circle, draw, very thick, Red, fill=Red!30, font=\scriptsize, label={left:$+2q$}] (p) at (0,1) {};
				\node[circle, draw, very thick, cyan, fill=cyan!30, font=\scriptsize, label={left:$-q$}] (n) at (0,-1) {};

				\draw[very thick,-latex] (n.center) -- (p.center) node[midway,left,math] {\va{d}};
			\end{tikzpicture}
		\end{center}

		\column{0.5\textwidth}
		\begin{enumerate}
			\item $\displaystyle \frac{kq}{r} + \frac{2qd}{r^2}\,\yhat\vdot\rhat$
			\item $\displaystyle 0 + \frac{3qd}{2r^2}\,\yhat\vdot\rhat$
			\item $\displaystyle \frac{kq}{r} + \frac{5qd}{2r^2}\,\yhat\vdot\rhat$
			\item \alert<2>{$\displaystyle \frac{kq}{r} + \frac{3qd}{2r^2}\,\yhat\vdot\rhat$}
		\end{enumerate}
	\end{columns}
\end{frame}

\begin{frame}{\qnum}
	You have a physical dipole, with $+q$ and $-q$ a distance $d$ apart. When can you use the expression:
	\[V(\pos) = \frac{1}{4\pi\epsilon_0}\frac{\va{p}\vdot\rhat}{r^2}\]
	\begin{enumerate}
		\item This expression is exact everywhere
		\item \alert<2>{It is valid for large $r$}
		\item It is valide for small $r$
		\item This expression is not true anywhere
	\end{enumerate}
\end{frame}

\begin{frame}{Exercise}
	In groups, for each of the charge distributions below decide what the dominating behavior will be when $r$ is large. (Will it go as $\tfrac{1}{r}$? $\tfrac{1}{r^2}$? $\tfrac{1}{r^4}$? etc)
	\begin{center}
		\begin{tikzpicture}
			\coordinate(o) at (0,0);
			\node[font=\LARGE] at (o) {A};
			\node[ncharge, font=\tiny, text=black] at ($(o)+(0,1)$) {-q};
			\node[pcharge, font=\tiny, text=black] at ($(o)+(0,2)$) {2q};

			\coordinate(o) at (3,0);
			\node[font=\LARGE] at (o) {B};
			\node[pcharge, font=\tiny, text=black] at ($(o)+(0,1)$) {2q};
			\node[ncharge, font=\tiny, text=black] at ($(o)+(0,2)$) {-2q};

			\coordinate(o) at (6,0);
			\node[font=\LARGE] at (o) {C};
			\node[pcharge, font=\tiny, text=black] at ($(o)+(-.5,1)$) {q};
			\node[ncharge, font=\tiny, text=black] at ($(o)+(-.5,2)$) {-q};
			\node[ncharge, font=\tiny, text=black] at ($(o)+(.5,1)$) {-q};
			\node[pcharge, font=\tiny, text=black] at ($(o)+(.5,2)$) {q};

			\coordinate(o) at (9,0);
			\node[font=\LARGE] at (o) {D};
			\node[pcharge, font=\tiny, text=black] at ($(o)+(0,2)$) {2q};
			\node[ncharge, font=\tiny, text=black] at ($(o)+(-.5,1)$) {-q};
			\node[ncharge, font=\tiny, text=black] at ($(o)+(.5,1)$) {-q};

		\end{tikzpicture}
	\end{center}
\end{frame}

\begin{frame}{\qnum}
	In terms of the multipole expansion
	\[V(r) = V_{mono}(r) + V_{dip}(r) + V_{quad}(r) + \cdots\]
	The below charge distribution would have what form?
	\begin{center}
		\begin{tikzpicture}
			\node[pcharge, font=\tiny, text=black] at (0,1) {q};
			\node[pcharge, font=\tiny, text=black] at (1,0) {q};
			\node[pcharge, font=\tiny, text=black] at (2,0) {q};
			\node[pcharge, font=\tiny, text=black] at (3,1) {q};
			\node[ncharge, font=\tiny, text=black] at (0,0) {-q};
			\node[ncharge, font=\tiny, text=black] at (1,1) {-q};
			\node[ncharge, font=\tiny, text=black] at (2,1) {-q};
			\node[ncharge, font=\tiny, text=black] at (3,0) {-q};
		\end{tikzpicture}
	\end{center}
	\begin{enumerate}
		\item $V(r) = V_{mono} + V_{dip} + \text{ higher order terms}$
		\item $V(r) = V_{dip} + \text{ higher order terms}$
		\item $V(r) = V_{dip}$
		\item \alert<2>{$V(r) = \text{ only higher order terms}$}
	\end{enumerate}
\end{frame}








\end{document}

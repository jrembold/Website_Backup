\documentclass[pdf,aspectratio=169]{beamer}
\usepackage[]{hyperref,graphicx,siunitx,booktabs,lmodern}
\usepackage{physics}
\usepackage{em-commands}
\mode<presentation>{\usetheme{EM}}

%Question Numbering
\newcounter{questionnumber}
\newcommand{\qnum}{%
	\stepcounter{questionnumber}%
	Q\arabic{questionnumber}
}
\resetcounteronoverlays{questionnumber}

\graphicspath{ {../Images/} }

\sisetup{per-mode=symbol}

%preamble
\title{Its a Polarizing Issue\ldots}
\date{October 8, 2018}
\author{Jed Rembold}

\begin{document}
\renewcommand{\theenumi}{\Alph{enumi}}

\begin{frame}{Announcements}
	\begin{itemize}
		\item Homework 6 due on tonight!
			\begin{itemize}
				\item If it isn't in tonight, I make no guarantees about having it graded by Friday
			\end{itemize}
		\item Exam 1 on Friday in class!
			\begin{itemize}
				\item More homework solutions cleaned up but not quite through all of them
				\item Still working on putting together learning objectives
			\end{itemize}
		\item Read Ch 4.2 for Tuesday.
	\end{itemize}
\end{frame}

\begin{frame}{Questions}
	\begin{itemize}
		\item Questions about expectations or format for the exam?
	\end{itemize}
\end{frame}

\begin{frame}{\qnum}
	In which of the below situations is the dipole term the leading non-zero contribution to the potential?
	\begin{center}
		\begin{tikzpicture}[scale=.9]
			\coordinate (c) at (0,0);
			\node[font=\Large] at (c) {1};
			\node[ncharge,font=\tiny,math,text=black] at ($(c)+(-.5,-1)$) {q};
			\node[ncharge,font=\tiny,math,text=black] at ($(c)+(-.5,-2)$) {q};
			\node[pcharge,font=\tiny,math,text=black] at ($(c)+(.5,-1)$) {q};
			\node[pcharge,font=\tiny,math,text=black] at ($(c)+(.5,-2)$) {q};

			\coordinate (c) at (3,0);
			\node[font=\Large] at (c) {2};
			\node[pcharge,font=\tiny,math,text=black] at ($(c)+(0,-1)$) {q};
			\node[pcharge,font=\tiny,math,text=black] at ($(c)+(0,-2)$) {2q};
			\node[pcharge,font=\tiny,math,text=black] at ($(c)+(0,-3)$) {q};

			\coordinate (c) at (6,0);
			\node[font=\Large] at (c) {3};
			\node[ncharge,font=\tiny,math,text=black] at ($(c)+(-.5,-1)$) {q};
			\node[pcharge,font=\tiny,math,text=black] at ($(c)+(-.5,-2)$) {2q};
			\node[pcharge,font=\tiny,math,text=black] at ($(c)+(.5,-1)$) {q};
			\node[pcharge,font=\tiny,math,text=black] at ($(c)+(.5,-2)$) {q};

			\coordinate (c) at (9,0);
			\node[font=\Large] at (c) {4};
			\node[circle, minimum size=8mm, draw, very thick, gray, fill=gray!50, font=\tiny, label={below:$\sigma = \sigma_0 \theta$}] at ($(c)-(0,1)$) {};

			\coordinate (c) at (12,0);
			\node[font=\Large] at (c) {5};
			\node[circle, draw, minimum size=10mm, very thick, gray, fill=gray!50, font=\tiny, label={below:$\sigma = \sigma_0\cos\theta$}] at ($(c)-(0,1)$) {};
		\end{tikzpicture}
	\end{center}
	\begin{enumerate}
		\item 1 and 3
		\item 2 and 4
		\item 1 and 4
		\item \alert<2>{1 and 5}
	\end{enumerate}
\end{frame}

\begin{frame}{\qnum}
	Consider a single point charge at the origin. It will have \emph{only} a monopole contribution to the potential at the location
	\[\pos = x\xhat + y\yhat + z\zhat\]
	However, if we move the charge to another location, say at $\pos_s = d\zhat$, the distribution now has a dipole contribute to the potential at $\pos$!

	What on earth is going on?

	\begin{enumerate}
		\item It's just how the math works out. Nothing has changed physically at $\pos$.
		\item \alert<2>{There is something different about the field at $\pos$ and the potential is showing us that.}
		\item The multipole expansion only applies for points far from the charge, so this doesn't matter.
		\item I'm confused and have no idea how to explain this.
	\end{enumerate}
\end{frame}

\begin{frame}{\qnum}
	A stationary point charge $+Q$ is near a block of insulating (of dielectric) material. The net electrostatic force on the block due to the point charge is:
	\begin{columns}
		\column{0.5\textwidth}
		\begin{enumerate}
			\item \alert<2>{attractive (to the left)}
			\item repulsive (to the right)
			\item 0
			\item attractive (upwards)
		\end{enumerate}
		\column{0.5\textwidth}
		\begin{center}
			\begin{tikzpicture}
				\node[pcharge] at (-2,1,1) {$+Q$};

				\draw[very thick, line join=round, fill=Teal]
					(0,0,2) -- (2,0,2) -- (2,2,2) -- (0,2,2) -- cycle;
				\draw[very thick, line join=round, fill=Teal!50]
					(0,2,0) -- (2,2,0) -- (2,2,2) -- (0,2,2) -- cycle;
				\draw[very thick, line join=round, fill=Teal!50!black]
					(2,0,0) -- (2,0,2) -- (2,2,2) -- (2,2,0) -- cycle;
				\draw[thick, line cap=round, dashed]
					(0,0,0) -- +(2,0,0)
					(0,0,0) -- +(0,2,0)
					(0,0,0) -- +(0,0,2);
			\end{tikzpicture}
		\end{center}
	\end{columns}
\end{frame}

\begin{frame}{\qnum}
	\begin{columns}
		\column{0.5\textwidth}
		The cube to the right (with side length $a$) has uniform polarization $\va{P}_0$ pointing in the $\zhat$ direction. What is the total dipole moment of this cube?
		\begin{enumerate}
			\item 0
			\item \alert<2>{$a^3 \va{P}_0$}
			\item $\va{P}_0$
			\item $\va{P}_0/a^3$
		\end{enumerate}
		\column{0.5\textwidth}
		\begin{center}
			\begin{tikzpicture}[scale=2]
				\draw[very thick, line join=round, fill=orange]
					(0,0,2) -- (2,0,2) -- (2,2,2) -- (0,2,2) -- cycle;
				\draw[very thick, line join=round, fill=orange!50]
					(0,2,0) -- (2,2,0) -- (2,2,2) -- (0,2,2) -- cycle;
				\draw[very thick, line join=round, fill=orange!50!black]
					(2,0,0) -- (2,0,2) -- (2,2,2) -- (2,2,0) -- cycle;
				\draw[thick, line cap=round, dashed]
					(0,0,0) -- +(2,0,0)
					(0,0,0) -- +(0,2,0)
					(0,0,0) -- +(0,0,2);
				\draw[|-|] ([yshift=-2mm]0,0,2) -- +(2,0,0) node[midway,below,math] {a};
			\end{tikzpicture}
		\end{center}
	\end{columns}
\end{frame}

\begin{frame}{\qnum}
	Consider a cylinder of radius $a$ and height $b$ that has its base at the origin and is aligned along the $z$ axis. The polarization of this cylinder in ``baked in'' and can be modeled by
	\[\va{P} = P_0 \left(\frac{z}{b}\right)\zhat\]
	What is the total dipole moment of the cylinder?
	\begin{enumerate}
		\item $\displaystyle P_0 \pi a^2 b \zhat$
		\item \alert<2>{$\displaystyle \frac{1}{2}P_0 \pi a^2 b \zhat$}
		\item $\displaystyle 2 P_0 \pi a^2 b \zhat$
		\item Something else
	\end{enumerate}
\end{frame}



\end{document}

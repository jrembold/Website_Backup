\documentclass[pdf,aspectratio=169]{beamer}
\usepackage[]{hyperref,graphicx,siunitx,lmodern,booktabs}
\usepackage{physics}
\usepackage{em-commands}
\mode<presentation>{\usetheme{EM}}

%Question Numbering
\newcounter{questionnumber}
\newcommand{\qnum}{%
	\stepcounter{questionnumber}%
	Q\arabic{questionnumber}
}
\resetcounteronoverlays{questionnumber}

\graphicspath{ {../Images/} }

\sisetup{per-mode=symbol}

%preamble
\title{Numerics}
\date{September 19, 2018}
\author{Jed Rembold}

\begin{document}
\renewcommand{\theenumi}{\Alph{enumi}}

\begin{frame}{Announcements}
	\begin{itemize}
		\item Homework 4 is posted!
			\begin{itemize}
				\item I made a small addendum/explanation to the plotting on \#2 since yesterday, so make sure you look at the latest version.
			\end{itemize}
		\item Poll up for time spent on HW3 on Campuswire
		\item Friday we'll be talking conductors, so have read Ch 2.5
	\end{itemize}
\end{frame}

\begin{frame}{Numerics Tutorial}
	\begin{itemize}
		\item Many times a charge distribution can't be described with a nice function
		\item We still can calculate our desired values, but need to do it numerically
		\item Things to keep in mind today
			\begin{itemize}
				\item Doing something numerically is just doing a bunch of point charge calculations and then adding them up
				\item To get the field or potentials everywhere in space, you'll need to loop over a grid of space and calculate the field values at each
			\end{itemize}
		\item This weeks tutorial is just Question 4 on your homework, so you'll turn it in as such!
	\end{itemize}
\end{frame}





\end{document}

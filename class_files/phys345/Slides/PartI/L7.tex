\documentclass[pdf,aspectratio=169]{beamer}
\usepackage[]{hyperref,graphicx,siunitx,lmodern,booktabs}
\usepackage{physics}
\usepackage{em-commands}
\mode<presentation>{\usetheme{EM}}

\graphicspath{ {../Images/} }

\sisetup{per-mode=symbol}

%preamble
\title{One point for Dirac}
\date{September 12, 2018}
\author{Jed Rembold}

\begin{document}
\renewcommand{\theenumi}{\Alph{enumi}}

\begin{frame}{Announcements}
	\begin{itemize}
		\item Homework 3 posted, due on Monday
			\begin{itemize}
				\item I might appreciate a few more people getting going and asking questions before Monday\ldots
			\end{itemize}
		\item I'll try to get HW2 graded before the end of the week
		\item I'm going to put a poll up on Campuswire about how you feel class hours are going or if there are things you think could be changed to assist you more
		\item Friday Reading: Ch 2.2.4, start Ch 2.3
	\end{itemize}
\end{frame}

\begin{frame}{}
	A solid cylindrical rod is $L$ meters long and has a radius of $R$. The pipe has a charge density given by
	\[\rho(\pos) = \rho_0 s\]
	where $s$ is the cylindrical coordinate. Approximately what is the magnitude of the electric field $2R$ from the center of the rod?
	\begin{enumerate}
		\item $\displaystyle \frac{\rho_0 R^2}{3\epsilon_0}$
		\item \alert<2>{$\displaystyle \frac{\rho_0 R^2}{6\epsilon_0}$}
		\item $\displaystyle \frac{2\pi\rho_0R^3 L}{\epsilon_0}$
		\item $\displaystyle \frac{\rho_0 R^3}{\epsilon_0}$
	\end{enumerate}
\end{frame}

\begin{frame}{}
	A solid cylindrical rod is $L$ meters long and has a radius of $R$. The pipe has a charge density given by
	\[\rho(\pos) = \rho_0 s\]
	where $s$ is the cylindrical coordinate. Approximately what is the magnitude of the electric field $\frac{R}{2}$ from the center of the rod?
	\begin{enumerate}
		\item $\displaystyle \frac{\rho_0 R}{6\epsilon_0}$
		\item $\displaystyle \frac{\rho_0 R^2}{6\epsilon_0}$
		\item \alert<2>{$\displaystyle \frac{\rho_0 R^2}{12\epsilon_0}$}
		\item $\displaystyle \frac{\rho_0 R^3 L}{8\epsilon_0}$
	\end{enumerate}
\end{frame}

\begin{frame}{}
	What is the value of
	\[\int_{-\infty}^{\infty}x^3\delta(x-2)\,dx\]
	\begin{enumerate}
		\item 0
		\item 4
		\item \alert<2>{8}
		\item $\infty$
	\end{enumerate}
\end{frame}

\begin{frame}{}
	In groups (to be determined), evaluate the following integrals, noting anything special you had to do or account for:
	\begin{itemize}
		\item Group A: $\displaystyle \int_{-\infty}^{\infty} xe^x \delta(x-1)\,dx$\onslide<2>{$=e$}
		\item Group B: $\displaystyle \int_{\infty}^{-\infty} \log(x) \delta(x-2)\,dx$\onslide<2>{$=-\log(2)$}
		\item Group C: $\displaystyle \int_{-\infty}^{0} xe^x \delta(x-1)\,dx$\onslide<2>{$=0$}
		\item Group D: $\displaystyle \int_{-\infty}^{\infty} (x+1)^2 \delta(4x)\,dx$\onslide<2>{$=\frac{1}{4}$}
	\end{itemize}
\end{frame}

\begin{frame}{}
	Compute the following:
	\[\int_{-\infty}^{\infty}x^2 \delta(3x+5)\,dx\]
	\begin{enumerate}
		\item $\displaystyle \frac{25}{3}$
		\item $\displaystyle -\frac{5}{3}$
		\item \alert<2>{$\displaystyle \frac{25}{27}$}
		\item $\displaystyle \frac{25}{9}$
	\end{enumerate}
\end{frame}

\begin{frame}{}
	A point charge $q$ is located at position $\va{R}$ as shown. What is $\rho(\pos)$, the charge density of all space?
	\begin{columns}
		\column{0.3\textwidth}
		\begin{enumerate}
			\item $\rho(\pos) = q\delta^3(\va{R})$
			\item $\rho(\pos) = q\delta^3(\pos)$
			\item \alert<2>{$\rho(\pos) = q\delta^3(\va{R}-\pos)$}
			\item \alert<2>{$\rho(\pos) = q\delta^3(\pos -\va{R})$}
		\end{enumerate}
		
		\column{0.5\textwidth}
		\begin{center}
			\begin{tikzpicture}
				\node[point, label={below:origin}] (o) at (0,0) {};
				\node[point, label={right:$q$}] (p) at (60:3) {};
				\draw[very thick, -latex] (o) -- (p) node[midway, below right, math] {\va{R}};
			\end{tikzpicture}
		\end{center}
	\end{columns}
\end{frame}

\begin{frame}{}
	What are the units of $\delta^3(\pos)$ if the components of $\pos$ are measured in meters?
	\begin{enumerate}
		\item $\delta^3(\pos)$ is a unitless quantity
		\item $[\si{\meter^3}]$: Units of length cubed
		\item $[\si{\meter}]$: Units of length
		\item \alert<2>{$[\si{\meter^{-3}}]$: Units of inverse cubic meters}
	\end{enumerate}
\end{frame}

%\begin{frame}{}
	%Evaluate
	%\[\int_{\text{all space}}(r^2 + \pos\vdot\va{a} +a^2)\delta^3(\pos-\va{a})\,d\tau\]
	%\begin{enumerate}
		%\item $3a^2$
		%\item $2a^2$
		%\item $a^2$
		%\item $0$
	%\end{enumerate}
	
%\end{frame}






\end{document}

\documentclass[pdf,aspectratio=169]{beamer}
\usepackage[]{hyperref,graphicx,siunitx,lmodern,booktabs}
\usepackage{physics}
\usepackage{em-commands}
\mode<presentation>{\usetheme{EM}}

%Question Numbering
\newcounter{questionnumber}
\newcommand{\qnum}{%
	\stepcounter{questionnumber}%
	Q\arabic{questionnumber}
}
\resetcounteronoverlays{questionnumber}

\graphicspath{ {../Images/} }

\sisetup{per-mode=symbol}

%preamble
\title{Traveling the E\&M Railroad}
\date{September 21, 2018}
\author{Jed Rembold}

\begin{document}
\renewcommand{\theenumi}{\Alph{enumi}}

\begin{frame}{Announcements}
	\begin{itemize}
		\item Be making progress on Homework 4! Don't save it all till Monday!
		\item I'm going to see about getting grade reports posted soon
		\item Monday we start Ch 3! Read Ch 3.1.
	\end{itemize}
\end{frame}

\begin{frame}{\qnum}
	A point charge $+q$ sits outside a \emph{solid neutral conducting copper sphere} of radius $A$. The charge $q$ is a distance $r > A$ from the center, on the right side. What is the electric field at the center of the sphere, assuming the system is in equilibrium?
	\begin{columns}
		\column{0.4\textwidth}
		\begin{enumerate}
			\item $\displaystyle \abs*{\ef} = \frac{1}{4\pi\epsilon_0}\frac{q}{r^2}$ to left
			\item $\displaystyle \frac{1}{4\pi\epsilon_0}\frac{q}{r^2} > \abs*{\ef} > 0$ to left
			\item $\abs*{\ef} > 0$, to right
			\item \alert<2>{$\abs*{\ef} = 0$}
		\end{enumerate}
		\column{0.5\textwidth}
		\begin{center}
			\begin{tikzpicture}
				\node (o) at (0,0) {};
				\fill[Teal] (o) circle (2cm);
				\node[circle, fill=Red, label={above:$+q$}] (q) at (2.5,0) {};
				\path[<->, very thick] (o) edge node[midway,above,math]{r} (q) 
					(o) edge node[midway,right,math]{A} (0,-2);
			\end{tikzpicture}
		\end{center}
		
	\end{columns}
\end{frame}

\begin{frame}{\qnum}
	A point charge $+q$ sits outside a \emph{solid conducting copper sphere} of radius $A$ with a total charge of $+Q$. The charge $q$ is a distance $r > A$ from the center, on the right side. Now what is the electric field at the center of the sphere, assuming the system is in equilibrium?
	\begin{columns}
		\column{0.5\textwidth}
		\begin{center}
			\begin{tikzpicture}
				\node (o) at (0,0) {};
				\fill[Teal] (o) circle (2cm);
				\node[circle, fill=Red, label={above:$+q$}] (q) at (2.5,0) {};
				\path[<->, very thick] (o) edge node[midway,above,math]{r} (q) 
					(o) edge node[midway,right,math]{A} (0,-2);
			\end{tikzpicture}
		\end{center}
		\column{0.4\textwidth}
		\begin{enumerate}
			\item $\displaystyle \abs*{\ef} = \frac{1}{4\pi\epsilon_0}\frac{q}{r^2}$ to left
			\item $\displaystyle \frac{1}{4\pi\epsilon_0}\frac{q}{r^2} > \abs*{\ef} > 0$ to left
			\item $\abs*{\ef} > 0$, to right
			\item \alert<2>{$\abs*{\ef} = 0$}
		\end{enumerate}
	\end{columns}
\end{frame}

\begin{frame}{\qnum}
	We have a large copper plate with a uniform surface charge density, $\sigma$. Imagine the Gaussian surface drawn below. Calculate the electric field a small distance $s$ above the conductor's surface.
	\begin{columns}
		\column{0.5\textwidth}
		\begin{center}
			\begin{tikzpicture}
				\fill[orange] (0,0) rectangle +(6,-1.5);
				\coordinate (e) at (1,0.5);
				\draw[very thick] (e) arc (180:-180:1cm and .25cm)
					(e) -- ++(0,-1) arc(180:360:1cm and .25cm) --++(0,1);
				\draw[<->, very thick] ([xshift=-2mm]e) --+(0,-0.5) node[midway,left,math] {s};
			\end{tikzpicture}
		\end{center}
		\column{0.5\textwidth}
		\begin{enumerate}
			\item \alert<2>{$\displaystyle \abs*{\ef} = \frac{\sigma}{\epsilon_0}$}
			\item $\displaystyle \abs*{\ef} = \frac{\sigma}{2\epsilon_0}$
			\item $\displaystyle \abs*{\ef} = \frac{1}{4\pi\epsilon_0}\frac{\sigma}{s^2}$
			\item $\displaystyle \abs*{\ef} = 0$
		\end{enumerate}
		
	\end{columns}
\end{frame}

\begin{frame}{Exercise}
	Consider a long coaxial cable with charge $+Q$ placed on the inside wire and charge $-Q$ placed on the outside metal sheath as shown.
	\begin{columns}
		\column{0.4\textwidth}
		Sketch the distribution of charge in this situation using plus signs to represent positive charges and minus signs to represent negative charges.
		\column{0.5\textwidth}
		\begin{center}
			\begin{tikzpicture}
				\draw[fill=gray!40] (0,0) circle (1);
				\node at (0,0) {$+Q$};
				\draw[fill=gray!40, even odd rule] (0,0) circle (2) (0,0) circle (2.5);
				\node at (0,2.25) {$-Q$};

				\onslide<2>{
					\foreach \a in {0,45,...,360}{
						\pic at (\a:0.8) {plus};
						\pic at (\a:2.1) {minus};
					}
				}
			\end{tikzpicture}
		\end{center}
	\end{columns}
\end{frame}


\begin{frame}{Exercise}
	\begin{columns}
		\column{0.5\textwidth}
		\begin{center}
			\begin{tikzpicture}
				\draw[fill=gray!40] (0,0) circle (1);
				\draw[fill=gray!40, even odd rule] (0,0) circle (2) (0,0) circle (2.5);
				\node[math] at (0,0) {+Q};
				
				\onslide<2>{
					\foreach \a in {0,45,...,360}{
						\pic at (\a:0.8) {plus};
						\pic at (\a:2.1) {minus};
						\pic at (\a:2.4) {plus};
					}
				}
			\end{tikzpicture}
		\end{center}
		\column{0.5\textwidth}
		Now draw the charge distribution ($+$ and $-$ signs) if the inner conductor has a total charge $+Q$ on it and the outer conductor is electrically neutral.
		
	\end{columns}
\end{frame}

\begin{frame}{Exercise}
	\begin{columns}
		\column{0.5\textwidth}
		\begin{center}
			\begin{tikzpicture}
				\draw[fill=gray!40] (0,0) circle (1);
				\draw[fill=gray!40, even odd rule] (0,0) circle (2) (0,0) circle (2.5);
				\node[math] at (0,2.25) {+Q};
				\onslide<2>{
					\foreach \a in {0,45,...,360} \pic at (\a:2.4) {plus};
				}
			\end{tikzpicture}
		\end{center}
		\column{0.5\textwidth}
		Now draw the charge distribution ($+$ and $-$ signs) if the outer conductor has a total charge $+Q$ on it and the inner conductor is electrically neutral.
	\end{columns}
\end{frame}

\begin{frame}{\qnum}
	If you were calculating the potential difference $\Delta V$ between the center of the inner conductor ($s=0$) and infinitely far away ($s\rightarrow\infty$), what regions of space would have a non-zero contribution to your calculation?
	\begin{columns}
		\column{0.4\textwidth}
		\begin{enumerate}
			\item $a < s < b$
			\item \alert<2>{$b < s < c$}
			\item $s > c$
			\item More than one of these
		\end{enumerate}
		\column{0.5\textwidth}
		\begin{center}
			\begin{tikzpicture}
				\draw[fill=gray!40] (0,0) circle (1);
				\draw[fill=gray!40, even odd rule] (0,0) circle (2) (0,0) circle (2.5);
				\path[<->, thick] (0,0) edge node[midway,above,math]{a} (1,0)
					(0,0) edge node[pos=.7,left,math]{b} (0,2)
					(0,0) edge node[midway,above,math]{c} (-2.5,0);
				\node at (0,-.5) {$+Q$};
				\node at (0,2.25) {$-Q$};
			\end{tikzpicture}
		\end{center}
	\end{columns}
\end{frame}

\begin{frame}{Uncovered Slides}
	Subsequent slides were not covered during class hours, but I'll leave them here for extra problems you can look at.
\end{frame}

\begin{frame}{\qnum}
	With $\curl \ef = 0$, we know that
	\[\oint \ef\vdot d\vell = 0\]
	So if we choose a loop that includes metal and vacuum (both in and outside of the metal), we know that the contribution inside the metal vanishes. What can we can we say about the contribution just outside the metal?
	\begin{enumerate}
		\item $\ef$ must be zero out there
		\item $\ef$ must be perpendicular to $d\vell$ everywhere out there
		\item \alert<2>{$\ef$ is perpendicular to the metal surface}
		\item More than one of these
	\end{enumerate}
\end{frame}

\begin{frame}{\qnum}
	A neutral copper sphere has a spherical hollow in the center. A charge $+q$ is placed in the center of the hollow. What is the total charge on the outside surface of the copper sphere, assuming electrostatic equilibrium?
	\begin{columns}
		\column{0.5\textwidth}
		\begin{center}
			\begin{tikzpicture}
				\draw[fill=orange!40, even odd rule] (0,0) circle (1) (0,0) circle (2);
				\node[circle,draw, fill=red!50] at (0,0) {$+q$};
				\draw[<-] (45:2) --+(45:.5) node[right,math] {q_{outer}};
			\end{tikzpicture}
		\end{center}
		\column{0.5\textwidth}
		\begin{enumerate}
			\item 0
			\item $-q$
			\item \alert<2>{$+q$}
			\item $0 < q_{outer} < +q$
		\end{enumerate}
		
	\end{columns}
\end{frame}


\end{document}

\documentclass[pdf,aspectratio=169]{beamer}
\usepackage[]{hyperref,graphicx,siunitx,lmodern,booktabs}
\usepackage{physics}
\usepackage{em-commands}
\mode<presentation>{\usetheme{EM}}

%Question Numbering
\newcounter{questionnumber}
\newcommand{\qnum}{%
	\stepcounter{questionnumber}%
	Q\arabic{questionnumber}
}
\resetcounteronoverlays{questionnumber}

\graphicspath{ {../Images/} }

\sisetup{per-mode=symbol}

%preamble
\title{Boundless Energy}
\date{September 17, 2018}
\author{Jed Rembold}

\begin{document}
\renewcommand{\theenumi}{\Alph{enumi}}

\begin{frame}{Announcements}
	\begin{itemize}
		\item Homework 3 is due tonight!
			\begin{itemize}
				\item \alert{Please} remember to associate pages with problems once you upload
				\item Take a look over your scans/upload to make sure everything is visible and readable
			\end{itemize}
		\item I'll be curious and posting a poll about how long HW3 took you, but HW2 took most people longer than I intended, so I'm trying to take that into account
		\item Bring your laptops on Wednesday as we'll be doing a tutorial on finding electric fields and potentials numerically!
	\end{itemize}
\end{frame}

\begin{frame}{\qnum}
	Determine which of the electric fields below are legitimate:
	\begin{align*}
		A&: \quad k(xy\xhat + 2yz\yhat + 3xz\zhat)\\
		B&: \quad k(y^2\xhat + (2xy + z^2)\yhat + 2yz\zhat)
	\end{align*}
	\begin{enumerate}
		\item A is legit but B is not
		\item \alert<2>{B is legit but A is not}
		\item Both A and B are legitimate electric fields
		\item Neither A nor B are legitimate electric fields
	\end{enumerate}
\end{frame}


\begin{frame}{\qnum}
	Three identical charges $+q$ sit at the vertices of an equilateral triangle. What would the final kinetic energy of the top charge be if you released it (keeping the other two fixed)?
	\begin{columns}
		\column{0.25\textwidth}
		\begin{enumerate}
			\item $\displaystyle \frac{1}{4\pi\epsilon_0}\frac{q^2}{a}$
			\item $\displaystyle \frac{1}{4\pi\epsilon_0}\frac{2q^2}{3a}$
			\item \alert<2>{$\displaystyle \frac{1}{4\pi\epsilon_0}\frac{2q^2}{a}$}
			\item $\displaystyle \frac{1}{4\pi\epsilon_0}\frac{3q^2}{a}$
		\end{enumerate}
		\column{0.5\textwidth}
		\begin{center}
			\begin{tikzpicture}
				\node[circle,draw, very thick] (a) at (0,0) {$+q$};
				\node[circle,draw, very thick] (b) at (0:3) {$+q$};
				\node[circle,draw, very thick] (c) at (60:3) {$+q$};
				\draw[very thick] (a) -- (b) node[midway,above,math]{a}
								  (b) -- (c) node[midway,above right,math]{a}
								  (a) -- (c) node[midway,above left,math]{a};
			\end{tikzpicture}
		\end{center}
		
	\end{columns}
\end{frame}

\begin{frame}{\qnum}
	Three identical charges $+q$ sit at the vertices of an equilateral triangle. What would the final kinetic energy of the top charge be if you released all three charges?
	\begin{columns}
		\column{0.25\textwidth}
		\begin{enumerate}
			\item \alert<2>{$\displaystyle \frac{1}{4\pi\epsilon_0}\frac{q^2}{a}$}
			\item $\displaystyle \frac{1}{4\pi\epsilon_0}\frac{2q^2}{3a}$
			\item $\displaystyle \frac{1}{4\pi\epsilon_0}\frac{2q^2}{a}$
			\item $\displaystyle \frac{1}{4\pi\epsilon_0}\frac{3q^2}{a}$
		\end{enumerate}
		\column{0.5\textwidth}
		\begin{center}
			\begin{tikzpicture}
				\node[circle,draw, very thick] (a) at (0,0) {$+q$};
				\node[circle,draw, very thick] (b) at (0:3) {$+q$};
				\node[circle,draw, very thick] (c) at (60:3) {$+q$};
				\draw[very thick] (a) -- (b) node[midway,above,math]{a}
								  (b) -- (c) node[midway,above right,math]{a}
								  (a) -- (c) node[midway,above left,math]{a};
			\end{tikzpicture}
		\end{center}
		
	\end{columns}
\end{frame}

\begin{frame}{\qnum}
	Does system energy obey superposition?\vspace{1cm}

	That is, if you have one system of charges with total stored energy $W_1$, and a second system of charges with stored energy $W_2$, if you take the two systems to be one single system, is the total energy of the new system $W_1 + W_2$?
	\begin{center}
		\begin{enumerate}
			\item Yes
			\item \alert<2>{No}
		\end{enumerate}
	\end{center}
\end{frame}

\begin{frame}{\qnum}
	What is the energy of a single point charge?
	\begin{enumerate}
		\item \alert<2>{0}
		\item $\displaystyle \frac{1}{4\pi\epsilon_0}\frac{q^2}{r}$
		\item $\displaystyle \frac{1}{4\pi\epsilon_0}\frac{q^2}{2r}$
		\item \alert<2>{$\infty$}
	\end{enumerate}
\end{frame}

\begin{frame}{\qnum}
	\begin{center}
		\begin{tikzpicture}
			\coordinate (n) at (-2,0);
			\coordinate (p) at (2,0);
			\draw[very thick, fill=cyan!30] (n) circle (5mm);
			\foreach \a in {0,180} \draw[line width=3pt] (n) -- +(\a:3mm);
			\draw[very thick, fill=red!30] (p) circle (5mm);
			\foreach \a in {0,90,...,360} \draw[line width=3pt] (p) -- +(\a:3mm);
			\draw[|-|] ([yshift=-1cm]n) -- ([yshift=-1cm]p) node[midway, below, math] {r};
			\path[dashed, thick, -latex] ([yshift=1cm]n) edge ++(1.5cm,0) ([yshift=1cm]p) edge +(-1.5,0);
		\end{tikzpicture}
	\end{center}
	Two charges, $+q$ and $-q$, are a distance $r$ apart. As the charges are slowly moved together, the total field energy:
	\[\frac{\epsilon_0}{2}\int E^2\,d\tau\]
	\begin{enumerate}
		\item Increases
		\item \alert<2>{Decreases}
		\item Remains constant
	\end{enumerate}
\end{frame}

%\begin{frame}{\qnum}
	%\begin{center}
		%\vspace{-1cm}
		%\begin{tikzpicture}
			%\coordinate (t) at (0,1);
			%\coordinate (b) at (0,-1);
			%\node[draw, very thick, fill=red!30, minimum width=8cm, minimum height=2mm, label={right:$+Q$}] at (t) {};
			%\node[draw, very thick, fill=cyan!30, minimum width=8cm, minimum height=2mm, label={right:$-Q$}] at (b) {};
			%\path[dashed, very thick, -latex] (t) edge +(0,1) (b) edge +(0,-1);
		%\end{tikzpicture}
	%\end{center}
	%A parallel-plate capacitor has $+Q$ on one plate and $-Q$ on the other. The plates are isolated so that $Q$ can not change. As the plates are pulled apart, the total electrostatic energy stored in the capacitor:
	%\begin{enumerate}
		%\item Increases
		%\item Decreases
		%\item Remains constant
	%\end{enumerate}
%\end{frame}




\end{document}

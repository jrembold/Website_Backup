\documentclass[pdf,aspectratio=169]{beamer}
\usepackage[]{hyperref,graphicx,siunitx,lmodern,booktabs}
\usepackage{physics}
\usepackage{em-commands}
\mode<presentation>{\usetheme{EM}}

%Question Numbering
\newcounter{questionnumber}
\newcommand{\qnum}{%
	\stepcounter{questionnumber}%
	Q\arabic{questionnumber}
}
\resetcounteronoverlays{questionnumber}

\graphicspath{ {../Images/} }

\sisetup{per-mode=symbol}

%preamble
\title{Just chill}
\date{September 28, 2018}
\author{Jed Rembold}

\begin{document}
\renewcommand{\theenumi}{\Alph{enumi}}

\begin{frame}{Announcements}
	\begin{itemize}
		\item Homework 5 due on Monday!
			\begin{itemize}
				\item It should all be doable after today, so don't save it all till Monday!
				\item Heavy use of plotting throughout the homework, but I'm trying to convey that you should be comfortable and familiar with visualizing results.
			\end{itemize}
		\item Polls up for HW4 time and preferred Test 1 type on Campuswire! Don't dally as they expire today!
		\item Have read Ch 3.3.1 and at least part of Ch 3.3.2 by Monday
	\end{itemize}
\end{frame}

\begin{frame}{\qnum}
	Consider a function $f(x)$ that is both continuous and differentiable over some domain. Given a step size of $a$, which could be an approximate derivative of this function somewhere in that domain?
	\[\dv{f}{x} \approx\]
	\begin{enumerate}
		\item $\displaystyle \frac{f(x_i + a) - f(x_i)}{x_i}$
		\item $\displaystyle \frac{f(x_i+a) - f(x)}{a}$
		\item $\displaystyle \frac{f(x_i) - f(x_i-a)}{a}$
		\item \alert<2>{More than one of these\onslide<2>{ (B and C)}}
	\end{enumerate}
\end{frame}

\begin{frame}{\qnum}
	Say we want to use
	\[\dv{f}{x} \approx \frac{f(x_i+a) - f(x_i)}{a}\]
	What point best describes the location at which we are computing the approximate derivative?
	\begin{enumerate}
		\item $a$
		\item $x_i$
		\item $x_i + a$
		\item \alert<2>{Somewhere else\onslide<2>{  ($x_i + a/2$)}}
	\end{enumerate}
\end{frame}

\begin{frame}{\qnum}
	Taking a second derivative is as simple as applying the same discrete derivative equation again, at the location of the first derivative.
	\[f^{\prime\prime}(x_i) \approx \frac{f^\prime(x_i + a/2) - f^\prime(x_i-a/2)}{a}\]
What is the value of the second derivative then in terms of $f$?
\begin{enumerate}
	\item \alert<2>{$\displaystyle \frac{f(x-a/2) - 2f(x+a/2) + f(x+3a/2)}{a^2}$}
	\item $\displaystyle \frac{f(x-a/2) + f(x+3a/2)}{a^2}$
	\item $\displaystyle \frac{- 2f(x+a/2) }{a^2}$
	\item $\displaystyle \frac{f(x-a/2) + 2f(x+a/2) + f(x+3a/2)}{a^2}$
\end{enumerate}
\end{frame}

\begin{frame}{Implementing Relaxation}
	\begin{enumerate}
		\item Break up region of interest into discrete chunks
		\item Set boundary conditions
		\item Set initial guess at all other starting values
		\item Choose max iterations and target accuracy
		\item Start relaxing!
			\begin{itemize}
				\item Update all non-boundary terms with average of neighbors
				\item Calculate difference from last iteration
				\item Compare to target accuracy to see if keep iterating or target reached!
			\end{itemize}
		\item Plot up those sweet sweet results
	\end{enumerate}
\end{frame}

\begin{frame}{\qnum}
	To investigate if we have converged to a solution, we must compare our estimate of $V$ before and after the averaging calculation. For our 1D relaxation code, $V$ will be a 1D array. For the $k$th estimate, we can compare $V_k$ against the previous value by taking the difference. If this difference is stored as \texttt{err}, what is the type of \texttt{err}?
	\begin{enumerate}
		\item A scalar
		\item \alert<2>{A 1D array}
		\item A 2D array
		\item A string
	\end{enumerate}
\end{frame}

\begin{frame}{Demo!}
	For the rest of class I'll walk you through how I'd approach putting together a method of relaxation solver for the 1D case. A video of this will be available, but the notebook itself will not! Feel free to follow along on your laptops and ask questions as we go!
	\vspace{1cm}

	\textbf{Our problem:} Solve the 1D Laplace equation where $V(x=0) = -5$ and $V(x=50)=10$. We'd like to be accurate at least to within \si{\milli\volt}.
\end{frame}



\end{document}

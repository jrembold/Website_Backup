\documentclass[pdf,aspectratio=169]{beamer}
\usepackage[]{hyperref,graphicx,siunitx,lmodern,booktabs}
\usepackage{physics}
\usepackage{em-commands}
\mode<presentation>{\usetheme{EM}}

\graphicspath{ {../Images/} }

\sisetup{per-mode=symbol}

%preamble
\title{Gaining Potential}
\date{September 14, 2018}
\author{Jed Rembold}

\begin{document}
\renewcommand{\theenumi}{\Alph{enumi}}

\begin{frame}{Announcements}
	\begin{itemize}
		\item Homework 3 due Monday night
			\begin{itemize}
				\item Post questions to Campuswire over the weekend!
			\end{itemize}
		\item I'll be out of the office for much of the afternoon at the SCRP talks
		\item Monday Reading: Ch 2.3 and Ch 2.4
	\end{itemize}
\end{frame}


\begin{frame}{}
	Evaluate
	\[\int_{\text{all space}}(r^2 + \pos\vdot\va{a} +a^2)\delta^3(\pos-\va{a})\,d\tau\]
	\begin{enumerate}
		\item \alert<2>{$3a^2$}
		\item $2a^2$
		\item $a^2$
		\item $0$
	\end{enumerate}
\end{frame}

\begin{frame}{}
	Evaluate
	\[\div \frac{1}{4\pi\epsilon_0}\frac{q}{r^2}\rhat\]
	\begin{enumerate}
		\item $\displaystyle \frac{q}{\epsilon_0}$
		\item \alert<2>{$\displaystyle \frac{q}{\epsilon_0}\delta^3(\pos)$}
		\item $\displaystyle \frac{q}{\epsilon_0}\rhat$
		\item $\displaystyle \frac{4\pi q}{\epsilon_0}$
	\end{enumerate}
\end{frame}

\begin{frame}{}
	Consider a small charged bead pierced by a long straight narrow charged wire, as seen below. Could you use Gauss's Law to solve for the electric field a short distance above the bead?
	\begin{columns}
		\column{0.5\textwidth}
		\begin{center}
			\begin{tikzpicture}
				\draw[very thick, name path=circ] (1.5,0) arc (0:360:1.5cm);
				\path[name path=tline] (-3,.5) -- + (6,0);
				\path[name intersections={of=circ and tline}];
				\draw[very thick] (-3,.5) -- (intersection-2) (intersection-1) -- (3,.5);
				\draw[very thick, dashed] (intersection-1) -- (intersection-2);
				\path[name path=bline] (-3,-.5) -- + (6,0);
				\path[name intersections={of=circ and bline}];
				\draw[very thick] (-3,-.5) -- (intersection-1) (intersection-2) -- (3,-.5);
				\draw[very thick, dashed] (intersection-1) -- (intersection-2);
			\end{tikzpicture}
		\end{center}
		\column{0.5\textwidth}
		\begin{enumerate}
			\item Yes, just choose the correct surface.
			\item \alert<2>{Yes, but it requires multiple surfaces.}
			\item Yes, but only if the bead and wire have opposite charges.
			\item No, this doesn't have the needed symmetry.
		\end{enumerate}
		
	\end{columns}
\end{frame}

%\begin{frame}{}
	%Is the following valid?
	%\[\curl\left( \frac{1}{4\pi\epsilon_0} \iiint_V \frac{\rho(\pos_s)\,d\tau_s}{\srmag^2}\srhat \right) = \frac{1}{4\pi\epsilon_0}\iiint_V\left( \curl \frac{\rho(\pos_s)\,d\tau_s}{\srmag^2}\srhat \right)\]
	%\begin{enumerate}
		%\item Yup. That's valid and I can explain why.
		%\item I think so, but I'm not sure I could explain it.
		%\item Nope. That's illegal and I can explain why.
		%\item I can't even venture a guess\ldots
	%\end{enumerate}
%\end{frame}

%\begin{frame}{}
	%If $\curl \ef = 0$, then $\displaystyle \oint_C \ef\vdot d\vell = $
	%\begin{enumerate}
		%\item 0
		%\item Something finite
		%\item $\infty$
		%\item Impossible to tell without knowing $C$
	%\end{enumerate}
%\end{frame}

\begin{frame}{}
	Does superposition apply to electric potential, $V$?
	\[V_{tot} \overset{?}{=} \sum_i V_i = V_1 + V_2 + V_3 + \cdots\]
	\begin{enumerate}
		\item \alert<2>{Yes}
		\item No
		\item Sometimes
	\end{enumerate}
\end{frame}

\begin{frame}{}
	The potential is zero at some point in space.

	You can conclude that:
	\begin{enumerate}
		\item The E-field is zero at that point
		\item The E-field is zero near that point
		\item The E-field is non-zero at that point
		\item \alert<2>{You can conclude nothing about the E-field at that point}
	\end{enumerate}
\end{frame}

\begin{frame}{}
	The potential is constant everywhere in some region of space.

	You can conclude that:
	\begin{enumerate}
		\item The E-field is changing at a constant rate in that space.
		\item The E-field has a constant magnitude in that space.
		\item \alert<2>{The E-field is zero in that space.}
		\item You can conclude nothing at all about the E-field in that space.
	\end{enumerate}
\end{frame}

\begin{frame}{}
	We usually choose the reference point $\mathcal{O}$ to be 0 at $\pos=\infty$ when calculating the potential of a point charge. ($V = kq/r$) How does this potential change if we choose our reference point to be $V(R)=0$ when $R$ is close to $q$?
	\begin{enumerate}
		\item $V(r)$ is higher than before
		\item \alert<2>{$V(r)$ is lower than before}
		\item $V(r)$ doesn't change
		\item Impossible to say without knowing how close it gets
	\end{enumerate}
\end{frame}

\begin{frame}{}
	\begin{center}
		\begin{tikzpicture}
			\draw[thick, -latex] (-.1,0) -- (6,0) node[right,math] {\rhat};
			\draw[thick, -latex] (0,-.1) -- (0,4);
			\draw[ultra thick, red] (0,0) -- (1.5,0) -- (1.5,3) .. controls +(-90:1) and +(180:3) .. (5.5,0);
		\end{tikzpicture}
	\end{center}
	Could the above be a plot of $\abs{\ef}$? or $V(r)$? (Assuming for some physical situation)
	\begin{enumerate}
		\item Could be $E(r)$ or $V(r)$
		\item \alert<2>{Could be $E(r)$, but not $V(r)$}
		\item Can't be $E(r)$, but could be $V(r)$
		\item Can't be either
	\end{enumerate}
\end{frame}





\end{document}

\documentclass[pdf,aspectratio=169]{beamer}
\usepackage[]{hyperref,graphicx,siunitx,booktabs,lmodern}
\usepackage{physics}
\usepackage{em-commands}
\mode<presentation>{\usetheme{EM}}

%Question Numbering
\newcounter{questionnumber}
\newcommand{\qnum}{%
	\stepcounter{questionnumber}%
	Q\arabic{questionnumber}
}
\resetcounteronoverlays{questionnumber}

\graphicspath{ {../Images/} }

\sisetup{per-mode=symbol}

\tikzstyle{plate}=[draw, very thick, minimum width=4cm, minimum height=1cm, fill=gray!40, anchor=south]

%preamble
\title{Current Restrictions}
\date{November 9, 2018}
\author{Jed Rembold}

\begin{document}
\renewcommand{\theenumi}{\Alph{enumi}}

\begin{frame}{Announcements}
	\begin{itemize}
		\item Homework
			\begin{itemize}
				\item HW9 is graded
				\item HW10 due tonight!
				\item I'll grade HW10 tomorrow
			\end{itemize}
		\item Exam 2
			\begin{itemize}
				\item Don't forget Exam 2 is Monday!
				\item In-class portion similar to Exam 1
				\item Also a small take-home portion due on Wednesday
				\item I'm working on objectives for you to study from
				\item I'll be around a portion of Sunday if you have questions
			\end{itemize}
	\end{itemize}
\end{frame}

\begin{frame}{\qnum}
	\begin{columns}
		\column{0.5\textwidth}
		\begin{center}
			\begin{tikzpicture}
				\pic (c) at (0,0) {cyl=cyan!50!black/1.5/3/3};
				\draw[<->] ([xshift=-.4cm]c-lt) -- ([xshift=-.4cm]c-lb) node[left,midway] {$L$};
				\draw[<->, transform canvas={yshift=-.3cm}] (c-b) -- ($ (c-lt)!(c-b)!(c-lb) $) node[below,midway] {$R$};
				\foreach \y in {0,-1,-2,-3} \draw[very thick, orange, -latex] (-1,\y) -- +(2,0);
			\end{tikzpicture}
		\end{center}
		
		\column{0.5\textwidth}
		A solid cylinder has uniform magnetization $\vmag$ throughout the volume in the $x$ direction as shown. What is the magnitude of the total magnetic dipole moment of the cylinder?
		\begin{enumerate}
			\item \alert<2>{$\pi R^2 L M$}
			\item $2\pi R L M$
			\item $2\pi R M$
			\item $\pi R^2 M$
		\end{enumerate}
	\end{columns}
\end{frame}

\begin{frame}{\qnum}
	\begin{columns}
		\column{0.5\textwidth}
		A solid cylinder has uniform magnetization $\vmag$ throughout the volume in the $z$ direction as shown. Where do bound currents show up?
		\begin{enumerate}
			\item All surfaces, but not volume
			\item Volume only, not on surface
			\item Top/bottom surface only
			\item \alert<2>{Side (curved) surface only}
		\end{enumerate}
		\column{0.5\textwidth}
		\begin{center}
			\begin{tikzpicture}
				\pic (c) (0,0) {cyl=orange!50!black/1.5/3/3};
				\foreach \x in {-1,0,1} \draw[very thick, white,-latex] (\x,-3) -- +(0,2);
			\end{tikzpicture}
		\end{center}
	\end{columns}
\end{frame}

\begin{frame}{\qnum}
	\begin{columns}
		\column{0.4\textwidth}
		\begin{center}
			\begin{tikzpicture}
				\pic (c) (0,0) {cyl=Red!50!black/1.5/3/3};
				\foreach \y in {-1,-2,-3}{
					\draw[very thick, -latex, white] (0,\y)++(90+20:1.2cm and .3cm) arc(90+20:90+340:1.2cm and .3cm);
				}
				\node[white] at (0,-1.75) {$\vmag$};
			\end{tikzpicture}
		\end{center}
		\column{0.5\textwidth}
		A solid cylinder has uniform magnetization $\vmag$ throughout its volume in the $\vu*{\phi}$ direction as shown. In what direction does the bound surface current flow on the curved sides?
		\begin{enumerate}
			\item There is no bound surface current
			\item The current flows in the $+\vu*\phi$ direction
			\item The current flows in the $+\vu*z$ direction
			\item \alert<2>{The current flows in the $-\vu*z$ direction}
		\end{enumerate}
	\end{columns}
\end{frame}

\begin{frame}{\qnum}
	A sphere has uniform magnetization $\vmag$ in the $+\zhat$ direction. What formula is correct to describe the bound surface current about the sphere?
	\begin{columns}
		\column{0.5\textwidth}
		\begin{enumerate}
			\item $M \sin\theta \vu*{\theta}$
			\item \alert<2>{$M\sin\theta \vu*{\phi}$}
			\item $M\cos\phi \vu*{\theta}$
			\item $M\cos\phi \vu*{\phi}$
		\end{enumerate}
		\column{0.5\textwidth}
		\begin{center}
			\begin{tikzpicture}
				\draw[very thick, ball=violet!50] (0,0) circle (1.5);
				\draw[very thick] (180:1.5) arc (180:360:1.5 cm and 0.7cm);
				\draw[very thick, -latex] (90:1.2) -- +(0,1.5) node[right] {$\vmag$};
			\end{tikzpicture}
		\end{center}
	\end{columns}
\end{frame}

\begin{frame}{\qnum}
	Predict the results of the following experiment.

	A paramagnetic bar and a diamagnetic bar are pushed inside of a solenoid.
	\begin{enumerate}
		\item The paramagnet is pushed out, the diamagnet sucked further in
		\item \alert<2>{The diamagnet is pushed out, the paramagnet sucked further in}
		\item Both are sucked further in, but with different force
		\item Both are pushed out, but with different force
	\end{enumerate}
\end{frame}

\begin{frame}{\qnum}
	A very long aluminum (paramagnetic) rod carries a uniformly distributed current $\mcur$ along the $+\zhat$ direction. What is the direction of the bound volume current?
	\begin{columns}
		\column{0.5\textwidth}
		\begin{center}
			\begin{tikzpicture}
				\pic (c) at (0,0) {cyl=green!50!black/1.5/3/3};
				\foreach \x in {-1,0,1} \draw[very thick, -latex, white] (\x,-3) -- +(0,2);
				\node[white] at (.5,-2) {$\cur$};
			\end{tikzpicture}
		\end{center}
		\column{0.5\textwidth}
		\begin{enumerate}
			\item \alert<2>{$+\zhat$}
			\item $-\zhat$
			\item $+\vu*{\phi}$
			\item $-\vu*{\phi}$
		\end{enumerate}
	\end{columns}
\end{frame}

%\begin{frame}{\qnum}
	%In the same long aluminum (paramagnetic) rod, what would be the direction of the auxiliary field $\af$ and $\vmag$?
	%\begin{columns}
		%\column{0.6\textwidth}
		%\begin{enumerate}
			%\item Both in the $\vu*{\phi}$ direction
			%\item Both in the $-\vu*{\phi}$ direction
			%\item $\af$ in $\vu*{\phi}$ direction, but $\vmag$ in $-\vu*\phi$ direction
			%\item $\af$ in $-\vu*{\phi}$ direction, but $\vmag$ in $\vu*\phi$ direction
		%\end{enumerate}
		%\column{0.3\textwidth}
		%\begin{center}
			%\begin{tikzpicture}
				%\pic (c) at (0,0) {cyl=yellow!50!black/1.5/3/3};
				%\foreach \x in {-1,0,1} \draw[very thick, -latex, white] (\x,-3) -- +(0,2);
				%\node[white] at (.5,-2) {$\cur$};
			%\end{tikzpicture}
		%\end{center}
	%\end{columns}
%\end{frame}


\end{document}

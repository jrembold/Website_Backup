\documentclass[pdf,aspectratio=169]{beamer}
\usepackage[]{hyperref,graphicx,siunitx,booktabs,lmodern}
\usepackage{physics}
\usepackage{em-commands}
\mode<presentation>{\usetheme{EM}}

%Question Numbering
\newcounter{questionnumber}
\newcommand{\qnum}{%
	\stepcounter{questionnumber}%
	Q\arabic{questionnumber}
}
\resetcounteronoverlays{questionnumber}

\graphicspath{ {../Images/} }

\sisetup{per-mode=symbol}

\tikzstyle{plate}=[draw, very thick, minimum width=4cm, minimum height=1cm, fill=gray!40, anchor=south]

%preamble
\title{Auxiliary is Hard to Spell}
\date{November 14, 2018}
\author{Jed Rembold}

\begin{document}
\renewcommand{\theenumi}{\Alph{enumi}}

\begin{frame}{Announcements}
	\begin{itemize}
		\item Exam 2
			\begin{itemize}
				\item Take-home portion due today at 7pm
				\item Deliver the hard-copy in person or email me a pdf of everything (including the signed cover page!)
			\end{itemize}
		\item Homework
			\begin{itemize}
				\item I'm trying to get Homework 11 out today, due after Thanksgiving
				\item On all the Ch 6 material
				\item Only 1 more homework after that!
			\end{itemize}
	\end{itemize}
\end{frame}

\begin{frame}{\qnum}
	A very long silver (diamagnetic) rod carries a uniformly distributed current $\mcur$ along the $+\zhat$ direction. What is the direction of the bound volume current?
	\begin{columns}
		\column{0.5\textwidth}
		\begin{center}
			\begin{tikzpicture}
				\pic (c) at (0,0) {cyl=green!50!black/1.5/3/3};
				\foreach \x in {-1,0,1} \draw[very thick, -latex, white] (\x,-3) -- +(0,2);
				\node[white] at (.5,-2) {$\cur$};
			\end{tikzpicture}
		\end{center}
		\column{0.5\textwidth}
		\begin{enumerate}
			\item $+\zhat$
			\item \alert<2>{$-\zhat$}
			\item $+\vu*{\phi}$
			\item $-\vu*{\phi}$
		\end{enumerate}
	\end{columns}
\end{frame}

\begin{frame}{\qnum}
	In the same long silver (diamagnetic) rod, what would be the direction of the auxiliary field $\af$ and $\vmag$?
	\begin{columns}
		\column{0.6\textwidth}
		\begin{enumerate}
			\item Both in the $\vu*{\phi}$ direction
			\item Both in the $-\vu*{\phi}$ direction
			\item \alert<2>{$\af$ in $\vu*{\phi}$ direction, but $\vmag$ in $-\vu*\phi$ direction}
			\item $\af$ in $-\vu*{\phi}$ direction, but $\vmag$ in $\vu*\phi$ direction
		\end{enumerate}
		\column{0.3\textwidth}
		\begin{center}
			\begin{tikzpicture}
				\pic (c) at (0,0) {cyl=yellow!50!black/1.5/3/3};
				\foreach \x in {-1,0,1} \draw[very thick, -latex, white] (\x,-3) -- +(0,2);
				\node[white] at (.5,-2) {$\cur$};
			\end{tikzpicture}
		\end{center}
	\end{columns}
\end{frame}

\begin{frame}{\qnum}
	Say you wanted to determine the magnitude of $\af$ inside the rod using an Amperian loop. What would be the most useful loop to draw?
	\begin{center}
		\begin{tikzpicture}
			\coordinate (c) at (0,0);
			\pic at (c) {cyl=green!50!black/1.5/3/3};
			\draw[dashed, very thick] (c)++(1,-1.5) arc (0:360:1cm and .3cm);
			\node<1>[above=1cm, font=\Large] at (c) {A};
			\node<2>[above=1cm, font=\Large, teal] at (c) {A};
		\end{tikzpicture}
		\begin{tikzpicture}
			\coordinate (c) at (0,0);
			\pic at (c) {cyl=cyan!50!black/1.5/3/3};
		\draw[dashed, very thick] (c)++(1,-1) --++ (0,-1.5) --++(-1,0) -- ++(0,1.5) -- cycle;
			\node[above=1cm, font=\Large] at (c) {B};
		\end{tikzpicture}
		\begin{tikzpicture}
			\coordinate (c) at (0,0);
			\pic at (c) {cyl=orange!50!black/1.5/3/3};
		\draw[dashed, very thick] (c)++(1,-1) --++ (0,-1) --++(-2,0) -- ++(0,1) -- cycle;
			\node[above=1cm, font=\Large] at (c) {C};
		\end{tikzpicture}
		\begin{tikzpicture}
			\coordinate (c) at (0,0);
			\pic at (c) {cyl=violet!50!black/1.5/3/3};
		\draw[dashed, very thick] (c)++(.5,-1) --++ (2,0) --++(0,-1) -- ++(-2,0) -- cycle;
			\node[above=1cm, font=\Large] at (c) {D};
		\end{tikzpicture}
	\end{center}
\end{frame}

\begin{frame}{\qnum}
	Take the case of a short, cylindrical iron magnet, which has some baked-in magnetization pointing in the $\zhat$ direction. What can you conclude about the auxiliary field $\af$?
	\begin{enumerate}
		\item $\af = 0$
		\item $\af = \vmag$
		\item $\af = -\vmag$
		\item \alert<2>{None of the above}
	\end{enumerate}
\end{frame}

\begin{frame}{\qnum}
	Take an infinitely long, cylindrical (radius$=R$), iron magnet with a "frozen-in" magnetization given by:
	\[\vmag = M_0 \zhat\]
	What is the magnitude of the magnetic field outside the magnet?
	\begin{enumerate}
		\item \alert<2>{0}
		\item $\mu_0 M_0$
		\item $\mu_0 M_0 R$
		\item $\displaystyle \mu_0 \frac{M_0 R}{s}$
	\end{enumerate}
\end{frame}


%\begin{frame}{\qnum}
	%For linearly magnetizable materials, the relationship between the magnetization and the H-field is
	%\[\vmag = \chi_m \af\]
	%What do you expect the sign of $\chi_m$ to be for a paramagnetic/diamagnetic material?
	%\begin{enumerate}
		%\item Both positive
		%\item Both negative
		%\item Para positive and Dia negative
		%\item Para negative and Dia positive
	%\end{enumerate}
%\end{frame}




\end{document}

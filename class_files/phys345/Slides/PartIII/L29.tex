\documentclass[pdf,aspectratio=169]{beamer}
\usepackage[]{hyperref,graphicx,siunitx,booktabs,lmodern}
\usepackage{physics}
\usepackage{em-commands}
\mode<presentation>{\usetheme{EM}}

%Question Numbering
\newcounter{questionnumber}
\newcommand{\qnum}{%
	\stepcounter{questionnumber}%
	Q\arabic{questionnumber}
}
\resetcounteronoverlays{questionnumber}

\graphicspath{ {../Images/} }

\sisetup{per-mode=symbol}

\tikzstyle{plate}=[draw, very thick, minimum width=4cm, minimum height=1cm, fill=gray!40, anchor=south]

%preamble
\title{Another spell of binding}
\date{November 7, 2018}
\author{Jed Rembold}

\begin{document}
\renewcommand{\theenumi}{\Alph{enumi}}

\begin{frame}{Announcements}
	\begin{itemize}
		\item Homework
			\begin{itemize}
				\item HW10 posted and due on Friday at midnight!
				\item I'm aiming to get HW9 graded tomorrow if I can
				\item I'll grade HW10 on Saturday
			\end{itemize}
		\item Exam 2
			\begin{itemize}
				\item Don't forget Exam 2 is next Monday!
				\item In-class portion similar to Exam 1
				\item Also a small take-home portion due on Wednesday
				\item I'm working on objectives for you to study from
			\end{itemize}
		\item Read at least the start of Chapter 6.3 for Friday
	\end{itemize}
\end{frame}

\begin{frame}{\qnum}
	Suppose a small current loop with magnetic moment pointing to the right is placed at various locations near the end of a large solenoid. At which point is the magnitude of the net force on the dipole greatest? You may or may not find it useful to recall that:
	\[\force = \grad{(\va{m}\vdot\mf)}\]
	\begin{center}
		\begin{tikzpicture}
			\coordinate (cap) at (0,0);
			\def\r{1}
			\def\len{4}
			\draw[thick] (0,0) ellipse (0.5 cm and \r cm);
			\draw[thick] (cap)++(0,\r) -- ++(-\len,0) arc (90:270:0.5cm and \r cm) -- ++(\len,0);
			\foreach \x in {-4.0,-3.5,...,-1}{
				\draw[very thick, Red, ->-=0.5]
					(\x,\r) .. controls +(90:1) and +(-90:1) .. (\x+.45,-\r);
			}
			\node[point,label={above:A}] at (-2,0) {};
			\node[point,label={above:B}] at (0,0) {};
			\node[point,label={above:C}] at (1,0) {};
			\node<1>[point,label={above:D}] at (3,0) {};
			\node<2>[point,Teal,label={[Teal]above:D}] at (3,0) {};
		\end{tikzpicture}
	\end{center}
\end{frame}

\begin{frame}{\qnum}
	Consider a paramagnetic material placed in a uniform external magnetic field $\mf$. The total magnetic field just outside the material is now\ldots
	\begin{enumerate}
		\item smaller than
		\item \alert<2>{larger than}
		\item the same as
	\end{enumerate}
	\ldots it was before the material was placed.
\end{frame}

\begin{frame}{\qnum}
	In our model for diamagnetism, let the angular momentum associated with an orbiting electron point in the $+\zhat$ direction.

	What is the direction of the magnetic moment?
	\begin{enumerate}
		\item $+\zhat$
		\item \alert<2>{$-\zhat$}
		\item $+\xhat$
		\item $-\xhat$
	\end{enumerate}
\end{frame}

\begin{frame}{\qnum}
	In our model for diamagnetism, the electron travels around the ``loop'' in a time of
	\[T = \frac{2\pi R}{v}\]
	What is the magnitude of the magnetic dipole moment for this configuration?
	\begin{enumerate}
		\item $evR$
		\item \alert<2>{$\displaystyle\frac{evR}{2}$}
		\item $evR^2$
		\item $\displaystyle\frac{evR^2}{2}$
	\end{enumerate}
\end{frame}

\begin{frame}{\qnum}
	\begin{columns}
		\column{0.5\textwidth}
		A small chunk of material is placed just above a solenoid. It magnetizes, weakly, as shown by the arrows inside. What kind of material is the cube made of?
		\begin{enumerate}
			\item Diamagnetic
			\item \alert<2>{Paramagnetic}
			\item Ferromagnetic
			\item Biomagnetic
		\end{enumerate}
		
		\column{0.5\textwidth}
		\begin{center}
			\begin{tikzpicture}
				\coordinate (cent) at (0,0);
				\def\r{1.5}
				\pgfmathsetmacro{\a}{\r/4}
				\def\l{4}
				\draw[very thick] (cent) ellipse (\r cm and \a cm)
					(cent)++(\r,0) -- +(0,-\l)
					(cent)++(-\r,0) -- +(0,-\l);
				\foreach \y in {-3.5,-3,...,-.5} {
					\draw[very thick, Red, ->-=0.5] (-\r,\y) ..controls +(200:1) and +(20:1) .. (\r,\y-.5);
				}
				\pic at (-.3,1) {box};
				\foreach \x in {-.5,-.3,...,.6} \draw[ultra thick, orange, -latex] (\x,.9) -- +(0,.8);
			\end{tikzpicture}
		\end{center}
	\end{columns}
\end{frame}

\begin{frame}{\qnum}
	\begin{columns}
		\column{0.5\textwidth}
		\begin{center}
			\begin{tikzpicture}
				\coordinate (cent) at (0,0);
				\def\r{1.5}
				\pgfmathsetmacro{\a}{\r/4}
				\def\l{4}
				\draw[very thick] (cent) ellipse (\r cm and \a cm)
					(cent)++(\r,0) -- +(0,-\l)
					(cent)++(-\r,0) -- +(0,-\l);
				\foreach \y in {-3.5,-3,...,-.5} {
					\draw[very thick, Red, ->-=0.5] (-\r,\y) ..controls +(200:1) and +(20:1) .. (\r,\y-.5);
				}
				\pic at (-.3,1) {box};
				\foreach \x in {-.5,-.3,...,.6} \draw[ultra thick, orange, -latex] (\x,.9) -- +(0,.8);
			\end{tikzpicture}
		\end{center}

		\column{0.5\textwidth}
		Considering the same chunk of material, what force will it feel due to the magnetic field?
		\begin{enumerate}
			\item \alert<2>{Downwards}
			\item Upwards
			\item Out of the page
			\item No force will be felt
		\end{enumerate}
		
	\end{columns}
\end{frame}

%\begin{frame}{\qnum}
	%\begin{columns}
		%\column{0.5\textwidth}
		%\begin{center}
			%\begin{tikzpicture}
				%\pic (c) at (0,0) {cyl=cyan!50!black/1.5/3};
				%\draw[<->] ([xshift=-.4cm]c-lt) -- ([xshift=-.4cm]c-lb) node[left,midway] {$L$};
				%\draw[<->, transform canvas={yshift=-.3cm}] (c-b) -- ($ (c-lt)!(c-b)!(c-lb) $) node[below,midway] {$R$};
				%\foreach \y in {0,-1,-2,-3} \draw[very thick, orange, -latex] (-1,\y) -- +(2,0);
			%\end{tikzpicture}
		%\end{center}
		
		%\column{0.5\textwidth}
		%A solid cylinder has uniform magnetization $\vmag$ throughout the volume in the $x$ direction as shown. What is the magnitude of the total magnetic dipole moment of the cylinder?
		%\begin{enumerate}
			%\item $\pi R^2 L M$
			%\item $2\pi R L M$
			%\item $2\pi R M$
			%\item $\pi R^2 M$
		%\end{enumerate}
	%\end{columns}
%\end{frame}

%\begin{frame}{\qnum}
	%\begin{columns}
		%\column{0.5\textwidth}
		%A solid cylinder has uniform magnetization $\vmag$ throughout the volume in the $z$ direction as shown. Where do bound currents show up?
		%\begin{enumerate}
			%\item All surfaces, but not volume
			%\item Volume only, not on surface
			%\item Top/bottom surface only
			%\item Side (curved) surface only)
		%\end{enumerate}
		%\column{0.5\textwidth}
		%\begin{center}
			%\begin{tikzpicture}
				%\pic (c) (0,0) {cyl=cyan!50!black/1.5/3};
				%\foreach \x in {-1,0,1} \draw[very thick, orange,-latex] (\x,-3) -- +(0,2);
			%\end{tikzpicture}
		%\end{center}
		
	%\end{columns}
%\end{frame}


\end{document}

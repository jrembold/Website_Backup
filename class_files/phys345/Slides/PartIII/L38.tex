\documentclass[pdf,aspectratio=169]{beamer}
\usepackage[]{hyperref,graphicx,siunitx,booktabs,lmodern}
\usepackage{physics}
\usepackage{em-commands}
\mode<presentation>{\usetheme{EM}}

%Question Numbering
\newcounter{questionnumber}
\newcommand{\qnum}{%
	\stepcounter{questionnumber}%
	Q\arabic{questionnumber}
}
\resetcounteronoverlays{questionnumber}

\graphicspath{ {../Images/} }

\sisetup{per-mode=symbol}

%preamble
\title{All is well that Maxwell}
\date{December 7, 2018}
\author{Jed Rembold}

\begin{document}
\renewcommand{\theenumi}{\Alph{enumi}}

\begin{frame}{Announcements}
	\begin{itemize}
		\item Final
			\begin{itemize}
				\item Will be sent out at 5pm today
				\item Due the 14th at 5pm
				\item 6 problems with an EC 7th
				\item Come the moment I send it out, my solutions sets will be locked down (figuratively)
				\item Feel free to ask me questions, but I'll be careful about what I say
					\begin{itemize}
						\item I know it is difficult, but I'd really appreciate if you ask most of these via Campuswire, or ask me in person but then repeat it on Campuswire. That way we can ensure that everyone gets fair access to any questions I might answer.
					\end{itemize}
				\item If you could email me in the near future as to who you are working with on the test, if you are preferring to do it alone, or if you are looking for a partner, I'd like to help make sure everyone is happy with their situation.
			\end{itemize}
	\end{itemize}
\end{frame}

\begin{frame}{\qnum}
	In the boxed region, what can you say about the sign of $\div \vcd$? You can assume that $\mcur\neq 0$.
	\begin{center}
		\begin{tikzpicture}
			\node[block, minimum height=3cm] (l) at (-2,0) {};
			\node[block, minimum height=3cm, fill=blue!50] (r) at (2,0) {};
			\draw[very thick, ->-=.5] (r.east) -- +(2,0) node[right] {$\mcur$};
			\draw[very thick, -<-=.5] (l.west) -- +(-2,0) node[left] {$\mcur$};
			%\draw[very thick,Teal,latex-] (l.east) -- +(45:1) node[right] {Here};
			\draw[very thick, dashed, Red] (l.south west)++(-1,-.5) rectangle +(3,4);
		\end{tikzpicture}
	\end{center}
	\begin{enumerate}
		\item \alert<2>{$\div\vcd < 0$}
		\item $\div\vcd > 0$
		\item $\div\vcd = 0$
		\item Can't say from this information
	\end{enumerate}
\end{frame}

\begin{frame}{\qnum}
	At the indicated position on the parallel plate capacitor, what can you say about the sign of $\pdv{\ef}{t}$? You can assume that $\mcur\neq 0$.
	\begin{center}
		\begin{tikzpicture}
			\node[block, minimum height=3cm] (l) at (-2,0) {};
			\node[block, minimum height=3cm, fill=blue!50] (r) at (2,0) {};
			\draw[very thick, ->-=.5] (r.east) -- +(2,0) node[right] {$\mcur$};
			\draw[very thick, -<-=.5] (l.west) -- +(-2,0) node[left] {$\mcur$};
			\draw[very thick,Teal,latex-] (0,0) -- +(90:1) node[right] {Here};
		\end{tikzpicture}
	\end{center}
	\vspace{-1cm}
	\begin{enumerate}
		\item $\displaystyle \pdv{\ef}{t} < 0$
		\item \alert<2>{$\displaystyle \pdv{\ef}{t} > 0$}
		\item $\displaystyle \pdv{\ef}{t} = 0$
		\item Can't say from this information
	\end{enumerate}
\end{frame}

\begin{frame}{\qnum}
	In the boxed region, what can you say about the sign of $\div\pdv{\ef}{t}$? You can assume that $\mcur\neq 0$.
	\begin{center}
		\begin{tikzpicture}
			\node[block, minimum height=3cm] (l) at (-2,0) {};
			\node[block, minimum height=3cm, fill=blue!50] (r) at (2,0) {};
			\draw[very thick, ->-=.5] (r.east) -- +(2,0) node[right] {$\mcur$};
			\draw[very thick, -<-=.5] (l.west) -- +(-2,0) node[left] {$\mcur$};
			\draw[very thick, dashed, Red] (l.south west)++(-1,-.5) rectangle +(3,4);
		\end{tikzpicture}
	\end{center}
	\vspace{-1cm}
	\begin{enumerate}
		\item $\displaystyle \div\pdv{\ef}{t} < 0$
		\item \alert<2>{$\displaystyle \div\pdv{\ef}{t} > 0$}
		\item $\displaystyle \div\pdv{\ef}{t} = 0$
		\item Can't say from this information
	\end{enumerate}
\end{frame}

\begin{frame}{In Conclusion}
	Maxwell's Equations:
	\begin{align*}
		\div\ef &= \frac{\rho}{\epsilon_0} \\
		\curl\ef &= - \pdv{\mf}{t} \\
		\div\mf &= 0 \\
		\curl\mf &= \mu_0 \vcd + \mu_0\epsilon_0 \pdv{\ef}{t}
	\end{align*}
	Lorentz Force Law:
	\[\force = q(\ef + \va{v}\cp \mf)\]
\end{frame}

\begin{frame}{Finishing Touches}
	\begin{itemize}
		\item<1-> We need a class picture!
		\item<2-> You must evaluate me!
			\begin{itemize}
				\item Evaluation reminders:
					\begin{itemize}
						\item Only use an X to mark a bubble
						\item Completely fill in a bubble only if you changed your mind and want to mark something else (with an X)
						\item Comments are extremely useful to me, especially as this was my first time teaching this course. What did you like? What did you find not useful?
					\end{itemize}
				\item Volunteer:
					\begin{itemize}
						\item Collect all forms, including blanks in folder and seal
						\item Deliver to Smullin 108
					\end{itemize}
			\end{itemize}
		\item<3-> You all are amazing. Thank you so much for a great semester! I can easily say this is the most I've ever enjoyed E\&M!
	\end{itemize}
\end{frame}

\end{document}

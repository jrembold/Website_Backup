\documentclass[pdf, aspectratio=169]{beamer}
\usepackage[]{hyperref,graphicx,siunitx,lmodern,booktabs,tikz,caption}
\usepackage[mode=buildnew]{standalone}
\mode<presentation>{\usetheme{Astro}}
\usepackage{pdfpc-commands}
\usepackage{intro-commands}

\graphicspath{ {../Images/} }

\sisetup{per-mode=symbol}
\usetikzlibrary{calc}
	  
\newcommand{\moon}[1] {
  \coordinate (moon) at (#1:4);
  \node[transform shape] at (moon) {\includegraphics[width=0.8cm]{ch2_moon_small2.png}};
  \fill[black, opacity=0.5] (moon) -- +(0,.4cm) arc (90:270:.39cm) -- cycle;
}
\newcommand{\moonphase}[1] {
  \coordinate (moon) at (-3.5,-3.5);
  \node[transform shape] at (moon) {\includegraphics[width=2cm]{ch2_moon_small2.png}};
  \fill[black, opacity=0.5] (moon) +(0,1cm) arc (90:270:1cm) arc (270:90:#1 cm and 1cm);
}

%preamble
\title{On Seasonal Moonings\ldots}
\date{September 12, 2018}
\author{Jed Rembold}

\begin{document}
\renewcommand*{\theenumi}{\Alph{enumi}}

\begin{frame}{Announcements}
	\begin{itemize}
		\item WebWorK due on Friday!
		\item Starting Ch 4 content today
		\item I'll be aiming to start posting weekly grade reports starting this weekend
		\item Polling: \href{http://rembold-class.ddns.net}{rembold-class.ddns.net}
	\end{itemize}
\end{frame}

\begin{frame}{Review Question}
	In which of the following situations would the gravitational force between the two masses be greatest?
	\begin{columns}
		\column{0.5\textwidth}
		\begin{block}{A}
			\begin{center}
				\begin{tikzpicture}
					\node[ball=cyan,black] (m1) at (0,0) {\SI{10}{\kilo\gram}};
					\node[ball=orange,black] (m2) at (3,0) {\SI{20}{\kilo\gram}};
					\draw[|-|] (0,-1) -- (3,-1) node[midway,above] {\SI{2}{\meter}};
				\end{tikzpicture}
			\end{center}
		\end{block}
		\begin{block}{\alert<2>{C}}
			\begin{center}
				\begin{tikzpicture}
					\node[ball=cyan,black] (m1) at (0,0) {\SI{10}{\kilo\gram}};
					\node[ball=orange,black,minimum size=5mm] (m2) at (2,0) {\SI{10}{\kilo\gram}};
					\draw[|-|] (0,-1) -- (2,-1) node[midway,above] {\SI{1}{\meter}};
				\end{tikzpicture}
			\end{center}
		\end{block}
		\column{0.5\textwidth}
		\begin{block}{B}
			\begin{center}
				\begin{tikzpicture}
					\node[ball=cyan,black] (m1) at (0,0) {\SI{10}{\kilo\gram}};
					\node[ball=orange,black,] (m2) at (4,0) {\SI{20}{\kilo\gram}};
					\draw[|-|] (0,-1) -- (4,-1) node[midway,above] {\SI{3}{\meter}};
				\end{tikzpicture}
			\end{center}
		\end{block}
		\begin{block}{D}
			\begin{center}
				\begin{tikzpicture}
					\node[ball=cyan,black] (m1) at (0,0) {\SI{10}{\kilo\gram}};
					\node[ball=orange,black,minimum size=2cm] (m2) at (4,0) {\SI{30}{\kilo\gram}};
					\draw[|-|] (0,-1) -- (4,-1) node[midway,above] {\SI{3}{\meter}};
				\end{tikzpicture}
			\end{center}
		\end{block}
	\end{columns}
\end{frame}

\begin{frame}{The Fine Art of Throwing Things}
  \begin{itemize}
	\item Say you throw a ball sideways
	  \begin{itemize}
		\item<2-> You force the ball (using your arm) up to some speed
		\item<3-> Inertia keeps the ball moving forward at that speed
		\item<4-> Gravity drags the ball down
	  \end{itemize}
  \end{itemize}
  \begin{center}
	\begin{tikzpicture}
	  \node at (0,0) {\includegraphics[width=3cm]{ch4_thrower.pdf}};
	  %\fill[ball color=orange] (0,0) circle (3pt);
	  \onslide<1>{\fill[ball color=green] (115:1.5) coordinate (p1) circle (3pt);}
	  \onslide<2>{
		  \fill[ball color=green] ($(p1)+(1,0)$) coordinate (p2) circle (3pt);
		\draw[orange, thick,-latex] (p1) -- +(.9,0);
	  }
	  \onslide<3>{
		\fill[ball color=green] ($(p2)+(1,0)$) circle (3pt);
	  }
	  \onslide<4>{
		\foreach[count=\x] \y in {1.8,1.5, 0.9, 0.2, -.6, -1.5}{
		  %\fill[ball color=green, opacity=\x/6] (2+\x,\y) circle (3pt);
		  \fill[ball color=green, opacity=\x/6] ($(p2)+(\x+1,{\y-1.8})$) circle (3pt);
		}
	  }
	\end{tikzpicture}
  \end{center}
\end{frame}

\begin{frame}{One Powerful Arm\ldots}
  \begin{itemize}
	\item Now, if we hiked up a tall mountain and had a REALLY REALLY strong arm\ldots
	\item \href{http://galileoandeinstein.physics.virginia.edu/more_stuff/flashlets/NewtMtn/home.html}{Newton's Mountain}
  \end{itemize}
\end{frame}

\begin{frame}{Moon Orbits}
  \begin{itemize}
	\item So we know that we can get an object into orbit by throwing it really hard in a certain direction
	\item Whatever process formed the Moon must have thus given it this ``initial velocity''
	\item So our Moon's orbit is a result of:
	  \begin{itemize}
		\item That velocity
		\item Inertia
		\item and the pull of gravity
	  \end{itemize}
  \end{itemize}
  \begin{flushright}
	\vspace{-2cm}
	\begin{tikzpicture}
	  \node at (0,0) {\includegraphics[width=1cm]{world.png}};
	  \draw[dashed, help lines] (0,0) circle (2cm);
	  \coordinate (moon) at (135:2cm);
	  \draw[-latex, green, thick] (moon) -- node[midway,sloped,above] {\scriptsize Inertia} +(225:1cm);
	  \fill[ball color=black!20] (moon) circle(1mm);
	  \foreach \a in {45,90,...,360}{
		\draw[-latex, orange, thick] (\a:1.8) -- (\a:1.1);
	  }
	  \node[align=center,orange, font=\scriptsize] at (2,-2) {Gravity pulls\\Moon toward\\Earth};
	\end{tikzpicture}
  \end{flushright}
\end{frame}


%\begin{frame}{Cavendish Experiment}
  %\begin{itemize}
	%\item A method to measure a tiny amount of gravitational force to determine $G$
  %\end{itemize}
  %\begin{center}
	%\begin{tikzpicture}
	  %\onslide<1>{
		%\draw[LOrange, line width=3pt] (2,0) -- (-2,0);
		%\fill[ball color=LOrange] (2,0) circle (2mm);
		%\fill[ball color=LOrange] (-2,0) circle (2mm);
		%\draw[Blue,line width=4pt] (0,2) -- (0,-2);
		%\fill[ball color=Blue] (0,2) circle (5mm);
		%\fill[ball color=Blue] (0,-2) circle (5mm);
		%\draw[line width=4pt] (4pt,2mm) -- (4pt,-2mm);
		%\draw[red,thick] (6pt,0) --+(-20:3cm);
	  %}
	  %\onslide<2>{
		%\begin{scope}[rotate=60]
		  %\draw[LOrange, line width=3pt] (2,0) -- (-2,0);
		  %\fill[ball color=LOrange] (2,0) circle (2mm);
		  %\fill[ball color=LOrange] (-2,0) circle (2mm);
		%\end{scope}
		%\begin{scope}[rotate=-5]
		  %\draw[Blue,line width=4pt] (0,2) -- (0,-2);
		  %\fill[ball color=Blue] (0,2) circle (5mm);
		  %\fill[ball color=Blue] (0,-2) circle (5mm);
		  %\draw[line width=4pt] (4pt,2mm) -- (4pt,-2mm);
		%\end{scope}
		%\draw[dashed,red!30,thick] (6pt,0) --+(-20:3cm);
		%\draw[red,thick] (6pt,0) --+(-25:3cm) node[xshift=5mm,right,fg,align=left, font=\scriptsize] {Movement and\\oscillations\\used to determine\\gravitational constant};
	  %}
	  %\draw[rotate=19, fill=black!20, rounded corners] (2.7cm,-2mm) rectangle (4.0cm,2mm);
	  %\draw[red,thick] (6pt,0) --+(20:2.5cm);
	%\end{tikzpicture}
  %\end{center}
%\end{frame}

\begin{frame}{Newton meets Kepler's 3rd}
  \begin{itemize}
	\item Recall that Kepler had worked out that
	  \[\frac{a^3}{p^2}= \text{ same value for all planets orbiting Sun}\]
	  \begin{itemize}
		\item \alert{This is why we need to use AU and year units, because those describe Earth's $a$ and $p$!}
	  \end{itemize}
	\item<2> Newton worked out from theory that two objects held in orbit by gravity would obey:
	  \[\frac{a^3}{p^2} = \frac{G \left( M_1 + M_2 \right)}{4\pi^2}\]
	  where
	  \vspace{-2mm}
	  \scriptsize
	  \begin{align*}
		M_1, M_2 &= \text{ masses of objects in \underline{kilograms}}\\
		a &= \text{ average separation of objects in \underline{meters}}\\
		p &= \text{ orbital period in \underline{seconds}}\\
		G &= \text{ gravitational constant in \underline{SI Units}}
	  \end{align*}
  \end{itemize}
\end{frame}

\begin{frame}{A Nicer Newton's Formulation}
  \begin{itemize}
	\item Put into more everyday units, Newton's formulation boils down to:
		\[\frac{a^3}{p^2} = (M_1 + M_2)\]
	  where
	  \vspace{-2mm}
	  \begin{align*}
		  M_1, M_2 &= \text{ masses of objects in \alert{solar masses}}\\
		a &= \text{ average separation of objects in \alert{AU}}\\
		p &= \text{ orbital period in \alert{years}}\\
	  \end{align*}
  \item For the Sun and Earth, $M_1 + M_2 \approx 1$
  \end{itemize}
\end{frame}

\begin{frame}{Newton and Kepler's 3rd Law Example}
  \begin{example}
	Suppose our Sun suddenly grew to have 30 times it's current mass. Assuming we stayed the same distance away, how long would our new year be?
	\begin{itemize}
	  \item $a = \SI{1}{AU}$
	  \item $M_{earth} + M_{sun} \approx M_{sun} = 1M_\odot$
	\end{itemize}
	\onslide<2>{
	  \begin{align*}
		p &= \sqrt{\frac{a^3}{\left( 30M_\odot \right)}}\\
		&= \SI{0.18257}{yrs}\\
		&= \SI{66.5}{days}
	  \end{align*}
	}
  \end{example}
\end{frame}

\begin{frame}{Angular Momentum}
  \begin{itemize}
	\item Related to how much rotational inertia an object has
	\item For an object moving in a circle, the \alert{angular momentum} is:
	  \[L = m\times v \times r\]
	\item The only way to change an objects angular momentum is to apply an angular force (called a torque)
  \end{itemize}
  \begin{center}
	\begin{tikzpicture}
	  \draw[help lines, dashed] (0,0) circle (2cm);
	  \node[draw, circle, orange, fill=LOrange, text=black] (m) at (45:2) {$m$};
	  \draw[thick] (0,0) -- node[midway, fill=Background] {$r$} (m);
	  \draw[very thick, cyan, -latex] (m) --+(-45:1) node[below] {$v$};
	\end{tikzpicture}
  \end{center}
\end{frame}

\begin{frame}{Figure Skating}
	\begin{center}
	  \inlineMovie{../Videos/SkaterSpin.ogv}{../Videos/SkaterSpin.jpg}{width=.8\textwidth}
	\end{center}
\end{frame}

\begin{frame}{Summary!}
  \begin{itemize}
	  \item Force is a measure of how something's motion will change
	  \item The gravitational force increases with mass and decreases rapidly with distance
	  \item Newton refined Kepler's 3rd law to account for the masses of the orbiting objects
	  \item Angular momentum is related to an object's rotational inertia, and is difficult to change
	  \item Decreases in radius will result in a faster rotating object
  \end{itemize}
\end{frame}

\fullFrameImage{ch4_seasons.png}

\begin{frame}{Concept Question}
  Seasons are determined by the proximity of Earth to the sun. Summer when we are closer and Winter when we are further away.
  \begin{enumerate}
	\item True
	\item False
  \end{enumerate}
\end{frame}

\begin{frame}{The Reason for the Season}
  \begin{itemize}
	\item Summer in Northern Hemisphere = Winter in Southern
	  \begin{itemize}
		\item Both would be the same if seasons were based on distance
	  \end{itemize}
  \end{itemize}
  \pause
  \begin{alertblock}{Fact!}
	Seasons are a result of the Earth's tilt.
  \end{alertblock}
\end{frame}

\begin{frame}{The Seasons}
  \begin{center}
	\begin{tikzpicture}[scale=0.95]
	  %\fill[black!70] (-10,-5) rectangle (10,5);
	  \draw[help lines, dashed] (0,0) ellipse (4cm and 2cm);
	  %\node at (0,0) {\includegraphics[width=2cm]{sun.pdf}};
	  \shade[inner color=yellow, outer color=orange] (0,0) circle (1cm);
	  \fill[black, rounded corners] (-3,-1) rectangle +(-3,-2);
	  \draw[white] (-5,-1.4) -- (-5,-2.6);
	  \node at (-5,-2) {\includegraphics[width=1cm]{world.png}};
	  
	  \onslide<1>{%Fall Equinox
		\coordinate (pos) at (0,-2);
		\begin{scope}[rotate around={-23:(pos)}, transform shape]
		  \draw[white] (pos) -- +(0,.6);
		  \draw[white] (pos) -- +(0,-.6);
		  \node at (pos) {\includegraphics[width=1cm]{world.png}};
		\end{scope}
		\node[below, yshift=-5mm] at (pos) {Fall Equinox};
		\foreach \y in {-2.6,-2.3,...,-1.3}{
		  \draw[yellow, -latex] (-3.5, \y) -- +(180:1);
		}
	  }
	  \onslide<2>{%Winter Equinox
		\coordinate (pos) at (4,0);
		\begin{scope}[rotate around={-23:(pos)}, transform shape]
		  \draw[white] (pos) -- +(0,.6);
		  \draw[white] (pos) -- +(0,-.6);
		  \node at (pos) {\includegraphics[width=1cm]{world.png}};
		\end{scope}
		\node[right, align=center, xshift=5mm] at (pos) {Winter\\Solstice};
		\fill[red, opacity=0.3] (-5.53,-2) arc (-180:0:5.3mm) arc (0:-180:5.3mm and 2mm);
		\begin{scope}[rotate around={-23:(-5,-2)}]
		\foreach \y in {-2.6,-2.3,...,-1.3}{
		  \draw[yellow, -latex] (-3.5, \y) -- +(180:1);
		}
		\end{scope}
	  }
	  \onslide<3>{%Spring Equinox
		\coordinate (pos) at (0,2);
		\begin{scope}[rotate around={-23:(pos)}, transform shape]
		  \draw[white] (pos) -- +(0,.6);
		  \draw[white] (pos) -- +(0,-.6);
		  \node at (pos) {\includegraphics[width=1cm]{world.png}};
		\end{scope}
		\node[above, yshift=5mm] at (pos) {Spring Equinox};
		\begin{scope}[rotate around={0:(-5,-2)}]
		\foreach \y in {-2.6,-2.3,...,-1.3}{
		  \draw[yellow, -latex] (-3.5, \y) -- +(180:1);
		}
		\end{scope}
	  }
	  \onslide<4>{%Summer Equinox
		\coordinate (pos) at (-4,0);
		\begin{scope}[rotate around={-23:(pos)}, transform shape]
		  \draw[white] (pos) -- +(0,.6);
		  \draw[white] (pos) -- +(0,-.6);
		  \node at (pos) {\includegraphics[width=1cm]{world.png}};
		\end{scope}
		\node[left, align=center, xshift=-5mm] at (pos) {Summer\\Solstice};
		\fill[red, opacity=0.3] (-5.53,-2) arc (180:0:5.3mm) arc (0:-180:5.3mm and 2mm);
		\begin{scope}[rotate around={23:(-5,-2)}]
		\foreach \y in {-2.6,-2.3,...,-1.3}{
		  \draw[yellow, -latex] (-3.5, \y) -- +(180:1);
		}
		\end{scope}
	  }
	\end{tikzpicture}
  \end{center}
\end{frame}

\begin{frame}{The Seasons}
  \begin{columns}
	\column{.5\textwidth}
	\begin{itemize}
	  \item More direct Sun heats better
	  \item Each position marks the start of our ``seasons''
	  \item Sun's position in the local sky changes as well
		\begin{itemize}
		  \item Highest on Summer Solstice
		  \item Lowest on Winter Solstice
		  \item Rises \emph{exactly} in the East and sets \emph{exactly} in the West on the Equinox's
		\end{itemize}
	\end{itemize}
	\column{.5\textwidth}
	\begin{figure}[h!]
	  \centering
	  \includegraphics[width=\textwidth]{ch2_SunPos.jpg}
	\end{figure}
  \end{columns}
\end{frame}

\begin{frame}{Alien Seasons}
  \begin{itemize}
	\item Earth's seasons
	  \begin{itemize}
		\item \alert{Planet Tilt}
		\item Orbital variations
		\item Distance differences due to tilt
	  \end{itemize}
	\item Pluto's seasons
	  \begin{itemize}
		\item Planet Tilt
		\item \alert{Orbital variations}
		\item Distance differences due to tilt
	  \end{itemize}
	\item TrES-4 (1.8x size of Jupiter, 0.05 AU from it's parent star)
	  \begin{itemize}
		\item Planet Tilt
		\item Orbital variations
		\item \alert{Distance differences due to tilt}
	  \end{itemize}
  \end{itemize}
\end{frame}

\begin{frame}{Our Wobbly Planet}
  \begin{columns}
	\column{.5\textwidth}
	\begin{itemize}
	  \item Just like a spinning top, Earth has a bit of a wobble
	  \item Relative to the celestial sphere, the direction of our North Pole oscillates every \SI{26000}{yrs}
	  \item ``North Star'' changes!
	\end{itemize}
	\column{.5\textwidth}
	\begin{figure}[h!]
	  \centering
	  \includegraphics[width=.9\textwidth]{ch2_pole_precess.jpg}
	\end{figure}
  \end{columns}
\end{frame}

%\begin{frame}{Let's Talk the Moon}
  %Two obvious observations we are interested in explaining:
  %\begin{itemize}
	%\item Why do we see the Moon going through phases?
	%\item What causes eclipses?
  %\end{itemize}
%\end{frame}

%\begin{frame}{Lunar Facts}
  %\begin{itemize}
	%\item Moon orbits Earth in \SI{27.3}{\day}
	%\item Lunar Phases repeat every \SI{29.5}{\day}
	  %\begin{itemize}
		%\item Difference due to Earth moving about the sun during this time
	  %\end{itemize}
	%\item Sunlight hits Earth and Moon at same angle because the Sun is so far away
	%\item Moon's phase is also closely related to its rising and setting times
	%\item The Man in the Moon is always watching
	  %\begin{itemize}
		%\item We always see the same face of the moon
	  %\end{itemize}
  %\end{itemize}
%\end{frame}

%\begin{frame}{It's just a phase}
  %\begin{center}
	%\begin{tikzpicture}[scale=.6]
	  %%\draw[help lines] (-5,-5) grid (5,5);
	  %\node[transform shape] at (0,0) {\includegraphics[width=1.44cm]{world.png}};
	  %\fill[black,opacity=0.5] (0,0) +(0,0.72) arc (90:270:0.72) -- cycle;
	  %\draw[help lines, dashed] (0,0) circle (4cm);
	  %\foreach \y in {-5,-4,...,5}{
		%\draw[yellow, -latex] (6,\y) -- +(-1,0);
	  %}
	  %\fill[black,rounded corners] (-5,-5) rectangle (-2,-2);
	  %\node at (-3.5,-5.25) {Observer View};
	  %\onslide<1>{
		%\moon{0};
		%\moonphase{-1};
		%\fill[red] (0:0.8) circle (1mm);
	  %}
	  %\onslide<2>{
		%\moon{45};
		%\moonphase{-.5};
		%\fill[red] (90:0.8) circle (1mm);
	  %}
	  %\onslide<3>{
		%\moon{90};
		%\moonphase{-0};
		%\fill[red] (90:0.8) circle (1mm);
	  %}
	  %\onslide<4>{
		%\moon{135};
		%\moonphase{.5};
		%\fill[red] (180:0.8) circle (1mm);
	  %}
	  %\onslide<5>{
		%\moon{180};
		%\moonphase{1};
		%\fill[red] (180:0.8) circle (1mm);
	  %}
	%\end{tikzpicture}
  %\end{center}
%\end{frame}

%\begin{frame}{Wax on, Wane off}
  %\begin{itemize}
	%\item Waxing - the moon is becoming more full
	  %\begin{itemize}
		%\item Visible afternoon/evening
	  %\end{itemize}
	%\item Waning - the moon is becoming less full
	  %\begin{itemize}
		%\item Visible late night / morning
	  %\end{itemize}
	%\item In Northern Hemisphere:
	  %\begin{itemize}
		%\item Light on Right $\Rightarrow$ waxing
		%\item Opposite in Southern Hemisphere
	  %\end{itemize}
  %\end{itemize}
  %\begin{center}
	%\begin{tikzpicture}[scale=.7, transform shape]
	  %%\draw[help lines] (0,0) grid (12,4);
	  %\node at (5.5,4.5) {\Large Waxing Moons};
	  %\fill[black!80!blue, rounded corners] (0,0) rectangle +(3,4);
	  %\fill[black!85!blue, rounded corners] (4,0) rectangle +(3,4);
	  %\fill[black!90!blue, rounded corners] (8,0) rectangle +(3,4);
	  %\node (north) at (1.5,2) {\includegraphics[width=2cm]{ch2_moon_small2.png}};
	  %\node (equator) at (5.5,2) {\includegraphics[width=2cm]{ch2_moon_small2.png}};
	  %\node (south) at (9.5,2) {\includegraphics[width=2cm]{ch2_moon_small2.png}};
	  %\fill[black, opacity=0.7] (north) +(0,1) arc (90:270:1) arc (-90:90:0.5cm and 1cm);
	  %\fill[black, opacity=0.7, rotate around={-90:(equator)}] (equator) +(0,1) arc (90:270:1) arc (-90:90:0.5cm and 1cm);
	  %\fill[black, opacity=0.7, rotate around={180:(south)}] (south) +(0,1) arc (90:270:1) arc (-90:90:0.5cm and 1cm);
	  %\node at (1.5,.5) {\SI{+45}{\degree N}};
	  %\node at (5.5,.5) {Equator};
	  %\node at (9.5,.5) {\SI{+45}{\degree S}};
	%\end{tikzpicture}
  %\end{center}
%\end{frame}

%\begin{frame}{Don't Turn Your Back!}
  %Why do we only see one side of the moon?
  %\begin{itemize}
	%\item If the moon did not spin as it orbited us, we'd see all parts of the moon
	%\item If the moon spun either quickly or slowly, we'd see all parts of the moon
	%\item But if it spun \emph{jussst riighht\ldots}
  %\end{itemize}
%\end{frame}

%\begin{frame}{Synchronous Spinning}
  %\begin{center}
	%\begin{tikzpicture}[scale=.9]
	  %%\draw[help lines] (0,0) grid (10,6);
	  %\coordinate (earth) at (5,3);
	  %\node at (earth) {\includegraphics[width=1cm]{world.png}};
	  %\draw[help lines, dashed] (earth) circle (3cm);
	  %\foreach \t in {0,45,...,360}{
		%\begin{scope}[rotate around={\t:(earth)}, transform shape]
		  %\coordinate (moon) at (8,3);
		  %\node at (moon) {\includegraphics[width=1cm]{ch2_moon_small2.png}};
		  %\draw[red] (moon) +(-.6,0) node {N} +(.6,0) node {F};
		  %\draw[red,-latex] (moon) +(.3,0) arc (0:180:.3);
		%\end{scope}
	  %}
	%\end{tikzpicture}
  %\end{center}
%\end{frame}

%\begin{frame}{Libration!}
  %In truth, we actually can see upwards of 60\% of the Moon's surface over a year
  %\begin{itemize}
	%\item \alert{Libration} describes this oscillating wobble
	%\item Libration in longitude due to the Moon's orbit not being perfectly circular
	%\item Libration in latitude due to the Moon's tilt (similar to Earth's seasons!)
	%\item Diurnal libration do to an observer getting two different perspectives from each side of the Earth daily
  %\end{itemize}
%\end{frame}

%\begin{frame}{The Extra Bits}
  %\begin{figure}[h!]
	%\centering
	%\includegraphics[width=.7\textwidth]{ch2_libration.jpg}
  %\end{figure}
%\end{frame}

%\begin{frame}{Understanding Check}
  %Imagine you built yourself a lovely house on the surface of the moon. One day you are talking to your mom back on Earth and she mentions that it is a lovely full moon tonight. From your vantage point on the moon, what is Earth's phase?
  %\begin{enumerate}
	%\item \alert<2>{New}
	%\item Waxing
	%\item Full
	%\item Waning
  %\end{enumerate}
%\end{frame}

%%\begin{frame}{Eclipse (of the non-Twilight variety)}
  %%All good things cast shadows!
  %%\begin{figure}[h!]
	%%\centering
	%%\begin{tikzpicture}
	  %%%\node[anchor=south west] at (0,0) {\includegraphics[width=\textwidth]{eclipse.png}};
	  %%\node[anchor=south west] at (0,0) {\includestandalone{../Images/Standalone/ch2_eclipse_fig}};
	  %%\draw[white,-latex] (5,3) node[above] {penumbra} -- (9,2);
	  %%\draw[white,-latex] (6,0) node[below] {umbra} -- (8.5,1.3);
	  %%%\draw[white] (0,0) grid (10,3);
	%%\end{tikzpicture}
  %%\end{figure}
%%\end{frame}

%%\begin{frame}{Eclipse Facts}
  %%\begin{itemize}
	%%\item Lunar eclipses can \emph{only} occur when the Moon is full!
	  %%\begin{itemize}
		%%\item Otherwise the Earth isn't between the Moon and the Sun
	  %%\end{itemize}
	%%\item Lunar eclipses can be penumbral, partial, or total, depending on what part of the shadow overlaps the moon
  %%\end{itemize}
  %%\begin{center}
	%%\begin{tikzpicture}
	  %%\node[anchor=south west] at (0,0) {\includestandalone[width=\textwidth]{../Images/Standalone/ch2_eclipse_fig}};
	  %%\onslide<1>{
		%%\node at (12.5,2.8) {\includegraphics[width=3mm]{ch2_moon_small2.png}};
		%%\node at (5,0) {Penumbral Eclipse};
	  %%}
	  %%\onslide<2>{
		%%\node (moon) at (12.5,2.5) {\includegraphics[width=3mm]{ch2_moon_small2.png}};
		%%\fill[black!70!red, opacity=0.5] (moon) ++(-.15,0) arc (180:350:1.5mm) -- cycle;
		%%\node at (5,0) {Partial Eclipse};
	  %%}
	  %%\onslide<3>{
		%%\node (moon) at (12.5,2.0) {\includegraphics[width=3mm]{ch2_moon_small2.png}};
		%%\fill[black!50!red, opacity=0.5] (moon) circle (1.5mm);
		%%\node at (5,0) {Total Eclipse};
	  %%}
	  %%%\draw[white] (0,0) grid (10,3);
	%%\end{tikzpicture}
  %%\end{center}
%%\end{frame}

\end{document}

\documentclass[pdf,aspectratio=169]{beamer}
\usepackage[]{hyperref,graphicx,siunitx,booktabs,lmodern}
\usepackage{physics}
\usepackage{em-commands}
\mode<presentation>{\usetheme{EM}}

%Question Numbering
\newcounter{questionnumber}
\newcommand{\qnum}{%
	\stepcounter{questionnumber}%
	Q\arabic{questionnumber}
}
\resetcounteronoverlays{questionnumber}

\graphicspath{ {../Images/} }

\sisetup{per-mode=symbol}

\tikzstyle{plate}=[draw, very thick, minimum width=4cm, minimum height=1cm, fill=gray!40, anchor=south]

%preamble
\title{Biot-Savart: Annoying to do and spell}
\date{October 26, 2018}
\author{Jed Rembold}

\begin{document}
\renewcommand{\theenumi}{\Alph{enumi}}

\begin{frame}{Announcements}
	\begin{itemize}
		\item Homework 8 due on Monday!
		\item I'm still working on grade reports (sorry)
		\item Physics Open House today at 3pm!
			\begin{itemize}
				\item Be social (I know\ldots) and recruit more physics majors!
			\end{itemize}
		\item Read 5.3 for Monday
	\end{itemize}
\end{frame}

\begin{frame}{\qnum}
	A ``ribbon'' (width $a$) of surface current flows with surface current density $\scd$. Right next to it is a second identical ribbon of current. Viewed collectively, what is the new total surface current density?
	\begin{columns}
		\column{0.5\textwidth}
		\begin{enumerate}
			\item 0
			\item $2\scd$
			\item $\scd/2$
			\item \alert<2>{Something else\onslide<2>{ ($\scd$)}}
		\end{enumerate}
		\column{0.5\textwidth}
		\begin{center}
			\tdplotsetmaincoords{70}{110}
			\begin{tikzpicture}[tdplot_main_coords]
				\draw[very thick, fill=blue!20, xyplane=0] (0,0) rectangle +(4,2);
				\draw[very thick, fill=blue!20, xyplane=0] (0,2) rectangle +(4,2);
				\draw[Red, very thick, -latex, xyplane=0] (1,1) -- +(2,0);
				\draw[Red, very thick, -latex, xyplane=0] (1,3) -- +(2,0);
			\end{tikzpicture}
		\end{center}
		
	\end{columns}
\end{frame}

\begin{frame}{\qnum}
	Which of the following is a statement of charge conservation?
	\begin{enumerate}
		\item $\displaystyle \pdv{\rho}{t} = - \oint \vcd \vdot d\vA$
		\item \alert<2>{$\displaystyle -\pdv{Q}{t} = \oint \vcd \vdot d\vA$}
		\item $\displaystyle \pdv{\rho}{t} = -\grad{\vcd}$
		\item $\displaystyle \pdv{Q}{t} = \int \div{\vcd} \, d\tau$
	\end{enumerate}
\end{frame}

\begin{frame}{\qnum}
	A ribbon of width $a$ with uniform surface current density $\scd$ passes through a uniform magnetic field $\mf$. Only the length $b$ along the ribbon is in the field. What is the magnitude of the force on the ribbon?
	\begin{columns}
		\column{0.5\textwidth}
		\begin{center}
			\begin{tikzpicture}
				\draw[fill=cyan!40] (0,0) rectangle +(5,1);
				\draw[very thick, Red, -latex] (1,.5) -- +(3,0) node[right,math] {\scd};
				\draw[<->] (-.1,0) -- +(0,1) node[midway,left,math] {a};
				\draw[<->] (1,-.1) -- +(3,0) node[midway,below,math] {b};
				\foreach \x in {1,1.5,...,4} \draw[green!50!black,very thick,-latex] (\x,1.2) -- +(0,2);
				\node[green!50!black,math] at (2.5,3.5) {\mf};
			\end{tikzpicture}
		\end{center}
		\column{0.5\textwidth}
		\begin{enumerate}
			\item $\displaystyle \frac{bKB}{a}$
			\item $aKB$
			\item \alert<2>{$abKB$}
			\item $\displaystyle \frac{KB}{ab}$
		\end{enumerate}
	\end{columns}
\end{frame}

\begin{frame}{\qnum}
	What is the direction of the infinitesimal contribution $d\mf$ at point P created by current in $d\vell$?
	\begin{columns}
		\column{0.5\textwidth}
		\begin{enumerate}
			\item Up
			\item Up and to the right
			\item \alert<2>{Into the page}
			\item Out of the page
		\end{enumerate}
		\column{0.5\textwidth}
		\begin{center}
			\begin{tikzpicture}
				\draw[double arrow=1.5mm colored by black and Background] (0,0) -- +(-5,0) node[left,math] {\cur};
				\draw[|<->|, thick] (-4,-.3) -- +(.5,0) node[midway,below] {$d\ell$};
				\node[point,pin={45:Origin}] at (-1,0) {};
				\node[point,label={above:P}] at (-2,2) {};
			\end{tikzpicture}
		\end{center}
	\end{columns}
\end{frame}

\begin{frame}{\qnum}
	What is the magnitude of $\displaystyle \frac{d\vell\cross\srhat}{\srmag^2}$?
	\begin{columns}
		\column{0.5\textwidth}
		\begin{center}
			\begin{tikzpicture}
				\draw[double arrow=1.5mm colored by black and Background] (0,0) -- +(-5,0) node[left,math] {\cur};
				\draw[|<->|, thick] (-4,-.3) -- +(.5,0) node[midway,below] {$d\ell$};
				\node[point,pin={45:Origin}] at (-1,0) {};
				\node[point,label={above:P}] (p) at (-2,2) {};
				\draw[teal, very thick, -latex] (-3.75,.2) -- (p) node[midway,sloped,above,math,black] {\sr} coordinate[pos=.3] (ang);
				\draw[<->, thick] (ang) arc (45:5:1) node[midway,math,fill=Background, inner sep=0pt] {\theta};
			\end{tikzpicture}
		\end{center}
		\column{0.5\textwidth}
		\begin{enumerate}
			\item \alert<2>{$\displaystyle \frac{d\ell \sin\theta}{\srmag^2}$}
			\item $\displaystyle \frac{d\ell \sin\theta}{\srmag^3}$
			\item $\displaystyle \frac{d\ell \cos\theta}{\srmag^2}$
			\item $\displaystyle \frac{d\ell \cos\theta}{\srmag^3}$
		\end{enumerate}
	\end{columns}
\end{frame}

\begin{frame}{\qnum}
	What is the value of $\displaystyle \mathcal{I}\frac{d\vell\cross\srhat}{\srmag^2}$?
	\begin{columns}
		\column{0.3\textwidth}
		\begin{enumerate}
			\item $\displaystyle \frac{\mathcal{I} y dx_s}{\left(x_s^2 + y^2\right)^{3/2}}\zhat$
			\item $\displaystyle \frac{\mathcal{I} x_s dx_s}{\left(x_s^2 + y^2\right)^{3/2}}\yhat$
			\item $\displaystyle \frac{-\mathcal{I} x_s dx_s}{\left(x_s^2 + y^2\right)^{3/2}}\yhat$
			\item \alert<2>{$\displaystyle \frac{-\mathcal{I} y dx_s}{\left(x_s^2 + y^2\right)^{3/2}}\zhat$}
		\end{enumerate}
		\column{0.5\textwidth}
		\begin{center}
			\begin{tikzpicture}
				\draw[thick, -latex] (0,0) -- +(3,0) node[right] {$\xhat$};
				\draw[thick, -latex] (0,0) -- +(0,3) node[above] {$\yhat$};
				\draw[line width=4pt, Red, -stealth, opacity=0.5] (2.6,0) -- (-3,0);
				\node[point, label={right:(0,y,0)}] (p) at (0,2.5) {};
				\draw[-latex, ultra thick] (-2,0) -- (-2.5,0) node[below,midway] {$d\ell$};
				\draw[teal, very thick, -latex] (-2,0) -- (p) node[midway,sloped,above,black] {$\sr$} coordinate[pos=.3] (ang);
				\draw[<->, thick] (ang) arc (45:5:1) node[midway,math,fill=Background,inner sep=0pt] {\theta};
			\end{tikzpicture}
		\end{center}
	\end{columns}
\end{frame}

%\begin{frame}{\qnum}
	%What do you expect for the direction of $\mf$ at point P? How about the direction of $d\mf$ at point P generated \textbf{only} by the segment of current $d\vell$ in red?
	%\begin{center}
		%\begin{tikzpicture}[use Hobby shortcut]
			%\draw[very thick, ->-=.5] (0,0) .. (2,1) .. (4,1) .. (6,3) coordinate[pos=.7] (r1) coordinate[pos=.9] (r2);
			%\draw[line width=2pt, Red, opacity=.75, -latex] (r1) -- (r2);
			%\node[point,label={below:P}] at (3,0) {};
		%\end{tikzpicture}
	%\end{center}
	%\begin{enumerate}
		%\item $\mf$ in the plane of the page, $d\mf$ in plane of page
		%\item $\mf$ into page, $d\mf$ into page
		%\item $\mf$ into page, $d\mf$ out of page
		%\item $\mf$ out of page, $d\mf$ out of page
	%\end{enumerate}
%\end{frame}

%\begin{frame}{\qnum}
	%\begin{columns}
		%\column{0.5\textwidth}
		%What is the magnitude of $\displaystyle \frac{d\vell\cross\srhat}{\srmag^2}$?
		%\begin{enumerate}
			%\item $\displaystyle \frac{d\ell \sin\phi}{z^2}$
			%\item $\displaystyle \frac{a\, d\ell}{z^2}$
			%\item $\displaystyle \frac{d\ell \sin\phi}{z^2 + a^2}$
			%\item $\displaystyle \frac{d\ell}{z^2+a^2}$
		%\end{enumerate}
		%\column{0.5\textwidth}
		%\begin{center}
			%\tdplotsetmaincoords{70}{100}
			%\begin{tikzpicture}[tdplot_main_coords,scale=1.25]
				%\draw[ -latex] (0,0,0) -- +(3,0,0) node[left]{$\xhat$};
				%\draw[ -latex] (0,0,0) -- +(0,3,0) node[right]{$\yhat$};
				%\draw[ -latex] (0,0,0) -- +(0,0,3) node[above]{$\zhat$};
				%\draw[ultra thick,->-=.9, xyplane=0] (2,0) arc (0:-360:2) coordinate[pos=.85] (l);
				%\node[below] at (l) {$\cur$};
				%\node[point,label={left:$(0,0,z)$}](p) at (0,0,2.5) {};
				%\draw[cyan,-latex,very thick,xyplane=0] (0,0) -- +(30:2) coordinate(dl);
				%\draw[<->, xyplane=0] (0,1) arc (90:30:1) node[midway,fill=Background,font=\small, inner sep=1pt] {$\phi$};
				%\draw[teal,-latex,very thick] (dl) -- (p) node[pos=.6,right,math,black]{\sr};
				%\draw[<->, shorten <>=1mm] (0,0,0) -- +(0,-2,0) node[below,math,midway] {a};
			%\end{tikzpicture}
		%\end{center}
		
	%\end{columns}
%\end{frame}


\end{document}

\documentclass[pdf,aspectratio=169]{beamer}
\usepackage[]{hyperref,graphicx,siunitx,booktabs,lmodern}
\usepackage{physics}
\usepackage{em-commands}
\mode<presentation>{\usetheme{EM}}

%Question Numbering
\newcounter{questionnumber}
\newcommand{\qnum}{%
	\stepcounter{questionnumber}%
	Q\arabic{questionnumber}
}
\resetcounteronoverlays{questionnumber}

\graphicspath{ {../Images/} }

\sisetup{per-mode=symbol}

\tikzstyle{plate}=[draw, very thick, minimum width=4cm, minimum height=1cm, fill=gray!40, anchor=south]

%preamble
\title{The Potential of becoming a Vector}
\date{November 2, 2018}
\author{Jed Rembold}

\begin{document}
\renewcommand{\theenumi}{\Alph{enumi}}

\begin{frame}{Announcements}
	\begin{itemize}
		\item Homework
			\begin{itemize}
				\item Homework 9 is due on Monday!
				\item Homework 10 is going to be super short (like 1 or 2 problems) and will be due a week from \emph{today}
			\end{itemize}
		\item Physics Tea at 3!
		\item Physics Seminar today on Laser Fusion!!
			\begin{itemize}
				\item 3:30pm in Collins 318
			\end{itemize}
		\item Read Chapter 6.1 for Monday
	\end{itemize}
\end{frame}

\begin{frame}{\qnum}
	For an infinite solenoid of radius $R$, with current $\mcur$, and $n$ turns per unit length, what would be a correct way of writing $\vcd$?
	\begin{enumerate}
		\item $\vcd = n\mcur \vu*{\phi}$
		\item \alert<2>{$\vcd = n\mcur\delta(r-R) \vu*{\phi}$}
		\item $\vcd = \frac{\mcur}{n}\delta(r-R) \vu*{\phi}$
		\item $\vcd = \mu_0 n \mcur\delta(r-R) \vu*{\phi}$
	\end{enumerate}
\end{frame}

\begin{frame}{\qnum}
	What is required in order to define a vector potential where:
	\[\laplacian\vA = -\mu_0 \vcd\]
	\begin{enumerate}
		\item $\curl\vA = 0$
		\item \alert<2>{$\div\vA = 0$}
		\item $\div\vA = \curl\vA$
		\item $\vA \rightarrow 0$ at $\infty$
	\end{enumerate}
\end{frame}

\begin{frame}{\qnum}
	What flexibility do you have in defining the vector potential, given the Coulomb gauge ($\div\vA=0$)? That is, what can $\vA_2$ be that gives us the same $\mf$? Here $C$ or $\va{C}$ is an arbitrary scalar or vector function.
	\begin{enumerate}
		\item $\vA_2 = \vA + C$
		\item $\vA_2 = \vA + \va{C}$
		\item \alert<2>{$\vA_2 = \vA + \grad{C}$}
		\item $\vA_2 = \vA + \div{\va{C}}$
	\end{enumerate}
\end{frame}

\begin{frame}{\qnum}
	Assuming $\vcd$ goes to 0 at $\infty$, we can calculate $\vA$ in Cartesian using:
	\[\vA(\pos) = \frac{\mu_0}{4\pi}\int \frac{\vcd(\pos_s)}{\srmag}d\tau_s\]
	Can this integral also be done in spherical coordinates?
	\begin{enumerate}
		\item Yes, no problem
		\item \alert<2>{Yes, $r_s$ can be spherical but $\vcd$ needs to be in Cartesian components}
		\item Yes, $\vcd$ can be spherical, but $r_s$ needs to be in Cartesian components
		\item No, this will not work due to cross terms in the spherical Laplacian
	\end{enumerate}
\end{frame}

\begin{frame}{\qnum}
	\begin{columns}
		\column{0.4\textwidth}
		Assuming the Coulomb gauge, the vector potential $\vA$ due to a long straight wire with current $\cur$ along the $z$-axis points in what direction?
		\begin{enumerate}
			\item \alert<2>{$\zhat$}
			\item $\vu*{\phi}$
			\item $\vu*{s}$
		\end{enumerate}
		\column{0.5\textwidth}
		\begin{center}
			\begin{tikzpicture}
				\draw[very thick, ->-=.7, Red] (0,0) -- +(0,4) node[pos=.7,right] {$\cur$};
				\draw[thick, -latex] (0,4.5) -- +(0,1) node[above] {$\zhat$};
			\end{tikzpicture}
		\end{center}
	\end{columns}
\end{frame}

%\begin{frame}{\qnum}
	%What is the physical interpretation of $\oint \vA \vdot d\vell$?
	%\begin{enumerate}
		%\item The current density $\vcd$
		%\item The magnetic field $\mf$
		%\item The magnetic flux through a surface
		%\item It has no particular physical interpretation
	%\end{enumerate}
%\end{frame}

%\begin{frame}{\qnum}
	%Consider a square loop of height $H$ and length $\ell$ enclosing some amount of magnetic field. What happens to $\Phi_B$ as $H$ becomes vanishingly small?
	%\begin{enumerate}
		%\item $\Phi_B$ stays constant
		%\item $\Phi_B$ gets smaller but doesn't vanish
		%\item $\Phi_B$ goes to 0
	%\end{enumerate}
%\end{frame}

\end{document}

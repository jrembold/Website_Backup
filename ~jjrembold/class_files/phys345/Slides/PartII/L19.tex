\documentclass[pdf,aspectratio=169]{beamer}
\usepackage[]{hyperref,graphicx,siunitx,booktabs,lmodern}
\usepackage{physics}
\usepackage{em-commands}
\mode<presentation>{\usetheme{EM}}

%Question Numbering
\newcounter{questionnumber}
\newcommand{\qnum}{%
	\stepcounter{questionnumber}%
	Q\arabic{questionnumber}
}
\resetcounteronoverlays{questionnumber}

\graphicspath{ {../Images/} }

\sisetup{per-mode=symbol}

%preamble
\title{Casting a Binding}
\date{October 10, 2018}
\author{Jed Rembold}

\begin{document}
\renewcommand{\theenumi}{\Alph{enumi}}

\begin{frame}{Announcements}
	\begin{itemize}
		\item YOU HAVE A TEST ON FRIDAY
			\begin{itemize}
				\item I'm trying to get your HW6's graded
				\item Most old homework solutions are complete in my binder if you want to check some of your ungraded problems against them
					\begin{itemize}
						\item I only have the one binder and it may be in high demand, so try to not monopolize it!
					\end{itemize}
				\item Learning objectives are posted on Campuswire
				\item You are probably looking at something like 3 calculation type problems and then a handful of conceptual problems
				\item Bring everything. Your book, your notes, your homework, your calculator.
			\end{itemize}
		\item I'll be around until about 6:30 tomorrow if you have questions
		\item There is a post category for ``Exam Preparation'' on Campuswire if you are having more questions!
		\item AOE (Physics Club) meeting at 4:30pm on Thursday to solve the world's problems
	\end{itemize}
\end{frame}

\begin{frame}{\qnum}
	Are $\rho_b$ and $\sigma_b$ due to real charges?
	\begin{enumerate}
		\item Nope! They are just fictitious charges we use to describe the summed dipoles!
		\item \alert<2>{Yes! Actual charges are living at those locations!}
		\item What are $\rho_b$ and $\sigma_b$?
	\end{enumerate}
\end{frame}

\begin{frame}{\qnum}
	In the following case, is the bound surface and volume charge density zero or nonzero?
	\begin{center}
		\begin{tikzpicture}
			\coordinate (mid) at (0,0);
			\foreach \x in {-1,0,1}{
				\foreach \y in {-1,0,1}{
					\coordinate (s) at ($(mid)+(\x,\y)$);
					\fill[cyan, opacity=.5] (s) circle (4mm);
					\fill[cyan] (s) circle (.5mm);
					\node[ball=red, minimum size=2mm] at ($(s)+(0,.3)$) {};
				}
			}
			\node at ([yshift=-2cm]mid) {Physical Dipoles};
			\coordinate (mid) at (5,0);
			\foreach \x in {-1,0,1}{
				\foreach \y in {-1,0,1}{
					\coordinate (s) at ($(mid)+(\x,\y)$);
					\draw[ultra thick,-latex, Red] ([yshift=-3mm]s) --+(0,.5);
				}
			}
			\node at ([yshift=-2cm]mid) {Ideal Dipoles};
		\end{tikzpicture}
	\end{center}
	\begin{enumerate}
		\centering
		\item $\sigma_b = 0 \qc \rho_b \neq 0$
		\item $\sigma_b \neq 0 \qc \rho_b \neq 0$
		\item $\sigma_b = 0 \qc \rho_b = 0$
		\item \alert<2>{$\sigma_b \neq 0 \qc \rho_b = 0$}
	\end{enumerate}
\end{frame}

\begin{frame}{\qnum}
	In the following case, is the bound surface and volume charge density zero or nonzero?
	\begin{center}
		\begin{tikzpicture}
			\coordinate (mid) at (0,0);
			\foreach \x in {-1,0,1}{
				\foreach \y in {-1,0,1}{
					\coordinate (s) at ($(mid)+(\x,\y)$);
					\fill[cyan, opacity=.5] (s) circle (4mm);
					\fill[cyan] (s) circle (.5mm);
					\node[ball=red, minimum size=2mm] at ($(s)+(0,.2+.1*\y)$) {};
				}
			}
			\node at ([yshift=-2cm]mid) {Physical Dipoles};
			\coordinate (mid) at (5,0);
			\foreach \x in {-1,0,1}{
				\foreach \y in {-1,0,1}{
					\coordinate (s) at ($(mid)+(\x,\y)$);
					\draw[ultra thick,-latex, Red] ([yshift=-3mm]s) --+(0,.4+.1*\y);
				}
			}
			\node at ([yshift=-2cm]mid) {Ideal Dipoles};
		\end{tikzpicture}
	\end{center}
	\begin{enumerate}
		\centering
		\item $\sigma_b = 0 \qc \rho_b \neq 0$
		\item \alert<2>{$\sigma_b \neq 0 \qc \rho_b \neq 0$}
		\item $\sigma_b = 0 \qc \rho_b = 0$
		\item $\sigma_b \neq 0 \qc \rho_b = 0$
	\end{enumerate}
\end{frame}

\begin{frame}{\qnum}
	A dielectric slab (top area $A$ and height $h$) has been polarized with $\va{P}=P_0\zhat$. What is the surface charge density, $\sigma_b$, on the bottom surface?
	\begin{columns}
		\column{0.3\textwidth}
		\begin{enumerate}
			\item 0
			\item \alert<2>{$-P_0$}
			\item $P_0$
			\item $P_0Ah$
		\end{enumerate}
		
		\column{0.5\textwidth}
		\begin{center}
			\begin{tikzpicture}
				\draw pic (b) {box=Red/4/2/2};
				\draw[|-|] ($(b-fbl)+(-.3,0)$) --+ (0,2) node[midway,left,math] {h};
				\node[] at (2,2,1) {A};

				\coordinate (o) at (0,0,4);
				\begin{scope}[font=\scriptsize]
					\path[thick, -latex] (o) edge node[at end,right,math] {\yhat} +(1,0,0) 
					(o) edge node[at end,left,math] {\zhat} +(0,1,0) 
					(o) edge node[at end,left,math] {\xhat} +(0,0,1);
				\end{scope}
			\end{tikzpicture}
		\end{center}
	\end{columns}
\end{frame}

\begin{frame}{\qnum}
	A dielectric sphere is uniformly polarized,
	\[\va{P} = +P_0 \zhat\]
	What is the surface charge density?
	\begin{columns}
		\column{0.5\textwidth}
		\begin{center}
			\begin{tikzpicture}
				\node[ball=cyan, minimum size=3cm] at (0,0) {};
				\draw[very thick, latex-] (0,1) -- (0,-.5) node[below,math] {P_0};

				\coordinate (o) at (0,0,5);
				\begin{scope}[font=\scriptsize]
					\path[thick, -latex] (o) edge node[at end,right,math] {\yhat} +(1,0,0) 
					(o) edge node[at end,left,math] {\zhat} +(0,1,0) 
					(o) edge node[at end,left,math] {\xhat} +(0,0,1);
				\end{scope}
			\end{tikzpicture}
			
		\end{center}
		\column{0.5\textwidth}
		\begin{enumerate}
			\item 0
			\item $C$ (but $C\neq 0$)
			\item $C\sin\theta$
			\item \alert<2>{$C\cos\theta$}
		\end{enumerate}
	\end{columns}
\end{frame}

\begin{frame}{\qnum}
	At the end of last class we had a cylinder of radius $a$ and height $b$ that had its base at the origin and was aligned along the $z$ axis. This cylinder had a polarization of 
	\[\va{P} = P_0 \left(\frac{z}{b}\right) \zhat\]
	What is the bound volume charge density? (\emph{Bonus! Also, what is the bound charge density on each cap and the sides?})
	\begin{enumerate}
		\item 0
		\item $\displaystyle \frac{P_0}{b}$
		\item \alert<2>{$\displaystyle -\frac{P_0}{b}$}
		\item $\displaystyle -\frac{P_0}{b^2}$
	\end{enumerate}
\end{frame}

\begin{frame}{Test Questions}
	Any questions or clarifications in preparation for the test?
\end{frame}








\end{document}

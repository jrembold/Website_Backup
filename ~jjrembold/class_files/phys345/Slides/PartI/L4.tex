\documentclass[pdf,aspectratio=169]{beamer}
\usepackage[]{hyperref,graphicx,siunitx,lmodern,booktabs}
\usepackage{physics}
\usepackage{em-commands}
\mode<presentation>{\usetheme{EM}}

\graphicspath{ {../Images/} }

\sisetup{per-mode=symbol}

%preamble
\title{Getting Amped for Coulomb}
\date{September 5, 2018}
\author{Jed Rembold}

\begin{document}
\renewcommand{\theenumi}{\Alph{enumi}}

\begin{frame}{Announcements}
	\begin{itemize}
		\item Homework 2
			\begin{itemize}
				\item Posted yesterday, due Monday night
				\item For extra notebook work, please save as a pdf and then combine with your handwritten work for a single pdf to submit
					\begin{itemize}
						\item Can use online sites like \url{https://combinepdf.com/} if you need
					\end{itemize}
			\end{itemize}
		\item If you are late on HW1, just remember you have 14 cumulative grace days!
		\item Friday: Bring your computers again for visualization day!
	\end{itemize}
\end{frame}

\begin{frame}{}
	I'd say I spent \rule{1cm}{1pt} hours on HW1.
	\begin{enumerate}
		\item 1-3 hours
		\item 3-6 hours
		\item 6-9 hours
		\item 9+ hours
	\end{enumerate}
\end{frame}

\begin{frame}{}
	5 charges $q$ are arranged in a regular pentagon as shown. What is the electric field at the center?
	\begin{center}
		\begin{tikzpicture}
			\foreach \a in {90,162, ..., 450} \draw (\a:1) -- (\a+72:1);
			\foreach \a in {90,162, ..., 450} \node[circle, draw, inner sep=1pt,fill=white] at (\a:1) {q};
			\node[point] at (0,0) {};
		\end{tikzpicture}
	\end{center}
	\begin{enumerate}
		\item \alert<2>{Zero, and I can show it mathematically}
		\item Zero, but I'm less confident with the math
		\item Nonzero, and I can show it mathematically
		\item Nonzero, but I'm less confident with the math
	\end{enumerate}
\end{frame}

\begin{frame}{}
	Suppose now we removed the topmost charge. Now what is the electric field at the center of the pentagon?
	\begin{center}
		\begin{tikzpicture}
			\foreach \a in {90,162, ..., 450} \draw (\a:1) -- (\a+72:1);
			\foreach \a in {162,234, ..., 420} \node[circle, draw, inner sep=1pt,fill=white] (\a) at (\a:1) {q};
			\node[point] (mid) at (0,0) {};
			\draw[<->] (mid) -- (162) node[midway,above,sloped,math] {a};

			\draw[-latex,thick] (2,0) -- + (1,0) node[right,math] {\xhat};
			\draw[-latex,thick] (2,0) -- + (0,1) node[right,math] {\yhat};
		\end{tikzpicture}
	\end{center}
	\begin{enumerate}
		\item $\frac{4}{4\pi\epsilon_0}\frac{q}{a^2}\yhat$
		\item \alert<2>{$\frac{1}{4\pi\epsilon_0}\frac{q}{a^2}\yhat$}
		\item $-\frac{1}{4\pi\epsilon_0}\frac{q}{a^2}\yhat$
		\item I need longer to work out this math
	\end{enumerate}
\end{frame}

\begin{frame}{}
	Say we want to find the electric field at point $P$ due to the line of charge to the left. Breaking it up into chunks of length $d\ell$, what is the value of $\srmag$?
	\begin{columns}
		\column{0.5\textwidth}
		\begin{center}
			\begin{tikzpicture}
				\node[point, label={below:P}] (P) at (4,0) {};
				\draw[fill=gray!30] (0,0) rectangle +(.2,3);
				\draw[fill=Teal] (0,1.5) rectangle +(.2,.4);

				\draw[-latex] (0,0) -- (P) node[below,midway,math] {\pos};
				\draw[-latex] (-.1,0) -- +(0,1.5) node[left,midway,math] {\pos_s};
				\draw[<->] (-.2,1.5) -- +(0,.4) node[left,midway,math] {d\ell_s};
				\draw[-latex] (.2,1.7) -- (P) node[above,sloped,midway,math] {\sr};

				\draw[-latex, thick] (0,-2) -- +(1,0) node[right,math] {\xhat};
				\draw[-latex, thick] (0,-2) -- +(0,1) node[right,math] {\yhat};
			\end{tikzpicture}
		\end{center}
		
		\column{0.5\textwidth}
		\begin{enumerate}
			\item $x$
			\item $y_s$
			\item $\sqrt{d\ell_s^2 + x^2}$
			\item \alert<2>{$\sqrt{x^2 + y_s^2}$}
		\end{enumerate}
		
	\end{columns}
\end{frame}

\begin{frame}{}
	Given that, in terms of $\sr$,
	\[\va{E}(\pos) = \int \frac{\lambda d\ell_s}{4\pi\epsilon_0 \srmag^3}\sr\]
	what is $\va{E}_x(x,0,0)$ (as a function of $x$)?
	\begin{enumerate}
		\item $\displaystyle \frac{\lambda}{4\pi\epsilon_0} \int \frac{ x}{x^3}dy_s $
		\item \alert<2>{$\displaystyle \frac{\lambda}{4\pi\epsilon_0} \int \frac{ x}{(x^2+y_s^2)^{3/2}} dy_s$}
		\item $\displaystyle \frac{\lambda}{4\pi\epsilon_0} \int \frac{ y_s}{(x^2+y_s^2)^{3/2}}dy_s$
		\item Something else
	\end{enumerate}
\end{frame}

\begin{frame}{}
	How would you go about solving this integral, if you didn't have computational assistance?
	\[\frac{\lambda}{4\pi\epsilon_0} \int \frac{ x}{(x^2+y_s^2)^{3/2}} dy_s\]
	\begin{enumerate}
		\item $u$ substitution
		\item \alert<2>{Trig substitution}
		\item Integration by parts
		\item Throwing up your hands in despair
	\end{enumerate}
\end{frame}


\begin{frame}{}
	How do people feel about Taylor series for working out limiting behavior?
	\begin{enumerate}
		\item I remember them and am comfortable with them
		\item I remember them, but am not particularly comfortable with them
		\item I've definitely used them before, but I don't recall how they work
		\item I am alarmed. What is a Taylor Series?!
	\end{enumerate}
\end{frame}

\begin{frame}{Taylor Series! {\scriptsize (playing a hunch here\ldots)}}
	\begin{itemize}
		\item The Taylor series of a function $f$ about some point $a$ is
			\[f(a) + \frac{f^\prime(a)}{1!}(x-a) + \frac{f^{\prime\prime}(a)}{2!}(x-a)^2 + \frac{f^{\prime\prime\prime}(a)}{3!}(x-a)^3 + \cdots\]
		\item Most commonly looking at when a variable is small, so $a$ commonly is 0
		\item Binomial Approximation (Special Taylor)
			\begin{itemize}
				\item Rewrite equation into form $(1+x)^\alpha$ where $x$ is small
				\item Then $\displaystyle (1+x)^\alpha \approx 1 + \alpha x$
			\end{itemize}
	\end{itemize}
\end{frame}

\begin{frame}{}
	What is the limiting behavior of the function
	\[f(x) = \frac{x}{\sqrt{x^2 + a^2}}\]
	for huge values of $x$? (Just focusing on $x$ dependency here, you can ignore any $a$'s floating around)
	\begin{enumerate}
		\item $\displaystyle 1 + x$
		\item 1
		\item $\displaystyle 1-\frac{1}{x}$
		\item \alert<2>{$\displaystyle 1-\frac{1}{x^2}$}
	\end{enumerate}
\end{frame}

\begin{frame}{}
	So, what do you expect to happen to the field as you get really far from the rod?
	\[E_x = \frac{\lambda}{4\pi\epsilon_0}\frac{L}{x\sqrt{x^2+L^2}}\]
	\begin{enumerate}
		\item $E_x$ will go to 0
		\item $E_x$ begins to look like a point charge
		\item $E_x$ goes to $\infty$
		\item \alert<2>{More than one of these is true}
	\end{enumerate}
\end{frame}

\end{document}

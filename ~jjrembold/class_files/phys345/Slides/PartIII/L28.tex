\documentclass[pdf,aspectratio=169]{beamer}
\usepackage[]{hyperref,graphicx,siunitx,booktabs,lmodern}
\usepackage{physics}
\usepackage{em-commands}
\mode<presentation>{\usetheme{EM}}

%Question Numbering
\newcounter{questionnumber}
\newcommand{\qnum}{%
	\stepcounter{questionnumber}%
	Q\arabic{questionnumber}
}
\resetcounteronoverlays{questionnumber}

\graphicspath{ {../Images/} }

\sisetup{per-mode=symbol}

\tikzstyle{plate}=[draw, very thick, minimum width=4cm, minimum height=1cm, fill=gray!40, anchor=south]

%preamble
\title{Raw Magnetism}
\date{November 5, 2018}
\author{Jed Rembold}

\begin{document}
\renewcommand{\theenumi}{\Alph{enumi}}

\begin{frame}{Announcements}
	\begin{itemize}
		\item Homework
			\begin{itemize}
				\item Homework 9 is due tonight!
				\item I'm getting the short HW10 posted today so you can start looking at it. Due Friday at midnight.
			\end{itemize}
		\item Some homework requests:
			\begin{itemize}
				\item Make it very clear when you start a new part of the problem (a,b,c,etc).
				\item Make absolutely sure that the quality of your image/scan gives work that is easy to read.
				\item When you submit multiple pages for a problem, please make sure they are in the proper order. You can drag them around to reorder them.
				\item Part of what makes Jupyter notebooks so nice is that you can easily annotate and comment on them to describe what is being done. A notebook without comments, headings or explanations is basically on par with an unlabeled plot for what it conveys. And might start being scored accordingly.
			\end{itemize}
		\item Read Chapter 6.2 for Wednesday
	\end{itemize}
\end{frame}

\begin{frame}{\qnum}
	What is the physical interpretation of $\oint \vA \vdot d\vell$?
	\begin{enumerate}
		\item The current density $\vcd$
		\item The magnetic field $\mf$
		\item \alert<2>{The magnetic flux through a surface}
		\item It has no particular physical interpretation
	\end{enumerate}
\end{frame}

\begin{frame}{\qnum}
	Consider a square loop of height $H$ and length $\ell$ enclosing some amount of magnetic field. What happens to $\Phi_B$ as $H$ becomes vanishingly small?
	\begin{enumerate}
		\item $\Phi_B$ stays constant
		\item $\Phi_B$ gets smaller but doesn't vanish
		\item \alert<2>{$\Phi_B$ goes to 0}
	\end{enumerate}
\end{frame}

\begin{frame}{\qnum}
	When you do a vector potential multipole expansion for magnetic fields, what is the monopole term?
	\begin{enumerate}
		\item $\displaystyle \frac{\mu_0 R}{4\pi}$
		\item $\displaystyle \frac{\mu_0\pi R}{2}$
		\item \alert<2>{$0$}
		\item Something else
	\end{enumerate}
\end{frame}

\begin{frame}{\qnum}
	\begin{columns}
		\column{0.5\textwidth}
		What is the magnitude of the magnetic dipole moment corresponding to the current loop on the right? You can assume a current $\mcur$ is flowing through the loop.
		\begin{enumerate}
			\item $3\ell\mcur$
			\item $\ell^2\sqrt{3}\mcur$
			\item \alert<2>{$\displaystyle \frac{\ell^2\sqrt{3}}{4}\mcur$}
			\item $\displaystyle \frac{3\ell\mcur}{2}$
		\end{enumerate}
		
		\column{0.5\textwidth}
		\begin{center}
			\begin{tikzpicture}
				\draw[very thick, ->>-=0.2 to 1 by 0.33] (0,0) -- (0:3) -- (60:3) -- cycle;
				\draw[<->] (0,-.2) -- +(3,0) node[below,midway] {$\ell$};
			\end{tikzpicture}
		\end{center}
		
	\end{columns}
\end{frame}

\begin{frame}{\qnum}
	Two currents (equal in magnitude) are oriented in three different ways.
	Which ones will produce a dipole field at a far distance from the currents?
	\begin{center}
		\tdplotsetmaincoords{70}{100}
		\begin{tikzpicture}[tdplot_main_coords,scale=1.2]
			\draw[very thick, ->>-=.25 to 1 by 0.25, xyplane=0] (0,0) -| (1,2) -| (0,0);
			\draw[very thick, ->>-=.25 to 1 by 0.25, xyplane=0] (1.5,0) -| +(1,2) -| (1.5,0);
			\node[font=\Large\bf] at (0,1,1) {i};
		\end{tikzpicture}
		\hspace{1cm}
		\begin{tikzpicture}[tdplot_main_coords,scale=1.2]
			\draw[very thick, ->>-=.25 to 1 by 0.25, xyplane=0] (0,0) -| (1,2) -| (0,0);
			\draw[very thick, ->>-=.15 to 1 by 0.25, xyplane=0] (1.5,0) |- +(1,2) |- (1.5,0);
			\node[font=\Large\bf] at (0,1,1) {ii};
		\end{tikzpicture}
		\hspace{1cm}
		\begin{tikzpicture}[tdplot_main_coords,scale=1.2]
			\draw[very thick, ->>-=.25 to 1 by 0.25, xyplane=0] (0,0) -| (1,2) -| (0,0);
			\draw[very thick, ->>-=.20 to 1 by 0.25, xzplane=-.5] (.5,0) -| +(1,2) -| (.5,0);
			\node[font=\Large\bf] at (0,.5,1) {iii};
		\end{tikzpicture}
	\end{center}
	\begin{enumerate}
		\item 1 only
		\item 1 and 2 only
		\item \alert<2>{1 and 3 only}
		\item 2 and 3 only
	\end{enumerate}
\end{frame}

\begin{frame}{\qnum}
	A current-carrying wire loop is in a constant magnetic field $\mf=B\zhat$ as shown. What is the direction of the torque on the loop?
	\begin{center}
		\tdplotsetmaincoords{70}{110}
		\begin{tikzpicture}[tdplot_main_coords,scale=.6]
			\draw[-latex] (-3,0,0) -- (3,0,0) node[left] {$\xhat$};
			\draw[-latex] (0,-3,0) -- (0,3,0) node[right] {$\yhat$};
			\draw[-latex] (0,0,-3) -- (0,0,3) node[above] {$\zhat$};
			\draw[very thick,->>-=0.06 to 1 by .5] (-1,-1,2) -- ++(2,0,0) -- ++(0,2,-4) -- ++(-2,0,0) -- cycle;
		\end{tikzpicture}
		\hspace{2cm}
		\begin{tikzpicture}
			\node[font=\Large, inner sep=.5pt](in) at (-30:1.2) {$\otimes$};
			\node[font=\Large, inner sep=.5pt](out) at (180-30:1.2) {$\odot$};
			\draw[dashed] (in) -- (out);
			\draw[thick, -latex] (0,.5) -- +(0,1) node[midway,right,math] {\mf};
			\draw[-latex] (-1,-2) -- +(1,0) node[right,math] {\yhat};
			\draw[-latex] (-1,-2) -- +(0,1) node[above,math] {\zhat};
		\end{tikzpicture}
	\end{center}
	\vspace{-1cm}
	\begin{enumerate}
		\item \alert<2>{$+\xhat$}
		\item $+\yhat$
		\item $+\zhat$
		\item Something else
	\end{enumerate}
\end{frame}

%\begin{frame}{\qnum}
	%Suppose a small current loop is placed at various locations near the end of a large solenoid. At which point is the magnitude of the force on the dipole greatest? You may or may not find it useful to recall that:
	%\[\force = \grad{(\va{m}\vdot\mf)}\]
	%\begin{center}
		%\begin{tikzpicture}
			%\coordinate (cap) at (0,0);
			%\def\r{1}
			%\def\len{4}
			%\draw[thick] (0,0) ellipse (0.5 cm and \r cm);
			%\draw[thick] (cap)++(0,\r) -- ++(-\len,0) arc (90:270:0.5cm and \r cm) -- ++(\len,0);
			%\foreach \x in {-4.0,-3.5,...,-1}{
				%\draw[very thick, Red, ->-=0.5]
					%(\x,\r) .. controls +(90:1) and +(-90:1) .. (\x+.45,-\r);
			%}
			%\node[point,label={above:A}] at (-2,0) {};
			%\node[point,label={above:B}] at (0,0) {};
			%\node[point,label={above:C}] at (1,0) {};
			%\node[point,label={above:D}] at (3,0) {};
		%\end{tikzpicture}
	%\end{center}
%\end{frame}

%\begin{frame}{\qnum}
	%Consider a paramagnetic material placed in a uniform external magnetic field $\mf$. The total magnetic field just outside the material is now\ldots
	%\begin{enumerate}
		%\item smaller than
		%\item larger than
		%\item the same as
	%\end{enumerate}
	%\ldots it was before the material was placed.
%\end{frame}

%\begin{frame}{\qnum}
	%In our model for diamagnetism, the electron travels around the ``loop'' in a time of
	%\[T = \frac{2\pi R}{v}\]
	%What is the magnitude of the magnetic dipole moment for this configuration?
	%\begin{enumerate}
		%\item $evR$
		%\item $\displaystyle\frac{evR}{2}$
		%\item $evR^2$
		%\item $\displaystyle\frac{evR^2}{2}$
	%\end{enumerate}
%\end{frame}

\end{document}
